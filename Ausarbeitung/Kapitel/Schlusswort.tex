\chapter{Zusammenfassung und Ausblick}

Diese Arbeit implementiert die auf Marken basierte Objekterkennung bzw. Verfolgung. Die Testobjekte werden mit Schwarz gef�rbt und mit wei�en Marken auf der Oberfl�che markiert. Eine PMD-Kamera beobachtet die ganze Szene von oben und liefert direkt die 3D-Daten. Das gesamte Programm kann in zwei Teile, Lernen und Wiedererkennung, zusammengefasst werden. Im Lernprozess wird ein Fremdobjekt unter der Kamera gezeigt. Die Marken des Objekts werden von dem Programm erkannt und in einem Strukturgraph eingef�gt, was die r�umliche Struktur des Objekt beschreiben kann. Der Strukturgraph wird nach dem Lernen als eine VTK-Daten gespeichert. Wenn es mehr Objekte im Eingabebildstrom gibt, wird die Markenmenge f�r jedes Objekt zuerst segmentiert. Am Anfang der Wiedererkennungsphase werden die vorhandenen Strukturgraphen eingelesen, was als die Eingabemodelle der Wiedererkennung definiert werden. F�r jedes Eingabebild werden alle Marken zuerst erkannt, genau wie es beim Lernen durchgef�hrt wird. Die erkannten Marken werden dann in unterschiedlichen Kandidaten des Objekts aufgeteilt. Ein Eingabemodel ist wiedererkannt, genau dann, wenn zumindest ein Kandidat existiert, der den gleichen Teilgraph zu diesem Eingabemodel erh�lt. Die aktuelle Orientierung und Lage des Objekt kann danach durch die Korrespondenzpunktepaare bestimmt werden. 

\section{Verbesserungen}
In dieser Arbeit werden viele vorhandene Ideen und Algorithmen verwendet. Bez�glich dieser Vorkenntnisse werden weiterhin viele Verbesserungen gemacht, z.B. die Helligkeitssteuerung, das neue Singul�rwertzerlegungsverfahren und die Bildsteuerung usw.. Dadurch k�nnen die Ergebnisse des Lernens und der Wiedererkennung stabiler und schneller erhalten werden. Trotzdem gibt es auch einigen Probleme in der Realisierung. 

\section{Problem}
Die existierenden Problems k�nnen in zwei Bereichen, Stabilit�t und Zeitaufwand, zusammengefasst werden.

\subsection{Stabilit�t}
Die Gr��e und Position der Marken muss exakt entworfen werden, damit die richtigen Erkennungsergebnisse erhalten werden k�nnen. Sonst ist das Objekt schwierig zu erkennen. Aus diesem Grund sind die Lange bzw. die Breite des Objekts stark beschr�nkt. Die Anordnung der Marken beeinflusst auch die Quote der erfolgreichen Erkennungen in der Wiedererkennungsphase. 

\subsection{Zeitaufwand}
Wie was in Tabellen~\ref{LZ} bis \ref{22} gezeigt, erf�llt der Zeitaufwand aber nur teilweise die Echtzeitbedingung. In der Lernphase kann das Programm aber nur die Framerate bis zu 18.6 fps erreichen. In der Wiedererkennungsphase h�ngt die Framerate von der Anzahl der betrachteten Eingabeobjekt und Eingabemodel ab. Wenn es nur einen Eingabemodel betrachtet wird, kann das Programm im Durchschnitt 30 Bilder pro Sekunde bearbeiten. F�r zwei Eingabemodel wird diese Zahl aber leider sofort auf 20 sinken. 

\section{M�gliche L�sungsverfahren}
Entweder die Stabilit�t oder der Zeitaufwand h�ngt stark von dem Detektor ab. Deshalb ist Verwendung einer neuen, besseren Detektor eine M�glichkeit, um die vorherigen Problems zu l�sen. Dieser Detektor soll bessere Genauigkeit und Stabilit�t haben. Auf dem idealen Fall sollen alle Marken eindeutlich erkannt werden k�nnen, und die Ergebnisse unabh�ngig von der Helligkeit der Eingabebilder sein. Dadurch kann die Anzahl der Schleifen in Helligkeitssteuerung stark reduziert und die Kombination der Merkmalen in Markenerkennung sogar komplett entfernt werden. Diese Vereinfachungen des Programm k�nnen mehr als die H�lfte der gesamten Laufzeit sparen, was in dem Kreisdiagramm in Abbildung~\ref{LZP} deutlich gezeigt wird. Die bessere Stabilit�t des Detektors fordert an, dass die Erkennungsergebnisse nicht empfindlich f�r die Abst�nde zwischen den Marken sein sollen. Damit k�nnen die Marken nicht genau mit den Regeln vom Abschnitt~\ref{Ma} angebracht werden.
\\
\\
Au�erdem ist die Umschreibung des Programm mit CUDA die andere M�glichkeit zur Abnahme der Laufzeit, wegen der st�rkeren Berechnungsf�higkeit �ber die Gleitkommazahl der GPU. 


%Das zweite Beschleunigungsverfahren ist gleich wie Vorher: ein besserer Markendetektor zu finden. Das Kreisdiagramm~\ref{LZP} zeigt deutlich, dass die Teilprogramme ,,CenSurE Detektor und Helligkeitssteuerung'' und ,,Markenerkennung'' mehr als Halb der gesamten Laufzeit brauchen. Diese beide Teile h�ngen von dem Markendetektor stark ab. Wenn ein besserer Detektor verwendet wird, kann die Anzahl der Schleifen in Helligkeitssteuerung stark reduziert. Die Kombination der Merkmalen in Markenerkennung, die der 1 zu 1 Korrespondenz zwischen der durch Detektor erkannten Merkmale und der realen Marken anpasst, kann aber auch komplett entfernt. Diese Vereinfachung der Programms k�nnen die Laufzeit erfolgreich erniedrigen.    

%Die Oberfl�chen der Zielobjekt werden mit retroreflektierenden Marker markiert.  Die totale Laufzeit kann in zwei Phasen zusammengefasst werden. In der Initialisierungsphase wird die Fremdobjekt unter der Kamera gezeigt und die Marker darauf sollen herausgekannt und im System gespeichert werden. Wenn es mehr Objekte gibt, wird eine Kalibrierung am Anfang durchgef�hrt. Mills in \cite{MN00} hat eine kompakte Segmentierung der Bewegung mithilfe des sogenannten \glqq Feature Interval Graph\grqq  dargestellt. \cite{AJ05} erweitert die Arbeit von Mills. Ein auf Pyramide basiertes Clustering-Verfahren wird statt des alten auf Dreiecke basierten Clustering-Verfahren vorgeschlagen. Nach der Initialisierungsphase wird das Objekt aus dem Gesichtsfeld der Kamera verschoben. Die Erkennungsphase f�ngt genau an, wenn das gleiche Objekt wieder unter der Kamera eingebracht wird. Die reflektierende Marker sollen nochmal gesammelt werden und ein von \cite{AJ05} repr�sentierter Teilgraph-Tracker wird dann implementiert, um die Objekt zu kennen und die Position zu bestimmen.         
