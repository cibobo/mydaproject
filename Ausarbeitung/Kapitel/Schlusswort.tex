\chapter{Schlusswort}


Diese Arbeit implementiert die auf Marker basierte Objekterkennung bzw. Verfolgung im Grund auf dem Rahmenwerk MAROCO. Die Oberfl�chen der Zielobjekt werden mit retroreflektierenden Marker markiert. Eine PMD-Kamera beobachtet die ganze Szene von oben und liefert direkt die 3D-Daten. Die totale Laufzeit kann in zwei Phasen zusammengefasst werden. In der Initialisierungsphase wird die Fremdobjekt unter der Kamera gezeigt und die Marker darauf sollen herausgekannt und im System gespeichert werden. Wenn es mehr Objekte gibt, wird eine Kalibrierung am Anfang durchgef�hrt. Mills in \cite{MN00} hat eine kompakte Segmentierung der Bewegung mithilfe des sogenannten \glqq Feature Interval Graph\grqq  dargestellt. \cite{AJ05} erweitert die Arbeit von Mills. Ein auf Pyramide basiertes Clustering-Verfahren wird statt des alten auf Dreiecke basierten Clustering-Verfahren vorgeschlagen. Nach der Initialisierungsphase wird das Objekt aus dem Gesichtsfeld der Kamera verschoben. Die Erkennungsphase f�ngt genau an, wenn das gleiche Objekt wieder unter der Kamera eingebracht wird. Die reflektierende Marker sollen nochmal gesammelt werden und ein von \cite{AJ05} repr�sentierter Teilgraph-Tracker wird dann implementiert, um die Objekt zu kennen und die Position zu bestimmen.         
