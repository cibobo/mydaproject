\chapter{Einleitung}

\section{Motivation}
Die St�rken von Robotern liegen in der Wiederholung von einfachen Handhabungst�tigkeiten. Dagegen sind Menschen mit ihren kognitiven F�higkeiten einzigartig, etwa in Bezug auf ihren Verst�ndnis der Aufgabe. Die Kombination von Mensch und Roboter kann Aufgaben stark rationalisieren, sofern jedem die optimalen Arbeitsteile zugewiesen wird. Die Anwendungsbereiche der Mensch-Roboter-Kooperation vergr��ern sich derzeit auf dem Feld der Medizin sowie der Industrie immer schneller. Damit Menschen und Roboter in einer geringen Entfernung sicher und effizient zusammenarbeiten k�nnen, ist die Erkennung bzw. die Verfolgung von Menschen und Objekten f�r ein Mensch-Roboter-Kooperation-System notwendig. Die Menschenerkennung garantiert die Sicherheit f�r den Menschen und liefert gleichzeitig Informationen �ber dessen Blickrichtung, um Aussagen �ber die Aufmerksamkeit des Menschen treffen zu k�nnen. Die Objekterkennung vereinfacht die Kommunikation zwischen Mensch und Roboter, wodurch die Fremdobjekte ohne weitere Programmierung direkt vom Roboter erkannt werden k�nnen. Au�erdem vermeidet die Objektverfolgung auch die Kollision zwischen Roboter und anderen Anlagenteilen.


\section{Aufgabenstellung}
Das Rahmenwerk MAROCO wird am Institut f�r Prozessrechentechnik, Automation und Robotik (IPR) entwickelt, damit Menschen und Roboter in einer gemeinsamen Umgebung sicher zusammenarbeiten k�nnen. Die Erfassung des Menschen und die Handlungsanalyse in der Szene erlauben es, die Gefahren von Roboterbewegungen f�r Menschen zu minimieren. Jedoch ist die Erkennung zurzeit auf das Menschmodell beschr�nkt. Alle anderen Objekte werden von dem System als Zylinder dargestellt. Das Ziel dieser Arbeit besteht darin, die verschiedenen Objekte zu kalibrieren und die entsprechenden geometrischen Charakteristika in dem System zu speichern, um dann die Objekte mit Hilfe der gespeicherten Informationen wiederzuerkennen und zu verfolgen. Beide Schritte sollen in Echtzeit durchgef�hrt werden.
\\
\\
%Eine spezielle auf Punktwolke basierte Methode werde f�r die Objektkalibrierung implementiert, die mithilfe der Pyramide die verschiedenen Objekte unterscheiden kann. Jedes Objekt wird als ein Graph mit Knoten und Kanten im System gespeichert. Im Teil der Objekterkennung bzw. Objektverfolgung wird ein Teilgraph-Tracker realisiert, was eine schnelle, stabile und pr�zise Spur liefern kann, obwohl einige Teile des Objekts verdeckt sind. Au�erdem ist die Stereo Korrespondenz als die Voraussetzung aller Funktionen sehr wichtig. Ein auf Singul�rwertzerlegung basiertes Matchingsverfahren wird f�r infrarot Bild implementiert. 


Folgende Ziele sollen erreicht werden:
\begin{itemize}
\item Objektsegmentierung,
\item Darstellung und Speichern des charakteristischen Modells der Objekte,
\item Objektwiedererkennung und Verfolgung,
\item Einhaltung der Echtzeitbedingungen (Framerate $\geq$ 30fps).
\end{itemize}
