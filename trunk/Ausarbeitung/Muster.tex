
% --------------------------------------------------------------------
% LaTeX2e - Modifizierter Rahmen fuer 
% Ausarbeitungen basierend auf
% dem Muster von M. Haag (27.09.1995) 
% durch J�rgen Graf (irgendwann 2007)
% und Stephan Puls (Ende 2010)
% -------------------------------------------------------------------

\documentclass[12pt,twoside,a4paper]{report}
\usepackage{mysty,epic,eepic,bibdef,bibhyphen,german,makeidx,longtable,time,theorem,amsmath,amssymb,uhrzeit,multirow,ifpdf}
\usepackage[latin1]{inputenc}
\usepackage{url}
\usepackage{algorithm, algpseudocode}
\urlstyle{sf}

% ------------------------------------------------------------------
% Pakete in Zusammenhang mit pdflatex einbinden
% -------------------------------------------------------------------
%\newif\ifpdf
%\ifx\pdfoutput\undefined
%    \pdffalse           % we are not running PDFLaTeX
%\else
%    \pdfoutput=1        % we are running PDFLaTeX
%    \pdftrue
%\fi
%----------------------------------------------------------------------
\ifpdf
    \usepackage[pdftex,
        colorlinks=true,
        urlcolor=blue,                      % \href{...}{...}
        anchorcolor=rltbrightblue,
        filecolor=green,                    % \href*{...}
        linkcolor=black,                % \ref{...} and \pageref{...}
        menucolor=webdarkblue,
        citecolor=rltgreen,
        pdftitle={},
        % TODO: Hier Deinen Namen eintragen
        pdfauthor={Dein Name},
        pdfsubject={},
        pdfkeywords={},
        pagebackref=true,
        %pdfpagemode=None,
        bookmarksopen=true]{hyperref}
    \pdfcompresslevel=9
    \usepackage[pdftex]{graphicx}
    \usepackage{thumbpdf}
\else
    \usepackage[
        colorlinks=true,
        urlcolor=black,                 % \href{...}{...}
        anchorcolor=black,
        filecolor=black,                    % \href*{...}
        linkcolor=black,                % \ref{...} and \pageref{...}
        menucolor=black,
    pagebackref=true,
        citecolor=black]{hyperref}
    \usepackage{graphicx}
\fi
 \usepackage{color}
\definecolor{rltbrightred}{rgb}{1,0,0}
\definecolor{rltred}{rgb}{0.75,0,0}
\definecolor{rltdarkred}{rgb}{0.5,0,0}
%
\definecolor{rltbrightgreen}{rgb}{0,0.75,0}
\definecolor{rltgreen}{rgb}{0,0.5,0}
\definecolor{rltdarkgreen}{rgb}{0,0.25,0}
%
\definecolor{rltbrightblue}{rgb}{0,0,1}
\definecolor{rltblue}{rgb}{0,0,0.75}
\definecolor{rltdarkblue}{rgb}{0,0,0.5}
%
\definecolor{webred}{rgb}{0.5,.25,0}
\definecolor{webdarkblue}{rgb}{0,0,0.75}
\definecolor{webbrightgreen}{rgb}{0,0.5,0}


%  Use alternative backref in newer versions of backref and
%  redefine the backreference. You can redefine \backrefalt
%  to suit your requirements.

   \newcounter{countCites}
   \newcounter{countNotCited}
   \backrefgerman
   \renewcommand*{\backref}[1]{}
   \newcommand{\backreftext}[1]{\footnotesize{\textsf{#1}}}

   \renewcommand*{\backrefalt}[4]{%
      \ifcase #1 %
          \backreftext{\textcolor{green}{\textbf{(NICHT ZITIERT)}}}%
      \stepcounter{countNotCited}
          \stepcounter{countCites}
      \or
         \backreftext{(Zitiert auf Seite~#2)}%
     \stepcounter{countCites}
      \else
         \backreftext{(Zitiert auf Seiten~#2)}%
     \stepcounter{countCites}
      \fi}




% -------------------------------------------------------------------
% Index erstellen (optional)
% -------------------------------------------------------------------
%\makeindex


%Grafiken ausblenden:
%\makeatletter
%  \renewcommand{\Ginclude@eps}[1]{}
%\makeatother

%Text ausblenden I:
%\usepackage{color}
%\color{white}


% -------------------------------------------------------------------
% globale Layout-Einstellungen:
% -------------------------------------------------------------------

\renewcommand\textfraction{0.0}
\renewcommand\bottomfraction{1.0}
\renewcommand\topfraction{1.0}
\renewcommand\arraystretch{1.2} %spacing between rows
\renewcommand{\baselinestretch}{1.0} % Zeilenabstand

% eigene Anpassungen (bei Bedarf):
% \input{my_commands}

\renewcommand\textfraction{0.15}
\addtolength\intextsep{5mm}
\addtolength\floatsep{5mm}
\addtolength\textfloatsep{5mm}

% Ende der eigenen Anpassungen

\global\parindent=0pt % 5pt
\global\parskip=1.5\smallskipamount\relax

\marginparsep 0.0cm
\textheight 22cm
\textwidth 15.5cm
\oddsidemargin 0cm
\evensidemargin 0cm

\setcounter{tocdepth}{3}
\setcounter{secnumdepth}{3}

\setlength{\unitlength}{1mm}

% ---- globale Definitionen
% evtl. in eigene Datei
%\input{Definitionen}

% ===================================================================
% Makros (J�rgen Graf, 07.06.2006)
% ===================================================================
\theoremstyle{plain}
\newtheorem{definition}{Definition}[chapter]
\newtheorem{theorem}{Satz}[chapter]
\newtheorem{lemma}{Lemma}[chapter]
\newtheorem{proposition}{Proposition}[chapter]
\newtheorem{corollary}{Korollar}[chapter]
\newtheorem{observation}{Beobachtung}[chapter]
\newtheorem{fact}{Fakt}[chapter]
\newtheorem{procedure}{Prozedur}[chapter]
\newtheorem{postulat}{Postulat}[chapter]
%\newtheorem{proof}{Beweis}[chapter]
{\theorembodyfont{\rmfamily}
\newtheorem{remark}{Abschlie�ende Bemerkung}[subsection]
\newtheorem{example}{Beispiel}[chapter]
}

\newbox\ProofSym
\setbox\ProofSym=\hbox{
  \unitlength=0.18ex
  \begin{picture}(10,10)
    \put(0,0){\framebox(6,6){}}
%    \put(0,3){\framebox(6,6){}}
  \end{picture}
}

\newenvironment{proof}%
{\noindent{\em \iflanguage{german}{Beweis}{Proof}.\hspace{2mm}}}%
{\hfill\copy\ProofSym\linebreak}

% Abk�rzungen f�r die reellen, nat�rlichen, ganzen,... Zahlen
\newcommand{\R}{{\ensuremath{\mathbb{R}}}}
\newcommand{\N}{{\ensuremath{\mathbb{N}}}}
\newcommand{\Z}{{\ensuremath{\mathbb{Z}}}}
\newcommand{\C}{{\ensuremath{\mathbb{C}}}}
\newcommand{\Q}{{\ensuremath{\mathbb{Q}}}}
\newcommand{\F}{{\ensuremath{\mathbb{F}}}}
\newcommand{\Prim}{{\ensuremath{\mathbb{P}}}}

\newcommand{\mo}{\texttt{Motris}}
% ===================================================================
% BEGIN DOCUMENT
% ===================================================================
\begin{document}
% PDFLATEX-CHANGE
\ifpdf
    \DeclareGraphicsExtensions{.jpg,.png,.pdf}
\else
    \DeclareGraphicsExtensions{.eps,.ps}
\fi
% PDFLATEX-CHANGE

\pagestyle{empty}

% -------------------------------------------------------------------
% Titel:
% -------------------------------------------------------------------
\input{Kapitel/titel}
\cleardoublepage  % nur fuer die Endfassung

% -------------------------------------------------------------------
% Bemerkungen:
% hier kommen Anmerkungen rein, z.B. noch nicht verarbeitete
% Literaturhinweise
% -------------------------------------------------------------------
%\input{Bemerkungen}
%\cleardoublepage

% -------------------------------------------------------------------
% Erklaerung:
% (nur fuer die Endfassung)
% -------------------------------------------------------------------

%\input{Erklaerung}
%\cleardoublepage

% -------------------------------------------------------------------
% Danksagung:
% (optional, nur fuer die Endfassung)
% -------------------------------------------------------------------
%\input{Danksagung}
%\cleardoublepage

% -------------------------------------------------------------------
% Kurzfassung:
% -------------------------------------------------------------------
%\input{Kurzfassung}
%\cleardoublepage

% -------------------------------------------------------------------
% Verzeichnisse:
% -------------------------------------------------------------------

\pagestyle{headings}    % jetzt mit Seitennummern und Kopfzeilen
\pagenumbering{roman}   % aber Seitennummern mit roemischen Ziffern

\tableofcontents        % Inhaltsverzeichnis
\cleardoublepage

%\listoffigures          % Abbildungsverzeichnis
%\cleardoublepage

%\listoftables           % Tabellenverzeichnis
%\cleardoublepage

%\clearpage

% -------------------------------------------------------------------
% Hauptteil:
% -------------------------------------------------------------------

\pagenumbering{arabic}  % ab hier arabische Seitennummern, bei 1
                        % beginnend

% TODO: Ab hier die Kapitel einf�gen

% Zum Beispiel:
\chapter{Einleitung}

\section{Motivation}
Die St�rken von Robotern liegen in der Wiederholung von einfachen Handhabungst�tigkeiten. Dagegen sind Menschen mit ihren kognitiven F�higkeiten einzigartig, etwa in Bezug auf ihren Verst�ndnis der Aufgabe. Die Kombination von Mensch und Roboter kann Aufgaben stark rationalisieren, sofern jedem die optimalen Arbeitsteile zugewiesen wird. Die Anwendungsbereiche der Mensch-Roboter-Kooperation vergr��ern sich derzeit auf dem Feld der Medizin sowie der Industrie immer schneller. Damit Menschen und Roboter in einer geringen Entfernung sicher und effizient zusammenarbeiten k�nnen, ist die Erkennung bzw. die Verfolgung von Menschen und Objekten f�r ein Mensch-Roboter-Kooperation-System notwendig. Die Menschenerkennung garantiert die Sicherheit f�r den Menschen und liefert gleichzeitig Informationen �ber dessen Blickrichtung, um Aussagen �ber die Aufmerksamkeit des Menschen treffen zu k�nnen. Die Objekterkennung vereinfacht die Kommunikation zwischen Mensch und Roboter, wodurch die Fremdobjekte ohne weitere Programmierung direkt vom Roboter erkannt werden k�nnen. Au�erdem vermeidet die Objektverfolgung auch die Kollision zwischen Roboter und anderen Anlagenteilen.


\section{Aufgabenstellung}
Das Rahmenwerk MAROCO wird am Institut f�r Prozessrechentechnik, Automation und Robotik (IPR) entwickelt, damit Menschen und Roboter in einer gemeinsamen Umgebung sicher zusammenarbeiten k�nnen. Die Erfassung des Menschen und die Handlungsanalyse in der Szene erlauben es, die Gefahren von Roboterbewegungen f�r Menschen zu minimieren. Jedoch ist die Erkennung zurzeit auf das Menschmodell beschr�nkt. Alle anderen Objekte werden von dem System als Zylinder dargestellt. Das Ziel dieser Arbeit besteht darin, die verschiedenen Objekte zu kalibrieren und die entsprechenden geometrischen Charakteristika in dem System zu speichern, um dann die Objekte mit Hilfe der gespeicherten Informationen wiederzuerkennen und zu verfolgen. Beide Schritte sollen in Echtzeit durchgef�hrt werden.
\\
\\
%Eine spezielle auf Punktwolke basierte Methode werde f�r die Objektkalibrierung implementiert, die mithilfe der Pyramide die verschiedenen Objekte unterscheiden kann. Jedes Objekt wird als ein Graph mit Knoten und Kanten im System gespeichert. Im Teil der Objekterkennung bzw. Objektverfolgung wird ein Teilgraph-Tracker realisiert, was eine schnelle, stabile und pr�zise Spur liefern kann, obwohl einige Teile des Objekts verdeckt sind. Au�erdem ist die Stereo Korrespondenz als die Voraussetzung aller Funktionen sehr wichtig. Ein auf Singul�rwertzerlegung basiertes Matchingsverfahren wird f�r infrarot Bild implementiert. 


Folgende Ziele sollen erreicht werden:
\begin{itemize}
\item Objektsegmentierung,
\item Darstellung und Speichern des charakteristischen Modells der Objekte,
\item Objektwiedererkennung und Verfolgung,
\item Einhaltung der Echtzeitbedingungen (Framerate $\geq$ 30fps).
\end{itemize}

\chapter{Stand der Forschung}

3D Objekterkennung und Verfolgung kommt in vielen Anwendungsbereichen zum Einsatz. Daher wurden viele Algorithmen bzw. Systeme daf�r in den vergangenen Jahrzehnten entwickelt. Ein gutes Beispiel ist das kommerzielle Erkennungssystem von VICON \cite{VIC}. In dem System werden 8 Kameras benutzt, die von verschiedenen Richtungen das Zielobjekt bzw. Person beobachten. Einige wei�e Marken werden vorher am Ziel angebracht, damit seine Positionen und Bewegungen von den Kameras gut erkannt werden k�nnen. Nach Vergleichen der Bilder von verschiedenen Kameras kann ein 3D Modell des Ziels in Echtzeit erzeugt werden. Das System wird im Bereich von Computerspielen und Filmindustrie sehr oft benutzt. Ein anderes Beispiel ist das neue Ger�t Kinect von Microsoft XBox360 \cite{KIN}. Eine 3D Kamera kann die 3D Daten von Spielern ansammeln, damit die Spieler das Spiel direkt mit ihren K�rpern statt des traditionellen Kontroller steuern k�nnen. Die Analyseverfahren der Objekterkennung basieren auf unterschiedlichen Charakteristika der Objekte und sind f�r verschiedenen Typen von Objekten geeignet. In der Arbeit von Lepetit und Fua sind die aktuelle Verfolgungsverfahren in zwei gro�e Gruppen unterschieden worden: auf Marken basierte Objektverfolgungen und auf nat�rlichen Merkmale basierte Objektverfolgungen. Die Verfahren in der zweiten Gruppe k�nnen weiter in kantenbasiertes Verfahren, optischer Fluss basiertes Verfahren, Templatebasiertes Verfahren, Punktebasiertes Verfahren und das SLAM-Verfahren unterteilt werden \cite{LF05}. Im folgenden Abschnitt werden kurz die Details jedes Verfahrens erkl�ren.    

%In unserem System k�nnen alle Objekte in zwei Typen eingeteilt werden: die Objekte, die immer eine Seite zur Kamera richten, d.h. nur Bewegungen der Objekte in der Ebene von der Kamera beobachtet werden; und andere Objekte, die sich frei im Raum bewegen. Wir benennen den ersten Typ als feste Objekte z.B. der Arbeitsplatz, und den zweiten Typ als frei bewegte Objekte, z.B. das Werkzeug. Im Folgenden werden aktuelle Forschungsarbeiten �ber Objekterkennungsverfahren f�r beide Typen von Objekten erkl�ren.


%\section{Feste Objekte}
%Das Problem f�r die Erkennung der festen Objekte kann als die Objekterkennung mit planaren Marken zusammengefasst werden. In diesem Bereich sind viele Systeme f�r die Erweiterte Realit�t implementiert worden. 

\section{Standardmarkenbasierte Verfahren}
Die Verfolgungsverfahren k�nnen in zwei Schritte unterteilt werden: zuerst der Informationen von Bildsequenzen ansammeln um dann die Position des erkannten Objekts zu bestimmen. Die vordefinierte Marken k�nnen in beiden Schritten mehr Information liefern, damit die Objekte schneller und einfacher verfolgt werden k�nnen. Deshalb sind in diesem Bereich viele Systeme f�r die Erweiterte Realit�t implementiert worden. Ein Echtzeitsystem f�r Erweiterte Realit�t wurde von Zhang und Navab f�r Objektverfolgung in einer Industrieumgebung realisiert \cite{ZN00}. Sie haben eine Gruppe von 4 Vierecken als eine Marke benutzt. Die Marke wird durch Farbe und wei�e Flecken innerhalb der Vierecke kodiert. 
\\
\\
Ein anderes System hei�t ARToolKit, was vom HITLab der Universit�t Washington entwickelt wird. Es ist eine bekannte Software-Bibliothek zur Entwicklung von Anwendungen f�r die Erweiterte Realit�t \cite{ART}. In ARToolKit wird ein Viereck mit schwarzer Umrandung als Marke benutzt. Das Muster in der Mitte kodiert die Marke und kann frei gew�hlt werden. Das Eingabebild wird zuerst in ein Bin�rbild umgewandelt und dann alle verbundenen schwarzen Pixel extrahiert. Die Figur innerhalb der schwarzen Umrandung wird segmentiert und mit dem fr�heren definierten Muster verglichen. Durch den Vergleich kann man die Projektivit�t zwischen Kamerakoordinatensystem und Musterkoordinatensystem bestimmen. 
\\
\\
ARToolKit liefert eine hohe Frame-Rate mit bis zu 30 fps bei niedrigem CPU-Bedarf. Eine dicke schwarze Umrandung garantiert die Stabilit�t des Systems und die Marke kann in niedriger Aufl�sung sehr gut erkannt werden. Ein anderer wichtige Vorteil ist, dass die Verfolgung der ARToolKit keine Initialisierung braucht. Dadurch wird nicht nur die Laufzeit am Anfang des Verfahren gespart, kann aber auch Chaos vermeiden, wenn die eingegebene Bildsequenz abgebrochen wird.   
%Wegen dieser Vorteile wird in unserem System ARToolKit benutzt werden, um die festen Objekte wie Arbeitspl�tze zu erkennen.


%\section{Frei bewegte Objekte}
%Wegen der Bewegung in mehren Freiheitsgraden, ist die Erkennung der frei bewegten Objekte viele komplexer. Das System soll die komplette charakteristische Information des Objekts betrachten, um das gleiche Objekt wiederzuerkennen, wenn es noch mal unter die Kamera gebracht wird. Im wesentlichen beruhen die aktuellen Objekterkennungsverfahren auf Templates, Kanten und Punktes. 

\section{Verfahren basierend auf nat�rlichen Merkmalen}

\subsection{Kantenbasiertes Verfahren}
Das kantenbasierte Verfahren wurde in fr�heren Objektverfolgungssystemen h�ufig benutzt, weil es effizient und einfach zu realisieren ist \cite{LF05}. Die Hauptidee dieses Verfahrens ist entweder die Kanten des Objekts direkt von dem Bild herauszufinden und zu verfolgen, oder den Teil des Bildes mit starkem Gradient zu betrachten, damit man die Konturen des Objekts zum n�chsten Zeitpunkt vorhersagen kann. RAPiD war eines der fr�hesten 3D Verfolgungsverfahren, das in Echtzeit laufen konnte \cite{H92}. Vacchetti und Lepetit haben ein neues, effizienteres Verfolgungsverfahren entwickelt, was mehr als eine Voraussagen f�r die ausgew�hlten Steuerpunkte darstellen \cite{VLF04}. Diese Erweiterung verst�rkt die Stabilit�t der Verfolgung und erf�llt weiterhin die Echtzeitbedingung.   
%Falls ein Teil des Objekts von H�nden oder anderem Werkzeug verdeckt wird, kann das Verfahren wegen des Verlustes der Kanteninformation keine richtige L�sungen liefern. Aus dem gleichen Grund entspricht das auf Punktwolke basiertes Objekterkennungsverfahren unseren Bedarf auch nicht. 

\subsection{Optischer Fluss basiertes Verfahren}
Der Optische Fluss ist ein Vektorfeld, das die Bewegungsrichtung und Bewegungsgeschwindigkeit f�r jeden Bildpunkt einer Bildsequenz bezeichnet. Die Berechnung des Optischen Fluss kann als eine Differentialgleichung zusammengefasst werden und das L�sungsverfahren wurde von Horn und Schunck entwickelt \cite{HS81}. Black und     Yacoob benutzten reine Optische Fluss basierte Verfahren f�r die Verfolgung kleiner Ver�nderungen auf menschlichem Gesicht, um den Gesichtsausdruck zu bestimmen \cite{BY97}. Au�erdem wurde ein Verfolgungssystem f�r den Innerstadt Verkehr von Haag und Nagel durch die Verkn�pfung der Information von Optischem Fluss und Kanten des Objekts implementiert \cite{HN99}.  


\subsection{Templatebasiertes Verfahren}
Im Templatebasierten Verfahren wird ein Objekt nicht durch lokale Merkmale z.B. Kanten oder Punkte, sondern durch das globale Charakteristikum erkannt und verfolgt. Das Verfahren ist geeignet f�r komplexe Objekte, die nicht einfach durch lokale Merkmale bezeichnet werden k�nnen \cite{LF05}. Der Lucas-Kanade Algortihmus wurde anf�nglich entwickelt, um den Optischen Fluss zu berechnen \cite{LK81}, ist aber auch f�r die 2D templatebasierte Verfolgung nutzbar. Jurie und Dhome haben einen Algorithmus f�r die Verfolgung von ebenen Objekten mithilfe von Hyperebenen entwickelt \cite{JD01}. In ihrer Arbeit wurde die Approximation der Abbildung des Objekts auf Hyperebenen abgesch�tzt, dadurch die Translation des Objekts bestimmt werden kann. 
%Das templatebasierte Verfahren braucht eine relative gro�e Menge der Samples f�r verschiedene Deformation des Templates f�r jedes Objekt. Da das ganze Template in der Verfolgungsphase mit den Samples verglichen werden soll, braucht das Verfahren viel Zeit. Obwohl das Verfahren f�r einige komplexe Objekte effektiv ist, ist es f�r unsere Situation aber nicht geeignet \cite{LF05}. 

\subsection{Punktebasiertes Verfahren}
Der Unterschied zwischen dem punktebasierten Verfahren und den oben beschriebenen Verfahren ist, dass nur lokale Merkmale betrachtet werden. Im Vergleich zum Verfahren, das globale Merkmale behandelt, ist das Verfolgungsverfahren mit lokalen Merkmalen viel stabiler, wenn es Kollision f�r mehr Objekte gibt, oder die Messung der Merkmalen stark st�rt wird \cite{LF05}. Ein Verfahren wurde von Zhang et al. im Jahre 1995 realisiert, was die nicht kalibrierten Bilder als Eingabe benutzen kann \cite{Z95}. In dem Verfahren wird kein Modell der Epipolargeometrie verwendet, wodurch viele komplexen Berechnungen vermieden werden. Eine andere M�glichkeit f�r die Punkteverfolgung ist der Kanade-Lucas-Tomasi(KLT) Tracker, was auf der Arbeit von \cite{LK81} begr�ndet wurde. Sie haben eine Approximation f�r den Unterschied zwischen zwei Bildern definiert. Mithilfe des Iterationsverfahrens von Newton-Raphson wird die Approximation minimiert. Dadurch kann die Translation des Objekts bestimmt werden. Tomasi erweiterte den Algorithmus von Lucas und Kanade mit einer besseren Strategie zur Auswahl von Merkmalen \cite{TK91}. Der dritte Schritt wurde von Shi und Tomasi vervollst�ndigt \cite{ST94}. Sie verbesserten weiter die Auswahl der Punkte mit der �hnlichkeit zwischen dem Anfangsbild und das aktuelle Bild. Diese �hnlichkeit wird durch einem Modell von affiner Abbildung bestimmt. Au�erdem benutzen sie gleichzeitig zweites Modell von reiner Translation, um das Objekt mit hoher Seriosit�t und Pr�zision zu verfolgen. 

\section{Markenbasierte Objekterkennung}
\label{MObjEr}
Rhijn und Mulder haben ein markenbasiertes Verfahren entwickelt, was das Verdeckungsproblem behandelt \cite{AJ05}. Einige runde, hoch reflektierende Marken werden auf dem Objekt angebracht. In der Initialisierungsphase speichert das System die Charakteristika des Objekts als einen 3-dimensionalen vollst�ndigen Graph. Wenn das Objekt wiedererkannt werden soll, f�hrt das System zuerst einen Kalibrationsalgorithmus durch, um die verschiedenen Objekte zu differenzieren. Dann vergleicht das System f�r jedes Objekt die sichtbaren Marken mit dem in der Initialisierungsphase gespeicherten vollst�ndigen Graphen.  

\subsection{Markenerkennung}
Der erste Schritt der Markenbasierten Objekterkennung ist, alle angebrachten Marken zu erkennen. Die freie Programmbibliothek OpenCV liefert viele verschiedenen Algorithmen f�r Markenerkennung, und die Geschwindigkeit, Stabilit�t bzw. die Genauigkeit dieser Algorithmen werden von Odessa in seiner Ausarbeitung verglichen \cite{O11}. In diese Arbeit wird der Erkennungsalgorithmus STAR ausgew�hlt, wegen der niedrigen durchschnittlichen Fehler-Rate, was besonders wichtig f�r die Markenerkennung ist \cite{AKB08}. Au�erdem betrachtet der STAR Algorithmus weniger Merkmale als andere Algorithmen, deshalb wird die gesamte Laufzeit der Markenerkennung verk�rzt. Diese Eigenschaft entspricht genau unseren Anforderungen, weil nur wenige Marken an dem Objekt angebracht werden (weniger als 10 an jeder Oberfl�che).

\subsection{Markenverfolgung}
Nach Bestimmung der Marken, sollen diese Marken in einer Bildsequenz verfolgt werden. Scott und Longuet-Higgins haben einen eleganten und einfachen Algorithmus erzeugt \cite{SL91}. Sie haben das Problem zu einer Matrix zusammengefasst, deren Elemente als die Distanz zwischen verschiedenen Merkmalen definiert werden. Durch eine Singul�rwertzerlegung und Matrixersetzung werden die Abbildungen der Marken zwischen zwei Bildern bestimmt. Rhijn und Mulder verbesserten den Algorithmus von Scott und Longuet-Higgins. Sie haben die Elemente der Matrix neu definiert und f�gten die Beschr�nkung von Epipolargeometrie ein \cite{AJ05}.  

\subsection{Sch�tzung der Transformation}
F�r ein 3D Objekt ist die Markenverfolgung innerhalb von Bildern nicht ausreichend, um die komplette geometrische Information zu rekonstruieren. Deshalb soll die Pose des Objekts gleichzeitig bestimmt werden, damit ein r�umlicher charakteristischer Graph f�r das Objekt erstellt werden kann. Nat�rlich kann man mit Hilfe der Ergebnisse der Markenverfolgung die relative Rotation und Translation des Objekts zwischen zwei Zeitpunkten berechnen, aber au�erdem gibt es Verfahren, die durch Vergleich der Punktwolken von zwei Bildern direkt die Transformation des Objekts bestimmen k�nnen. Diese Verfahren wurden von Eggert et al. in ihrer Arbeit durch Merkmale und L�sungsverfahren in verschiedene Typen unterteilt \cite{ELF97}. Die Merkmale k�nnten die Oberfl�che, die Kanten bzw. die Punkte sein. Die Verfahren, die auf Punkte basieren, werden im Praktisch h�ufig benutzt und sind geeignet f�r die Markenbasierte Objekterkennung \cite{ELF97}. Die L�sungsverfahren k�nnen in iterative Verfahren und geschlossene Form unterschieden werden.

\subsubsection{Geschlossene Form}
Ein effizientes Verfahren mit geschlossener Form f�r das Berechnen der Transformation wurde von Arun et al. zuerst am Jahr 1987 ver�ffentlicht \cite{AHB87}. Die Hauptidee des Verfahrens ist, eine approximierte Rotations- bzw. Translationsmatrix zwischen zwei aufeinander folgenden Bildern zu bestimmen, damit die Summe des Unterschieds zwischen den durch approximierte Transformation berechnete Positionen der Punkte und die genaue Positionen der Punkte minimiert wird. Dieses Verfahren basiert auf der Singul�rwertzerlegung einer Korrelationsmatrix. Die Rotations- und Translationsmatrix werden am Ende ausgegeben. Wenn die zwei eingegebenen Punktwolken auf gleich Oberfl�che liegen, liefert das Verfahren leider keine richtige Rotationsmatrix. Deshalb soll eine korrigierte Matrix darauf aufbauen, die von Umeyama \cite{U91} und Kanatani \cite{K94} vorschlagen wurde. Ein anderes Verfahren wurde von Horn entwickelt, was die relative Rotation durch Einheitsquaternionen beschreibt \cite{H87}. Im Vergleich zu der Standard-Beschreibung der Rotation als Matrix ist die Quaterniondarstellung viel effizienter und stabiler. Die verbesserte Stabilit�t kann den Fehler vermeiden, wenn der Winkel der Rotation zur Singularit�t wird, z.B. $0^\circ$ oder $180^\circ$. D.h. dieses Verfahren braucht keine Ma�nahme f�r Behandlung des speziellen Winkels, was aber im Verfahren von Arun n�tig ist \cite{AHB87}. 
\\
\\
Eggert et al. haben in ihrer Arbeit die obengenannte zwei Verfahren mit zwei anderen geschlossene Form-Verfahren verglichen \cite{ELF97}. Wenn die Anzahl der betrachteten Punkten weniger als 100 ist, braucht das Verfahren mit Einheitsquaternionen weniger Zeit als die anderen Verfahren. Auf der anderen Seite, wenn die Anzahl der betrachteten Punkten weniger als 10 ist, liefert das Verfahren von Arun den kleinsten Fehler.

\subsubsection{Iterative Closest Point Algorithmus}
Der Iterative Closest Point Algorithmus ist ein Algorithmus, der es erm�glicht, Punktwolken aneinander anzupassen \cite{IW}. In jedem Iterationsschritt wird der korrespondierende Punkt f�r jeden Punkt einer Punktwolke aus anderer Punktwolke gefunden. Die Transformation zwischen beiden Punktwolken werden so bestimmt, dass die Summe des Abstands der korrespondierenden Punkte minimiert wird. Dieser Vorgang wird wiederholt, bis die Ver�nderung des mittleren quadratischen Fehler zwischen zwei folgenden Schritten unter einer Schranke liegt. Der Algorithmus wurde erst von Chen und Medioni \cite{CM91} vorgestellt und ICP wurde als Name des Algorithmus von Besl und McKay in ihrer Arbeit erstmals benutzt \cite{BM92}.  Doria et al. erweiterten den  Algorithmus mit gewichtetem Kriterium f�r die Behelligung \cite{DMJ97}. Ein sch�ner Vergleich der verschiedene ICP Algorithmen vor dem Jahr 2001 wurde von Rusinkiewicz und Levoy ausgegeben \cite{RL01}. Der grundlegende ICP Algorithmus wurde von ihnen in 6 Schritten unterteilt. In jedem Schritt wurde die Leistung des Algorithmus verglichen und ihr Einfluss auf den ganzen Algorithmus diskutiert. Eine andere Ver�nderung wurde von Chavarria und Sommer vorgeschlagen, in der die Kontur des Objekts auch in der Sch�tzung der Pose betrachtet wird \cite{CS07}.      

\subsection{Objektkalibrierung}
Das Problem der Kalibrierung der Objekte ist �hnlich zum Problem der Segmentierung der Bewegungen in einer langen Bildsequenz. Eine traditionelle L�sungsstrategie ist, die Positionen der Marken im n�chsten Zeitpunkt mit Kalman Filter vorherzusagen. Dann werden alle Marken anhand der kinematischen Parameter in verschiedene Gruppen eingeteilt. Mills und Novins haben eine andere M�glichkeit geliefert, damit die Objekte direkt mit 2D Graphen kalibriert werden k�nnen \cite{MN00}. Am Anfang ihres Algorithmus wird jede Marke mit einander verbunden. Dann werden alle Marken und die Kanten dazwischen zusammen als ein vollst�ndig Graph dargestellt. Zwischen den Bewegungen der Objekte ver�ndert sich die L�nge der Kanten. Die Kanten, die l�nger als eingesetzte Beschr�nkung sind, werden aus dem Graphen Schritt um Schritt gel�scht. Eine Marke geh�rt einem Objekt, genau dann wenn das Dreieck, das von dieser Marke und anderen Marken in diesem Objekt erzeugt wird, mindestens eine gleiche Kante mit den anderen Dreiecken des Objekts hat. Am Ende des Algorithmus wird der urspr�ngliche vollst�ndige Graph in viele Teilgraphen zerlegt, die genau den kalibrierten Objekten entsprechen. Es gibt jedoch die Einschr�nkung in dem Algorithmus von Mills und Novins, dass zwei Objekte nicht auseinanderzuhalten sind, wenn ihre Marken mit einigen besonderen Strukturen angebracht werden \cite{AJ05}. Rhijn und Mulder haben dieses Problem gel�st, indem sie die Voraussetzung des Algorithmus ver�nderten. In der neuen Voraussetzung darf die Marke in einem Objekt erkannt werden, nur wenn sie mit anderen drei Marken in dem Objekt zusammen eine Pyramide erzeugen kann. Die Pyramide hier bezieht sich auf einen K�rper der Geometrie, der vier Knoten hat und jede zwei davon eine Kanten erzeugen. Die neue st�rkere Beschr�nkung erh�ht die Erfolgsquote deutlich.

\subsection{Objekterkennung und Verfolgung} 
Das Ziel dieser Arbeit ist, dass ein Objekt nach Initialisierung von dem System wieder erkannt werden kann. Eine Wiedererkennung des 3D Objekts durch 2D Bildfolgen wird von Lamdan et al. in ihrer Arbeit erfolgreich durchgef�hrt \cite{LSW88}. Sie haben einige interessante Punkte ausgew�hlt, um das totale Objekt zu beschreiben. Alle drei Punkte, die nicht in gleicher Gerade liegen, definieren ein Koordinatensystem, auf dem die entsprechenden Koordinaten von anderen Punkten berechnet und in einem HashMap gespeichert werden. Die richtige Korrespondenz wird so bestimmt, dass das kleinste Quadrate Modell der Transformation zwischen dem neuen Koordinatensystem und 2D Bild die beste L�sung liefert.
\\
\\ 
Andererseits kann die charakteristische Information jedes Objekts einen einzigen vollst�ndigen Graphen erzeugen. Alle sichtbaren Marken des erkannten Objektes k�nnen als ein Teilgraph des vollst�ndigen Graphen definiert werden, wodurch das Problem der Objekterkennung bzw. Objektverfolgung als das Teilgraph Isomorphismus Problem abgeleitet werden kann. Es gibt eine gro�e Menge an Algorithmen um das Problem zu behandeln, da das Isomorphismus Problem nicht nur im Bereich der Bildanalyse, sonder auch im Vergleich der Struktur von chemischen Verbindungen oder in biometrischer Identifikation betrachtet wird. Conte et al. haben eine sch�ne Zusammenfassung �ber diese Algorithmen ver�ffentlicht \cite{C04}. Durch ihre Taxonomie k�nnen alle diese Verfahren in zwei Gruppen von genauen bzw. ungenauen Graph Matching Algorithmen unterteilt werden. 
\\
\\
In einem genauen Graph Matching wird die strenge Korrespondenz zwischen zwei Graphen bestimmt. Die Abbildung von einen Graphen zu einem anderen soll bijektiv sein. Ullmann hat ein rekursives R�cksetzverfahren beschrieben, das sehr bekannt ist und bis heute f�r genaues Matching h�ufig benutzt wird\cite{U76}. Was von Rhijn und Mulder in ihrer Arbeit f�r die Objekterkennung implementiert wurde, ist auch ein genaues Graph Matching Verfahren \cite{AJ05}. Sie folgen der Idee von Lamdan, aber verbessern das Verfahren mit einer Beschr�nkung f�r die Gr��e des Teilgraphs, die ausreichend f�r die Unterscheidung von zwei Objekten ist. Nach der Verbesserung ist der neue Algorithmus viel effizienter und stabiler. Cordella et al. haben einen Algorithmus mit Namen VF2 f�r das Graph und Teilgraph Isomorphismus Problem in gro�en Graphen entwickelt \cite{CF04}. In dem Vorgang des Matching haben sie einige Regeln definiert, wodurch die Komplexit�t des Rechnens stark reduziert wird. Eppstein konzentriert auf das Teilgraph Isomorphismus Problem von planaren Graphen \cite{E99}. Der Graph wird in viele kleine B�ume unterteilt. Auf diesen wird mittels dynamischer Programmierung das Matching in linearer Zeit durchgef�hrt. 
\\
\\
Das genaue Graph Matching ist manchmal ungeeignet, beispielsweise f�r nicht komplett fest definierte Graphen, was z.B. bei Rauschen oder instabilen Komponenten vorkommt. Wegen des Unterschieds zwischen dem beobachteten Modell und idealen Modell, soll das Matchingsverfahren tolerant sein. Dadurch kann eine Korrespondenz zwischen zwei Graphen gefunden werden, obwohl es keine strenge Transformation dazwischen gibt. Au�erdem ben�tigt das genaue Graph Matching Verfahren exponentielle Laufzeit im Worst-Case, die durch eine Approximation in ungenauen Matchingsverfahren stark reduziert werden kann. Messmer und Bunke haben ein Fehler-tolerantes Verfahren f�r Teilgraph Isomorphismus mit unbekanntem Graph als Eingabe entwickelt \cite{MB98}. Die Modellgraphen werden durch eine Vorverarbeitung in kleine Teilgraphen unterteilt. Alle diese Teilgraphen werden so zusammengefasst, dass die oftmals vorkommenden Teilgraphen nur ein mal repr�sentiert werden. Der eingegebene Graph wird mit diesen verdichteten Graphen verglichen. Dadurch h�ngt die Laufzeit nur von der Anzahl der Modellgraphen ab.




\chapter{Grundlagen}

\section{Generierung der 3D-Daten}

\subsection{TOF-Sensor}

\subsubsection{TOF Kamera}
\label{TOF Kamera}
Die im MAROCO-System verwendete Kamera geh�rt zur Klasse der TOF-Sensoren, die au�er den normalen Graufarbenbildern auch Tiefbilder liefern kann. Die Tiefmessung basiert auf dem sogenannten Laufzeitverfahren. Dazu wird die Szene durch ein Lichtpuls ausgeleuchtet und f�r jeden Bildpunkt wird die Zeit gemessen, die das Licht bis zum Objekt und wieder zur�ck ben�tigt. Die Distanz ist direkt proportional zu dieser  Zeit und kann durch die folgende Formel berechnet werden:

\begin{equation}
\label{tof1}
d = \frac{t_d}{2c}
\end{equation} 

wobei $t_d$ die gemessene Zeit bezeichnet. Die Konstante $c$ steht f�r die Lichtgeschwindigkeit. 
\\
\\
Im Vergleich zu anderen 3D-Kamerasystemen hat die TOF (englisch: time of flight) Kamera viele Vorteile \cite{TOFWiki}. Zuerst kann die TOF Kamera  die interessierenden Bereiche einfach aus einem Bild extrahieren und nur die Pixel nah vor der Kamera betrachten. Zweitens kann die TOF Kamera eine hohe Bildrate bis zu 100 fps erreichen. Diese Eigenschaft erm�glicht somit eine Echtzeitanwendungen. Au�erdem ben�tigt die TOF Kamera weniger Platz im Vergleich zum  Triangulationssystem, und hat eine  niedrigere Abh�ngigkeit von der Systemstruktur gegen�ber dem Stereosystem.


\subsubsection{PMD Sensor}
Der PMD (englisch: Photonic Mixing Device) Sensor ist eine wichtige Art der TOF Kamera. Er liefert eine hohe Aufl�sung bis zu 204x204 Pixel und einer maximalen Bildrate bis zu 25 fps. Durch die Formel \eqref{tof1} wird der maximale Distanzbereich auf 7 m festgelegt. Die anderen wichtigen Parameter findet man in Abbildung~\ref{PMDParam} \cite{pmde}.
\\
\\
%\graphicspath{}
\begin{figure}[ftb]
\centering
\includegraphics[scale=0.6]{Abbildungen/PMDParam2}
\caption{Die grundlegende Parameter der PMD CamCube2. \cite{pmde}}
\label{PMDParam}
\end{figure}

Das Hintergrundlicht, z.B. das Sonnenlicht, k�nnte die Messung der Distanz stark st�ren. Die PMD Kamera benutzt das aktive Sendersignal und einen Fremdlicht-Filter (SBI), um das Hintergrundsignal zu unterdr�cken. Au�erdem bietet die PMD Kamera die M�glichkeit, die Integrationszeit der Kamera f�r jede Messung individuell einzustellen. Die Integrationszeit bezieht sich auf die Zeitspanne, in der die Kamera zur Aufzeichnung eines Bildes dem reflektierten Licht ausgesetzt wird. F�r ein schwach reflektierendes Objekt ben�tigt der Sensor eine l�ngere Integrationszeit als ein stark reflektierendes Objekt, um genug Information anzusammeln. Andererseits wird aber ausreichendes Licht von hellen Objekten auf den Sensor reflektiert, wenn die Integrationszeit zu lang definiert wird. In Abbildung~\ref{PMDIntTime} wird ein Beispiel der Tiefbilder mit verschiedenen Integrationszeiten gezeigt \cite{pmdd}. 

\begin{figure}[ftb]
\centering
\includegraphics[scale=0.6]{Abbildungen/PMDIntTime}
\caption{Integrationszeit von 140 $\mu s$, 1400 $\mu s$ bzw. 14000 $\mu s$. Bitte beachten Sie die niedrige Signalst�rke an der linken Seite und die S�ttigung an der rechten Seite wegen der unangemessenen Integrationszeit. \cite{pmdd}}
\label{PMDIntTime}
\end{figure}

\subsubsection{Unterschied zwischen TOF Kamera und Kinect}
Kinect ist eine Hardware zur Steuerung der Videospielkonsole Xbox360, die ein sogenanntes hands-free Kontrollieren liefert, wodurch die Spieler mit einigen bestimmten Gesten oder einer kurzen Bewegung ihres K�rpers das Spiel steuern k�nnen \cite{KIN}. Um dieses Ziel zu erreichen, sammelt Kinect abgesehen von der normalen Bildeingabe aber auch die Tiefdaten der Szene an. Wegen dieser Eigenschaft wird das Ger�t im Bereich von Computer Vision benutzt. Mithilfe des SDK ist die Programmierung der Kinect unter normalen Betriebssystemen, wie z.B. Windows, Linux bzw. MacOS m�glich.
\\
%\begin{figure}[bft]
%\centering
%\includegraphics[scale=0.6]{Abbildungen/Kinect}
%\caption{Das Arbeitsprinzip des Kinects. \cite{KINH}}
%\label{Kinect}
%\end{figure}
\\
Der Unterschied zwischen TOF Kamera und Kinect kann auf dem Arbeitsprinzip zur�ckgef�hrt werden. In TOF Kameras wird die Tiefdaten durch das Laufzeitverfahren berechnet, was im \ref{TOF Kamera} erkl�rt wird. Das Abtastverfahren der Kinect hei�t ,,Light Coding''. Eine gro�e Menge von Punkten wird als Mustern auf die Szene bzw. die Objekte durch infrarotes Licht projiziert. Die ganze Szene mit diesen zus�tzlichen Mustern wird von einer infraroten Kamera des Kinects aufgenommen. Durch die Verzerrung zwischen dem vordefinierten Muster im infraroten Licht und dem von der infraroten Kamera erkannten Muster kann das Tiefbild der Szene ausgerechnet werden. Weitere Informationen findet man im technischen Dokument des Firma Cadet \cite{KINH}. Der Vergleich �ber die genauen technischen Daten von PMD Kamera und Kinect wird in Tabelle~\ref{PMD and Kinect} zusammengefasst. Obwohl Kinect eine bessere Aufl�sung und gr��eres Sichtfeld hat, ist die PMD Kamera wegen ihrer hohen Bildrate und ihres gro�en Messbereichs f�r das MAROCO-System geeignet. Au�erdem ist mithilfe des SBI Systems die Arbeit der PMD Kamera unter schwieriger Umgebungsbedingung, wie z.B. au�erhalb des Zimmers mit starker St�rung von Sonneneinstrahlung, auch m�glich.

\begin{table}[ftb]
\begin{center}
\begin{tabular}{| l || c | c |}
\hline
Sensor & PMD CamCube & Kinect \\ \hline
Aufl�sung & 204$\times$204 Pixel & 640$\times$480 Pixel\\ \hline
Sichtfeld & $40^\circ \times 40^\circ$ & $57^\circ \times 43^\circ$ \\ \hline
Max Bildrate & 25 fps & 30 fps \\ \hline
Messbereich & 0.3 $\rightarrow$ 7.0 m & 1.2 $\rightarrow$ 3.5 m (mit Xbox Software) \\ \hline
Lange der Tiefdaten & 8 bit (unsigned char) & 11 bit \\ \hline
\end{tabular}
\caption{Die technische Daten von PMD Kamera und Kinect. \cite{pmde}}
\label{PMD and Kinect}
\end{center}
\end{table}

\subsection{Daten der PMD Kamera}
\label{PMDData}
Die PMD Kamera CamCube kann insgesamt vier verschiedene Vermessungsdaten ausgeben. Hierzu z�hlen Amplitude, Intensit�t, Distanz und 3D-Koordinaten. Die ersten und letzten beiden Datentypen k�nnen durch die Dimension in zwei Gruppen unterteilt werden.

\subsubsection{2D Daten}
Die Intensit�t bezieht sich auf Graustufen. Nach einer Abbildung k�nnen diese Graustufen auf das Intervall zwischen 0 und 255 beschr�nkt werden. Das Ergebnis ist das Bild einer normalen Schwarz-Wei� Kamera. Die Amplitude zeigt die St�rke der Beleuchtung an, die vom Objekt wegen des aktiven Sendersignals von der PMD Kamera selbst reflektiert wird. Dieser Wert kann die Qualit�t der Distanzinformation absch�tzen, d.h. das Objekt weit von Kamera liefert niedriger Amplitude als das Objekt in der N�he von der Kamera, wenn sie mit identischem Material dargestellt werden. An der Gegenseite sind die Merkmale mit h�herem R�ckstrahlverm�gen durch die Daten der Amplitude einfacher betrachtet, was f�r die Erkennung der gro�en k�nstlichen Marken in dieser Arbeit sinnvoll ist.

\subsubsection{3D Daten}
Die PMD Kamera liefert 3D-Daten in zwei Formen: die reine Distanzinformation und die 3D-Koordinaten. Die Distanzinformation ist die gemessene Distanz zwischen der Kamera und dem Objekt, welches direkt durch die Formel \eqref{tof1} berechnet wird. Bez�glich dieser Distanzinformation und der 2D-Daten der normalen Kamera werden die 3D-Koordinaten innerhalb der PMD Kamera berechnet und k�nnen durch die Schnittstelle abgefragt werden. Die Abbildung~\ref{C3D} zeigt jeweils die Visualisierung einer Szene mit der 3D-Koordinaten und Distanzinformation von PMD Kamera. Eine Transformation ist notwendig, wenn man die Distanzinformation direkt im kartesischen Koordinatensystem visualisieren m�chte.

\subsubsection{Aufl�sung und erkennbare Markengr��e}
\label{AueM}
Von weiterem Interesse ist, wie genau die Objekte von der Kamera in der Praxis beobachtet werden k�nnen, d.h. wie gro� ein Pixel der Kamera ist, der den Bereich in realer Welt beschreibt. Die Vorderansicht des Kamerasystems dieser Arbeit ist in Abbildung~\ref{MSF} links gezeigt.

\begin{figure}[hbft]
\centering
\includegraphics[scale=0.44]{Abbildungen/MarkerSizeForme}
\caption{Die Vorderansicht (links) und die Draufsicht (rechts) des PMD Kamera Systems}
\label{MSF}
\end{figure}

$\theta$ ist der halbe Sichtwinkel und wird hier als $20^\circ$ angenommen. $H$ zeigt den Abstand von der Kamera zum Boden, der in diesem System mit 3,2m festgelegt ist. Durch Trigonometrie kann der Radius des Sichtbereiches auf dem Boden berechnet werden als:

\[
R = \tan \theta \cdot H =  0.36397 \cdot 3,2m = 1,16470m.
\]

Die halbe Seitenl�nge des Sehnenquadrats $L$ (Siehe Abb~\ref{MSF} rechts) ist gleich:

\[
L = \cos 45^\circ \cdot R = 0,70711 \cdot 1,16470m = 0,82357m.
\]

Dadurch erh�lt man den Fl�cheninhalt des Sichtbereiches auf dem Boden:
\[
S = 4 \cdot L^2 = 2,7131 m^2.
\] 

Das Ergebnis der Division von dem Fl�cheninhalt und der Aufl�sung beschreibt die Gr��e der Zelle auf dem Boden, die in der Kamera als eigener Pixel dargestellt wird.

\[
S_z = S / (204 \times 204) = 2,7131 m^2 / 41616 = 6,5194 \times 10^{-5} m^2 = 0,65194 cm^2
\]

Die Marken, die kleiner als $S_z$ sind, werden in der Kamera kleiner als ein Pixel abgebildet und sind deshalb nat�rlich schwer zu erkennen. Daraus folgt, dass $S_z$ die minimale Gr��e der erkennbaren Marken definiert. Die minimale Seitenl�nge kann nun berechnet werden durch:

\[
L_z = \sqrt{S_z} = 0,80743 cm
\]

Die minimalen Gr��en der Marken f�r andere, zum Boden parallelen Ebenen k�nnen analog berechnet werden. So kann z.B. in der normalen Arbeitsebene dieser Arbeit, die zum Boden $1,1m$ entfernt ist, die Mindestanzahl der Quadrate mit Seitenl�nge von $0,52987cm$ erkannt werden.

\begin{figure}
\centering
\includegraphics[scale=0.45]{Abbildungen/Compare-3D-Dis}
\caption{Die Visualisierung von 3D Koordinaten (links) und Distanzinformation (rechts).}
\label{C3D}
\end{figure}

\section{Vorverarbeitung}
\subsection{Schwellwert-basierte Segmentierung}
\label{sbSeg}
Um bessere Erkennungsergebnisse zu erhalten, sollen die wesentlichen Bereiche von der Umgebung getrennt werden. In dieser Arbeit wird eine Schwellwert-basierte Segmentierung bez�glich der 3D-Daten verwendet. Eine Schwellwert-basierte Segmentierung kann als eine Abbildung $f$ vom originalen Bild $I$ zum Ergebnisbild $H$ definiert werden:

\[
f: I \xmapsto{} H
\]

mit 

\begin{equation}
H_{ij} = 
\begin{cases}
1 & \text{f�r } I_{ij} > \Theta \\
0 & \text{sonst}
\end{cases}
\end{equation}

wobei $\Theta$ der eingegebene Schwellwert ist.
 
\subsection{Steuerung der Helligkeit und Kontrast}
Au�er der Umgebung beeinflussen die Helligkeit und der Kontrast die Qualit�t bzw. Stabilit�t der Markenerkennung. Deshalb ist die Optimierung dieser zwei Parameter in der Vorverarbeitungsphase notwendig. In dieser Arbeit wird ein affiner Operator auf jeden Punkt durchgef�hrt, um die geeignete Helligkeit bzw. den Kontrast zu bestimmen. Der affine Operator ist eine Abbildung von Originalbild $I$ zum Ergebnisbild $H$ mit:

\begin{equation}
H_{ij} = a I_{ij} + b,
\end{equation}

wobei die Parameter $a \in R^+$ und $b \in R$ jeweils den Kontrast und die Helligkeit kontrollieren. Es gibt vier M�glichkeiten f�r die verschiedenen Zuordnungen dieser Parameter:

\begin{itemize}
\item $a>1$, $b=0$     Kontrasterh�hung
\item $0<a<1$, $b=0$ Kontrastminderung
\item $a=1$, $b>0$     Helligkeitserh�hung
\item $a=1$, $b<0$     Helligkeitsminderung
\end{itemize}

\section{Markenerkennung}
Wie im Abschnitt \ref{mErkennung} beschrieben, sind viele Erkennungsalgorithmen in der Open Source Library OpenCV realisiert. Wegen verbreiteter Benutzung von OpenCV, werden der Vergleich dieser Algorithmen h�ufig gemacht. Im folgenden Abschnitt wird auf einen Vergleich eingegangen, um den geeigneten Algorithmus f�r diese Arbeit auszuw�hlen.
     
\subsection{Auswahl des Erkennungsalgorithmus}
\label{AusAlgo}

\begin{figure}[bft]
\centering
\includegraphics[scale=0.22]{Abbildungen/SampleBild}
\caption{Die vier Beispielbilder. Von links nach rechts sind Barbara, Lena, Peppers und Mandril. \cite{O11}}
\label{4Samp}
\end{figure}

\begin{figure}[bft]
\centering
\includegraphics[scale=0.7]{Abbildungen/Number-of-detected-features}
\caption{Die Anzahl der erkannten Merkmale von allen vier Beispielbildern durch verschiedene Erkennungsalgorithmen. \cite{O11}}
\label{Nodf}
\end{figure}

Odessa hat in seinem Blog einen sehr guten Vergleich f�r alle Erkennungsalgorithmen von OpenCV durchgef�hrt \cite{O11}. Vier h�ufig benutzte Beispielbilder wurden betrachtet (Siehe Abb.~\ref{4Samp}).
\\
\\
Von besonderem Interesse in seiner Arbeit ist der Vergleich �ber die Anzahl der betrachteten Punkte bzw. der durchschnittlichen Fehler. Wegen der verschiedenen Prinzipien erkennt der Algorithmus FAST viel mehr Merkmale als SURF und STAR(Name des CenSurE Algorithmus in OpenCV). Den Unterschied sieht man deutlich in der Abbildung~\ref{Nodf}. Je mehr Punkte betrachtet werden, desto mehr Rauschen wird in das System gebracht, weil viele normale Pixel auch als Merkmale erkannt werden. Das ist offensichtlich ein negativer Einfluss f�r die weitere Analyse der Daten. Abbildung~\ref{Ate} zeigt den durchschnittlichen Fehler in Pixeln von den Punktpaaren, der durch den gleichen Erkennungsalgorithmus von Bezugsbildern erkannt wird. Der Algorithmus STAR erzeugt deutlich weniger Fehler in der Erkennung, und das Ergebnis h�ngt auch leicht von der Eingabe ab. Wegen der niedrigeren Fehlerquote und der besseren Konzentration an gro�en kreisf�rmigen Merkmalen ist der STAR Algorithmus f�r die Erkennung der Objekte mit k�nstlichen Marken sehr geeignet.

\begin{figure}[bft]
\centering
\includegraphics[scale=0.7]{Abbildungen/Average-tracking-error}
\caption{Der durchschnittliche Fehler (in Pixeln) zwischen den assoziierten Punkten zweier folgender Bilder. \cite{O11}}
\label{Ate}
\end{figure}

\subsection{CenSurE Algorithmus}
Der Algorithmus CenSurE (Center Surround Extrema) wird von Agrawal et al. 2008 entwickelt \cite{AKB08}. Analog zum SIFT Algorithmus werden die Extreme zwischen den Skalenr�umen herausgefunden, welche als die Merkmale des betrachteten Bilds erkannt werden. Der Bi-Level Filter wird in dieser Arbeit anstelle der Differenz der Gaussian verwendet, um die Laufzeit zu reduzieren. Weiterhin wird die Approximation des gefilterten Bereichs von Bi-Level Filter durchgef�hrt. Dadurch kann das Algorithmus in Echtzeit laufen lassen, obwohl alle Merkmale in allen Skalenr�umen betrachtet werden sollen. 

%Sie verbessern die SIFT bzw. SURF Verfahren durch Ber�cksichtigung aller Merkmale in allen Skalenr�ume. Das Extremum durch die Skalen und Lagen werden ausgew�hlt, um die Merkmale zu bestimmen. Der Bi-level Filter wird hier statt Gaussian Filter verwendet, damit der Algorithmus in Echtzeit laufen kann, obwohl alle Merkmale in aller Skalenr�ume betrachtet werden sollen.

\subsubsection{Bi-level Filter}
Der Bi-level Filter ist eine einfache Approximation des Laplacian-Operators durch die Multiplikation der Bilder mit 1 und -1. Die Abbildung~\ref{bi-level} zeigt die Progression des Bi-level Filters mit verschiedenen Symmetrischen Stufen. Der kreisf�rmige Filter an linker Seite der Abbildung~\ref{bi-level} kann den Laplacian-Operator zwar am besten approximieren, ist aber leider schwierig zu implementieren. Deshalb werden die Progressionen der Filter durch Polygone angen�hert, wodurch die Berechnung vereinfacht werden kann. Zum Vergleich der �brigen Formen der Abbildung~\ref{bi-level} liefert der Filter mit Achtecken die beste Leistung und der Filter mit Rechtecken die k�rzeste Laufzeit. Diese Polygon-Filter k�nnen durch die integralen Bilder einfach dargestellt werden.  

\begin{figure}[bft]
\centering
\includegraphics[scale=0.6]{Abbildungen/Bi-level-filter2}
\caption{Progression der Center-Surround Bi-level Filter. Der Kreis ist die ideale voll-symmetrische Approximation der Laplacian. Die Filter daneben, von links nach rechts, haben eine niedrigere Symmetrie, brauchen aber weniger Zeit zur Berechnung. \cite{AKB08}}
\label{bi-level}
\end{figure}

\subsubsection{Integrale Bilder}
Ein integrales Bild $I$ ist eine mittlere Repr�sentation eines Bildes, welches die Summe der Grauwerte von Bild $N$ mit Breite $x$ und H�he $y$ enth�lt. 

\begin{equation}
I(x,y) = \sum _{x'=0}^x \sum_{y'=0}^y N(x',y')
\label{InteBild1}
\end{equation}   

Das integrale Bild ist rekursiv berechenbar und ben�tigt nur eine einmalige Durchf�hrung aller Pixel des Bildes. Mithilfe des integralen Bildes kann die Intensit�t des beliebigen rechteckigen Bereiches einfach durch vier Additionen berechnet werden. Die Erweiterung des integralen Bildes wird f�r die Berechnung der Polygonen-Filter benutzt. Die Kombination von zwei verschiedenen Bildern, die schr�g integriert werden, kann den f�r Polygone ben�tigten, trapezf�rmigen Bereich einfach darstellen. Die mathematische Beschreibung des schr�gen Integrals des Bildes ist:

\begin{equation}
I_\alpha(x,y) = \sum_{y'=0}^y \sum_{x'=0}^{x+\alpha(y-y')} N(x',y'),
\label{InteBild2}
\end{equation}

wobei $\alpha$ den schr�gen Winkel erkl�rt und wenn es 0 gleicht, ist die Formel \eqref{InteBild1} genauso wie die Formel \eqref{InteBild2}, und beschreibt ein rechteckiges integrales Bild. Die in der Abbildung~\ref{bi-level} gezeigte Achteck-Filter und Sechseck-Filter k�nnen mit jeweils drei bzw. zwei Trapezen schnell aufgebaut werden.
  
\subsubsection{Non-maximal Suppression}
Non-maximal Suppression ist eine Strategie, die das lokale Extremum finden kann. Die Response des Pixels wird unterdr�ckt, wenn es ein Pixel in seiner Nachbarschaft f�r die Lage bzw. die Skala gibt, dessen Response gr��er oder kleiner ist, als das zu betrachtende Pixel. Die Pixel mit entweder Maximum oder Minimum Response werden als Merkmale erkannt. Der Suchumfang wird in der Arbeit von Agrawal als 3x3x3 eingestellt, d.h. acht Pixel um den betrachteten Pixel und jeweils neun Pixel in zwei benachbarten Skalenr�umen werden zusammen ber�cksichtigt. Die Pixel mit h�herer Response k�nnen zwischen der Transformation des Bildes stabiler wiedererkannt werden, weshalb die ausgew�hlten Extrema nach der Non-maximal Suppression noch einmal durch einen Schwellwerte-Filter gefiltert werden sollen, um die besten Merkmale zu bestimmen. 
  
\subsection{Kalman-Filter}
\label{KF}
Der Kalman-Filter basiert auf einem linearen, dynamischen System in einem diskreten Zeitraum. Die Zustandsgleichung des Systems wird h�ufig durch eine Differenzengleichung beschrieben. In vielen F�llen werden die Zust�nde nur durch einen voneinander getrennten Zeitpunkt bestimmt. Kalman hat den Sonderfall der linearen Abh�ngigkeit der Zust�nde untereinander betrachtet, und die Zustandsgleichung zur linearen Differenzengleichung vereinfacht.

\subsubsection{Zustandsraummodellierung}
Der n�chste Systemzustand kann basierend auf dem aktuellen Systemzustand durch:

\begin{equation}
X_k = F_{k-1} X_{k-1} + B_{k-1} u_{k-1} + w_{k-1} 
\end{equation}

gesch�tzt werden. Der Index $k$ bzw. $k-1$ bezieht sich auf den Zeitpunkt $t_{k}$ und $t_{k-1}$, wobei $t_k = t_0 + k \Delta t$ und $t_0$ der Anfangszeitpunkt und $k$ eine nat�rliche Zahl von Interesse ist. Deshalb beschreibt der mehrdimensionale Vektor $X_k$ den Zustand des Systems zum Zeitpunkt $t_k$. Die Matrix $F_{x-1}$ ist die �bergangsmatrix f�r die zeitlich aufeinanderfolgenden Zust�nde $X_k$ und $X_{k-1}$. Die Matrix $B_{k-1}$ und der Kontrollvektor $u_{k-1}$ stellen den deterministischen Anteil der weiteren, �u�eren Einfl�sse auf das System dar.  Die zuf�lligen, nicht erfassbaren Komponenten der �u�eren Einfl�sse werden durch die stochastische Gr��e $w_{k-1}$ gesch�tzt, die einer Normalverteilung mit Mittelwert 0 und Kovarianz $Q_{k-1}$ folgt.

\[
w_{k-1} \sim N(0,Q_{k-1})
\]

Wegen dieser Zufallsvariable bilden die Menge aller Zustandsvektoren eine Markov-Kette, d.h. der Zustand zu einem Zeitpunkt $k$ h�ngt lediglich vom unmittelbaren zeitlichen Vorg�nger an $k-1$ ab.
\\
\\
Die Beobachtungen des Systems bestehen aus einer modellierbaren Verzerrung und einem unvorhersehbaren Messrauschen:

\begin{equation}
Z_k = H_k X_k + v_k.
\end{equation} 

$Z_k$ bezieht sich auf die Messung zum Zeitpunkt $k$. Die Multiplikation von der Beobachtungsmatrix $H_k$ und Zustandsvektor $X_k$ beschreibt die lineare Approximation der Verzerrung des Systems. Das Rauschen $v_k$ wird im Kalman-Filter als zeitlich unabh�ngig und normalverteilt angenommen:

\[
v_k \sim N(0,R_k).
\]

\subsubsection{Das Kalman-Filter}
Das Ziel eines Filters besteht darin, die Zust�nde durch die Informationen einer Messreihe besser sch�tzen zu k�nnen. Da die Rauschterme $w$ und $v$ f�r alle Zeit die Normalverteilung erf�llen, k�nnen die zeitdiskreten Zust�nde $X_k$ auch durch eine Normalverteilung mit dem Mittelwert $\hat{x}_k$ und der Kovarianz $\hat{P}_k$ ermessen werden.

\[
\hat{X}_k \sim N(\hat{x}_k , \hat{P}_k)
\]

Die Idee des Kalman-Filters ist es, eine rekursive Formulierung aufzubauen, die aber nur die Sch�tzung eines vorherigen Zeitpunktes und die aktuelle Messung ben�tigt, um die Sch�tzung des aktuellen Zeitpunktes zu bestimmen. Es gibt haupts�chlich zwei Phasen im Kalman-Filter.

\textbf{Pr�diktion}
\\
In ersten Schritt dieser Phase wird eine vorangegangene Sch�tzung $X_{k|k-1}$ f�r den aktuellen Zeitpunkt vorausgesagt.

\begin{equation}
\hat{x}_{k|k-1} = F_{k-1} \hat{x}_{k-1} + B_{k-1} u_{k-1}
\end{equation}

Die Indizierungsschreibweise $k|k-1$ bezieht sich auf die Bedingtheit zu den Zeitpunkten $k$ und $k-1$. F�r die Kovarianz gilt:

\begin{equation}
\hat{P}_{k|k-1} = F_{k-1} \hat{P}_{k-1} F_{k-1}^T + Q_{k-1}.
\end{equation}

\textbf{Korrektur}
\\
Die Vorhersagen von letztem Schritt werden hier durch die neue Messung korrigiert:

\begin{equation}
\hat{x}_k = \hat{x}_{k|k-1} + \hat{K}_k \tilde{y}_k,
\end{equation}
 
\begin{equation}
\hat{P}_k = \hat{P}_{k|k-1} - \hat{K}_k S_k \hat{K}_k^T.
\end{equation}

Die Hilfsgr��e der Innovation $\tilde{y}_k$ beschreibt, wie genau die aktuellen Messungen von der vorhergesagten Sch�tzungen mithilfe der Beobachtungsgleichung approximiert werden:

\[
\tilde{y}_k = Z_k - H_k \hat{x}_{k|k-1}.
\]

$S_k$ bezieht sich auf die Residualkovarianz, wobei gilt:

\[
S_k = H_k \hat{P}_{k|k-1} H_k^T + R_k
\]

und $\hat{K}_k$ ist die zugeh�rige Kalman-Matrix:

\[
\hat{K}_k = \hat{P}_{k|k-1} H_k^T S_k^{-1}.
\]

\section{Markenverfolgung}
Die Marken, die vom Markendetektor erkannt werden, sollen w�hrend der Bildsequenz verfolgt werden. Damit kann die Transformation des betrachteten Objekts zwischen je zwei Bildern bestimmt werden, d.h. die gleichen Marken von verschiedenen Bildern sollen zuerst erkannt werden. In dieser Arbeit wird das Singul�rwertzerlegungsverfahren verwendet, um diese Korrespondenzpunktpaare herauszufinden.

\subsection{Singul�rwertzerlegung}
Sei M eine komplexe $m \times n$ Matrix mit Rang $r$. Die Singul�rwertzerlegung bezeichnet dann das Produkt:

\begin{equation}
M = U \Sigma V^*
\end{equation}

wobei $U$ eine unit�re Matrix mit Gr��e $m \times m$ und $V^*$ die Adjungierte einer unit�re Matrix mit Gr��e $n \times n$ ist. $\Sigma$ bezieht sich auf eine $m \times n$ Diagonalmatrix:

\[
\Sigma = 
\begin{pmatrix}
\sigma_1 & & & \vline & & \vdots & \\
& \ddots & & \vline & \cdots & 0 & \cdots \\
& & \sigma_r & \vline & & \vdots & \\
\hline
& \vdots & & \vline & & \vdots & \\
\cdots & 0 & \cdots & \vline & \cdots & 0 & \cdots \\
& \vdots & & \vline & & \vdots & 
\end{pmatrix}
\]

mit $\sigma_1 \geq \cdots \geq \sigma_r > 0$, wobei $\sigma_i , i=1,\ldots , r$ als die Singul�rwerte von $\Sigma$ genannt werden.

\subsection{Korrespondenzuntersuchung durch Singul�rwertzerlegung}
\label{KdS}
Mithilfe der Singul�rwertzerlegung haben Scott und Longuet-Higgins einen Algorithmus zur Bestimmung der assoziierenden Merkmale entwickelt \cite{SL91}. Seien $I$ und $J$ zwei nachfolgende Bilder und haben jeweils $m$ und $n$ Merkmale, die als $I_i (i=1,\ldots ,m)$ und $J_j (j=1,\ldots ,n$) bezeichnet werden, dann wird eine $m \times n$ Matrix $G$ mit den Elementen

\[
G_{ij} = exp(- \frac{r_{ij}^2}{2 \sigma^2})
\]

definiert, wobei $r_{ij}$ den Abstand zwischen den Merkmalen $I_i$ und $J_j$ beschreibt. $\sigma$ wird als ein Standard f�r den Abstand definiert, wodurch das Vergr��ern oder Verkleinern der Verschiebung des Objekts gesch�tzt werden kann. Der Wert von $G_{ij}$ nimmt durch die Erh�hung der Distanz von 1 bis 0 monoton ab. Der zweite Schritt des Algorithmus von Scott und Longuet-Higgins ist die Singul�rwertzerlegung der Matrix $G$:

\[
G = TDU.
\]

wobei $T$ und $U$ unit�re Matrizen mit den jeweiligen Gr��en $m \times m$ und $n \times n$ sind, und $D$ eine Diagonalmatrix bildet. Sei $E$ eine neue Matrix mit gleicher Gr��e von $D$, in der aber jedes diagonale Element als 1 ersetzt wird. Nach Austausch der Matrix $D$ durch Matrix $E$ erh�lt man eine neue orthogonale Matrix:

\[
P = TEU
\]

Die Aufgabe des dritten Schritts ist das Element $P_{ij}$ zu finden, welches gleichzeitig das Maximum der Reihe und Spalte ist. Wenn $P_{ij}$ diese Bedingung erf�llt, sagt man, dass es eine Eins zu Eins Korrespondenz zwischen den Merkmalen $I_i$ und $J_j$ gibt. Der ganze Algorithmus ist in Algorithmus~\ref{alg1} aufgef�hrt, der durch die Arbeit von Scott und Longuet-Higgins \cite{SL91} erzeugt wurde. 

\begin{algorithm}                      % enter the algorithm environment
\caption{Bestimmung der Korrespondenz der Merkmalen von zwei Bildern}          % give the algorithm a caption
\label{alg1}                           % and a label for \ref{} commands later in the document
\begin{algorithmic}                    % enter the algorithmic environment
    \State $I,J,\sigma, Result$
    \For{$i=1 \to m$, $j=1 \to n$}
    	\State $r_{ij} \gets Dis(I_i, J_j)$
    	\State $G_{ij} \gets exp(-\frac{r_{ij}^2}{\sigma^2})$
    \EndFor
    \State $T,U \gets$ Singul�rwertzerlegung von G
    \State $E \gets m \times n$ Diagonalmatrix mit $E_{ii} = 1$
    \State $P \gets TEU$
    \For{$i=1 \to m$}
    	\State $MaxSpalteIndex[i] \gets$ Index der Spalte des maximalen Elements an Reihe $i$.
    	%\State $MaxSpalteIndex_i \gets Max$ 
    \EndFor
    \For{$i=1 \to m$}
    	\If {$P_{iMaxSpalteIndex[i]}$ ist Maximum der Spalte $MaxSpalteIndex[i]$}
    		\State $Result \gets$ Punktpaar($I_i, J_{MaxSpalteIndex[i]}$) 
    	\EndIf
    \EndFor 
\end{algorithmic}
\end{algorithm}


\section{Bestimmung der Transformation}
Es gibt viele M�glichkeiten, die Lage eines Objekts im dreidimensionalen Raum zu sch�tzen. In dieser Arbeit wird das Verfahren von Horn \cite{H87} verwendet, das durch Einheitsquaternionen die Transformation eines Objekts zwischen zwei Zeitpunkten bestimmen kann.

\subsection{Quaternionen}
%Das Quaternion
Ein Quaternion besteht aus einem Vektor mit vier Elementen, wobei ein Element als ein Skalar bezeichnet und die anderen drei eine Richtung im dreidimensionalen Raum beschreiben. Quaternionen k�nnen aber auch als eine Erweiterung der komplexen Zahlen betrachtet werden, deren Imagin�rteil nach drei neuen Zahlen $i$, $j$ und $k$ entwickelt wird. Eine Normalform einer Quaternion ist gegeben durch:

\[
q = q_0 + i q_x + j q_y + k q_z,
\]

wobei $i$, $j$ und $k$ die sogenannte Hamilton-Regeln erf�llen:

\[
i^2 = j^2 = k^2 = i j k = -1, 
\]
\[
ij = k, \quad jk = i, \quad ki = j, 
\]
\[
ji = -k, \quad kj = -i, \quad ik = -j.
\] 

Eine andere Form mit getrenntem Realteil und Imagin�rteil wird definiert durch:

\begin{equation}
\label{QForm2}
q = (q_0, \vec{q}),
\end{equation}

wobei $q_0 \in \mathbb{R}$ ein Skalar und $\vec{q} \in \mathbb{R}^3$ ein Vektor ist. 
\\
\\
Sei $r$ eine andere Quaternion mit:

\[
r = r_0 + i r_x + j r_y + k r_z.
\]

Analog zu Vektoren im $\mathbb{R}^4$ wird das Skalar Produkt zwischen zwei Quaternionen definiert als:

\[
\langle q, r \rangle := q \cdot r := q_0 r_0 + q_x r_x + q_y r_y + q_z r_z.
\]

Weiterhin kann die Quaternion Multiplikation mithilfe von \eqref{QForm2} berechnet werden als:

\begin{align}
qr & = (q_0 r_0 - \vec{q} \cdot \vec{r} , \quad q_0 \vec{r} + \vec{q} r_0 + \vec{q} \times \vec{r}) \\
\label{QMulti}
& = (q_0 r_0 - q_x r_x - q_y r_y - q_z r_z) \nonumber \\
& + i ( q_0 r_x + q_x r_0 + q_y r_z - q_z r_y ) \nonumber \\
& + j ( q_0 r_y - q_x r_x + q_y r_0 + q_z r_z ) \nonumber \\
& + k ( q_0 r_z + q_x r_y - q_y r_x + q_z r_0 ).
\end{align} 

Die rechte Multiplikation von $r$ in Formel \eqref{QMulti} kann aber auch zu einer links multiplizierten Matrix umgeschrieben werden:

\begin{equation}
qr = 
\begin{pmatrix}
r_0 & -r_x & -r_y & -r_z \\
r_x & r_0 & r_z & -r_y \\
r_y & -r_z & r_0 & r_x \\
r_z & r_y & -r_x & r_0
\end{pmatrix}
 = \mathbf{R} q.
\end{equation}

Die konjugierte Quaternion von $q$ ist definiert als:

\[
\overline{q} = q_0 - i q_x - j q_y - k q_z.
\]

Das Produkt einer Quaternion und dessen Konjugierte ist eine nicht negative reelle Zahl:

\[
q \cdot \overline{q} = q_0^2 + q_x^2 + q_y^2 + q_z^2. 
\]

Mithilfe der konjugierten Quaternion kann man die L�nge der Quaternion $|q|$ definieren:

\[
|q| = \sqrt{q \cdot \overline{q}}.
\]

Ist die L�nge einer Quaternion gleich 1, nennt man die Quaternion eine Einheitsquaternion. F�r eine Einheitsquaternion gilt:

\[
q \cdot \overline{q} = 1 \iff \overline{q} = q^{-1}.
\]

D.h. die Inverse und Konjugierte sind identisch. F�r jede Einheitsquaternion $q \neq \pm 1$ gibt es eine entsprechende Polardarstellung:

\begin{equation}
q = \cos \alpha + \mathit{v} \cdot \sin \alpha
\label{polarQ}
\end{equation}

mit $\alpha = \arccos (q_0) \in (0,\pi)$ und $v = \frac{1}{\sin \alpha} (i q_x + j q_y + k q_z)$. 

\subsection{Beschreibung der Drehungen im Dreidimensionalen Raum mit Quaternionen}
Die Drehungen im dreidimensionalen Raum k�nnen durch die Einheitsquaternionen sehr gut beschrieben werden. Eine Abbildung der Rotation $\rho_q$ kann in folgender Form definiert werden:

\[
\rho_q : x \rightarrow q x \overline{q},
\] 

wobei $q$ eine Einheitsquaternion und $\overline{q}$ dessen Konjugierte ist. Mithilfe der Polardarstellung \eqref{polarQ} kann die Abbildung $\rho_q$ sich auf eine Drehung im $\mathbb{R}^3$ um die Achse $\mathit{v}$ mit Winkel $2\alpha \in (0, 2\pi)$ beziehen. Die entsprechende orthogonale Matrix von $q$ ist

\begin{equation}
\label{QuaR}
R =
\begin{pmatrix}
q_0^2 + q_x^2 - q_y^2 - q_z^2 & 2q_x q_y - 2q_0 q_z           & 2q_x q_z + 2q_0 q_y \\
2q_x q_y + 2q_0 q_z           & q_0^2 - q_x^2 + q_y^2 - q_z^2 & 2q_y q_z - 2q_0 q_z \\
2q_x q_z - 2q_0 q_y           & 2q_y q_z + 2q_0 q_z           & q_0^2 - q_x^2 - q_y^2 + q_z^2,
\end{pmatrix}
\end{equation} 

was zur Drehgruppe SO(3) geh�rt und eine Drehung in der Matrixform repr�sentiert.

\subsection{Orientierung mit Einheitsquaternion}
\label{QmE}
Das Verfahren f�r die Orientierung der Objekte mithilfe der Quaternionen wurde von Horn im Jahre 1987 ver�ffentlicht \cite{H87}. Seien $D$ und $M$ zwei Punktmengen mit gleicher Gr��e $n$. Dann kann die Transformation zwischen den Punkten von zwei Mengen formuliert werden als:

\begin{equation}
d_i = \mathbf{R} m_i + \mathbf{T} + e_i,
\end{equation}

wobei $d_i$ und $m_i$ die i-ten Punkte der Punktmengen $D$ bzw. $M$ bezeichnen. $\mathbf{R}$ ist die Rotationsmatrix und $\mathbf{T}$ ist die Translationsmatrix. $e_i$ beschreibt den Fehler f�r die Transformation, und kann umformuliert werden als:

\begin{equation}
\label{QError}
e_i = d_i - \mathbf{R} m_i + \mathbf{T}.
\end{equation}

Das Ziel des Verfahrens ist, eine Rotations- bzw. Transformationsmatrix mit minimalem Fehler zu finden. Dadurch wird die Summe des Quadrats von $e_i$ betrachtet:

\begin{equation}
\sum_{i=1}^n \| e_i \|^2 = \sum_{i=1}^n \| d_i - \mathbf{R} m_i + \mathbf{T} \|^2. 
\end{equation} 

Seien $\overline{d}$ und $\overline{m}$ jeweils die Schwerpunkte der Punktmengen $D$ und $M$. Dann gilt

\begin{equation}
\label{QuaSch}
\overline{d} = \frac{1}{n} \sum_{i=1}^n d_i 
\quad , \quad
\overline{m} = \frac{1}{n} \sum_{i=1}^n m_i.
\end{equation}

Der Abstand von jedem Punkt zum Schwerpunkt wird berechnet als:

\begin{equation}
d_i' = d_i - \overline{d}
\quad , \quad
m_i' = m_i - \overline{m}.
\end{equation}

und die Summe der Abst�nde erf�llt nat�rlich

\begin{equation}
\label{QDis}
\sum_{i=1}^n d_i' = 0
\quad und \quad
\sum_{i=1}^n m_i' = 0.
\end{equation}

Dann kann der Fehler in Formel \eqref{QError} mit den Abst�nden zum Schwerpunkt $\overline{d}$ und $\overline{m}$ umgeschrieben werden:

\begin{equation}
e_i = d_i' - \mathbf{R} m_i' + \mathbf{T}',
\end{equation}

wobei $\mathbf{T}'$ als 

\[
\mathbf{T}' = \mathbf{T} - \overline{d} + \mathbf{R} \overline{m}
\]

definiert wird. Analog kann die Summe des Quadrats des Fehlers neu formuliert werden.

\begin{align}
\label{QError2}
\sum_{i=1}^n \| e_i \|^2 &=
\sum_{i=1}^n \| d_i' - \mathbf{R} m_i' + \mathbf{T}' \|^2 \nonumber \\
&= \sum_{i=1}^n \| d_i' - \mathbf{R} m_i' \|^2 - 2\mathbf{T}' \cdot \sum_{i=1}^n (d_i' - \mathbf{R} m_i')
+ n \| \mathbf{T}' \|^2
\end{align}

Wegen \eqref{QDis} ist der zweite Term gleich 0. Der dritte Term kann nicht negativ werden und wird auch 0 sein, wenn der gesamte Fehler minimiert wird, d.h.:

\begin{align}
\label{QTran}
& \mathbf{T}' = \mathbf{T} - \overline{d} + \mathbf{R} \overline{m} = 0 \nonumber \\
\Rightarrow & \mathbf{T} = \overline{d} + \mathbf{R} \overline{m}.
\end{align}

Die Formel \eqref{QTran} berechnet direkt die Translationsmatrix durch die Rotationsmatrix und die Schwerpunkte der beiden Punktmengen. Der erste Term von \eqref{QError2} kann zu

\begin{equation}
\sum_{i=1}^n \| d_i' - \mathbf{R} m_i' \|^2 = \sum_{i=1}^n (d_i'^t d_i' + m_i'^t m_i' - 2 d_i'^t \mathbf{R} m_i')
\end{equation}

weiter formuliert werden. Dann wird die Minimierung des Fehlers durch die Bestimmung des Maximums der Summe

\begin{equation}
\sum_{i=1}^n d_i'^t \mathbf{R} m_i'
\end{equation}

erreicht. Durch Ersetzen der Rotationsmatrix mit Quaternion $q$ wird das maximierte Problem umformuliert als:

\begin{equation}
\sum_{i=1}^n (q m_i'' \overline{q}) \cdot d_i'',
\end{equation}

wobei $\overline{q}$ das konjugierte Quaternion von $q$ ist, $m_i'' = (0, m_{i,x}', m_{i,y}', m_{i,z}')$ und $d_i'' = (0, d_{i,x}', d_{i,y}', d_{i,z}')$ die erweiterte Quaternion f�r Punkte $m_i'$ bzw. $d_i'$ sind.
Dann gilt:

\begin{align}
\label{qmdq}
\sum_{i=1}^n (q m_i'' \overline{q}) \cdot d_i''
& = \sum_{i=1}^n (q m_i'' \overline{q}) \cdot ( d_i'' q \overline{q}) \nonumber \\
& = \sum_{i=1}^n (q m_i'') \cdot (d_i'' q).
\end{align}

Die beiden Multiplikationen in Klammen k�nnen als Formel \eqref{QMulti} zum Produkt von einem Quaternion und einer Matrix umschrieben werden:

\[
q m_i'' = 
\begin{pmatrix}
0 & -m_{i,x}' & -m_{i,y}' & -m_{i,z}' \\
-m_{i,x}' & 0 & -m_{i,z}' & -m_{i,y}' \\
-m_{i,y}' & -m_{i,z}' & 0 & -m_{i,x}' \\
-m_{i,z}' & -m_{i,y}' & -m_{i,x}' & 0
\end{pmatrix}
 = \mathbf{M}_i q
\]

und 

\[
d_i'' q = 
\begin{pmatrix}
0 & -d_{i,x}' & -d_{i,y}' & -d_{i,z}' \\
d_{i,x}' & 0 & -d_{i,z}' & d_{i,y}' \\
d_{i,y}' & d_{i,z}' & 0 & -d_{i,x}' \\
d_{i,z}' & -d_{i,y}' & d_{i,x}' & 0
\end{pmatrix}
 = \mathbf{D}_i q.
\]

Dann kann \eqref{qmdq} weiter abgeleitet werden:

\begin{align}
\label{qNq}
 \sum_{i=1}^n ( \mathbf{M}_i q) \cdot ( \mathbf{D}_i q)
 & = \sum_{i=1}^n q^t \mathbf{M}_i^t \mathbf{D}_i q \nonumber \\
 & = q^t \Big( \sum_{i=1}^n \mathbf{M}_i^t \mathbf{D}_i \Big) q \nonumber \\
 & = q^t \Big( \sum_{i=1}^n \mathbf{N}_i \Big) q \nonumber \\
 & = q^t \mathbf{N} q,
\end{align}
 
wobei $\mathbf{N}_i = \mathbf{M}_i^t \mathbf{D}_i$ ist, und $\mathbf{N}$ die Summe von $\mathbf{N}_i$ beschreibt. Sei $\mathbf{H}$ die Summe der Kreuzprodukte des Punktpaars von Punktmengen $D$ und $M$:

\begin{equation}
\label{QuaH}
\mathbf{H} = \sum_{i=1}^n m_i' d_i'^t.
\end{equation}

Es ist deutlich, dass die Gr��e der Matrix $\mathbf{H}$ 3 $\times$ 3 ist, deshalb kann $\mathbf{H}$ auch als

\begin{equation}
\mathbf{H} = 
\begin{pmatrix}
S_{xx} & S_{xy} &S_{xz} \\
S_{yx} & S_{yy} &S_{yz} \\
S_{zx} & S_{zy} &S_{zz} 
\end{pmatrix}
\end{equation}  
  
geschrieben werden, wobei

\[
S_{xx} = \sum_{i=1}^n m_{i,x}' d_{i,x}' \quad ,\quad
S_{xy} = \sum_{i=1}^n m_{i,x}' d_{i,y}' \quad , \quad \cdots
\]

Dann kann die Matrix $\mathbf{N}$ im \eqref{qNq} durch die Elemente von $\mathbf{H}$ dargestellt werden als:

\begin{equation}
\label{QuaN}
\mathbf{N} = 
\begin{pmatrix}
S_{xx} + S_{yy} + S_{zz} & S_{yz} - S_{zy} & S_{zx} - S_{xz} & S_{xy} - S_{yx} \\
S_{yz} - S_{zy} & S_{xx} - S_{yy} - S_{zz} & S_{xy} + S_{yx} & S_{zx} + S_{xz} \\
S_{zx} - S_{xz} & S_{xy} + S_{yx} & -S_{xx} + S_{yy} - S_{zz} & S_{yz} + S_{zy} \\
S_{xy} - S_{yx} & S_{zx} + S_{xz} & S_{yz} + S_{zy} & -S_{xx} - S_{yy} + S_{zz} 
\end{pmatrix}
\end{equation}

Nach dem Beweis von Horn \cite{H87} wird die Formel \eqref{qNq} genau dann maximal, wenn $q$ der Eigenvektor der Matrix $\mathbf{N}$ ist, der zu dem maximalen, positiven Eigenwert entspricht. 

\section{Objekterkennung}
Die gesamte Objekterkennung kann in zwei Phasen aufgeteilt werden: die Segmentierung und der Vergleich des Strukturgraphen. In diesem Abschnitt werden die wichtigsten Algorithmen f�r beide Teile erkl�rt. 

\subsection{DBSCAN}
\label{Dbscan}
DBSCAN, kurz f�r den englischen Namen ,,Density-Based Spatial Clustering of Applications with Noise'', ist ein auf Dichte basierter Data-Mining-Algorithmus \cite{E96}. Die Hauptidee des Algorithmus h�ngt stark von dem Begriff ,,Dichteverbundenheit'' ab, der durch folgende Definitionen erkl�rt wird.
\\
\\
Seien $D$ eine Punktmenge im Raum $\mathbb{R}^n$ und $Dist(p,q)$ eine darauf definierte Distanzfunktion. $\epsilon$ und MinPts sind zwei Eingaben des Algorithmus.

\begin{definition}
\textsf{$\epsilon$-Umgebung} $N_{\epsilon}(p)$ ist eine Menge der Punkte um $p$, die 
\[
N_{\epsilon}(p) = \{ q \in D | Dist(p,q) \geq \epsilon \}
\]
\end{definition}

erf�llt.

\begin{definition}
Ein Punkt $p$ ist \textsf{direkt Dichte-erreichbar} zum Punkt $q$, g.d.w:
\begin{align*}
1) \quad & p \in N_{\epsilon}(q)  \nonumber \\
2) \quad & | N_{\epsilon}(q) | \geq MinPts \nonumber
\end{align*}
\end{definition}

\begin{definition}
Ein Punkt $p$ ist \textsf{Dichte-erreichbar} zum Punkt $q$, g.d.w es eine Kette von Punkten $p_1, \ldots , p_n$ mit $p_1 = p$ und $p_n = q$ gibt, wobei $p_{i+1}$ \textsf{direkt Dichte-erreichbar} zum $p_i$ f�r alle $i \in [1,n]$ ist.
\end{definition}

\begin{definition}
Zwei Punkte $p$ und $q$ hei�en \textsf{Dichte-verbunden}, g.d.w es einen Punkt $o$ gibt, wobei $p$ und $q$ jeweils zu $o$ \textsf{Dichte-erreichbar} sind.
\end{definition}

Mithilfe dieser Definitionen der Beziehung zwischen den Punkten kann man eine einzige Definition des Clusters ausgeben.

\begin{definition}
Sei $C$ eine nicht leere Teilmenge von $D$. Dann hei�t $C$ ein Cluster, wenn die folgenden zwei Bedingungen erf�llt werden:
\begin{align*}
1) \quad & \text{f�r $\forall p, q \in D$, wenn $p \in C$ und $q$ ist \textsf{Dichte-erreichbar} zum $p$, dann gilt } q \in C, \nonumber \\
2) \quad & \text{f�r $\forall p, q \in C$, $p$ ist \textsf{Dichte-verbunden} zu $q$.} \nonumber
\end{align*}
\end{definition}

Dann k�nnen alle Punkte von $D$ in drei verschiedene Typen zusammengefasst werden.

\begin{itemize}
\item Kernpunkt: Der Punkt, dessen $\epsilon$-Umgebung gr��er als MinPts ist.
\item Grenzpunkt: Der Punkt, dessen $\epsilon$-Umgebung nicht gro� genug ($<$MinPts), aber zu anderen Punkten Dichte-erreichbar ist.
\item Ger�usch: Der Punkt, der zu keinem Cluster geh�rt.  
\end{itemize}

\begin{figure}[htbp]
\centering
\includegraphics[scale=0.3]{Abbildungen/DBSCAN}
\caption{Ein Beispiel f�r die drei Typen der Punkte in DBSCAN-Algorithmus. Der rote Punkt A ist ein Kernpunkt. Die gelbe Punkte B und C sind die Grenzpunkte, die von roten Punkten Dichte-erreichbar sind. Wegen fehlender Dichte-Verbindung zwischen Punkt N und anderen Punkten, wird N als Rauschen erkannt. \cite{DWiki}}
\label{DBSCAN}
\end{figure}

Abbildung~\ref{DBSCAN} zeigt ein Beispiel f�r alle diese drei Typen eines Punkts, wobei die Kernpunkte mit Rot, Grenzpunkte mit Gelb und Rauschen mit Blau dargestellt werden. Die Kreise zeigen die $\epsilon$-Umgebungen f�r die Punkte an ihren Urspr�ngen.
\\
\\
Der Algorithmus durchl�uft f�r jeden Punkt von Punktmenge $D$ und die L�sung h�ngt von der Reihenfolge der Punkte nicht ab. Ein Kernpunkt soll zuerst gefunden werden, und dann die andere Punkte in ihrer $\epsilon$-Umgebung betrachtet werden. Zwei $\epsilon$-Umgebungen werden verkn�pft, wenn sie mindesten einen identischen Punkt haben. Der genaue Ablauf des DBSCAN-Algorithmus ist in Algorithmus~\ref{algDBSCAN} ausgegeben.

\begin{algorithm}[htbp]                      % enter the algorithm environment
\caption{DBSCAN}          % give the algorithm a caption
\label{algDBSCAN}                      % and a label for \ref{} commands later in the document
\begin{algorithmic}                    % enter the algorithmic environment
	\State $D, \epsilon, MinPts$
	\State Setzen $C$ zum ersten Cluster
	\For {jede $P_i \in D$}
		\If {$P_i$ ist nicht besucht}
			\State Markieren $P_i$ als besucht
			\State $N = \{ P_j \in D | Dist(P_i, P_j)<\epsilon \}$
			\If {Anzahl der Elemente von $N$ $<$ MinPts}
				\State Markieren $P_i$ als Ger�usch
			\Else
				\State F�gen $P_i$ in aktuellem Cluster $C$ ein
				\For {jede $P_j \in N$}
					\If {$P_j$ ist noch nicht besucht}
						\State Markieren $P_j$ als besucht
						\State $N' = \{ P_k \in D | Dist(P_j, P_k)<\epsilon \}$
						\If {Anzahl der Elemente von $N'$ $\geq$ MinPts}
							\State $N = N \cup N'$
						\EndIf 
					\EndIf
					\If {$P_j$ geh�rt zu keinem Cluster}
						\State F�gen $P_j$ in aktuellem Cluster $C$ ein
					\EndIf
				\EndFor
				\State Setzen $C$ zum n�chsten Cluster
			\EndIf
		\EndIf
	\EndFor
\end{algorithmic}
\end{algorithm}

\subsection{Teilgraph Isomorphismus}
\label{TI}
Die Objekterkennung bzw. Objektverfolgung wird durch den Vergleich der Modellgraphen und Eingabegraphen realisiert. Ein f�r diese Arbeit hilfreicher Algorithmus wurde von Rhijn und Mulder entwickelt \cite{AJ05}. Um das Problem des Graph Isomorphismus zu vereinfachen, wird nur ein Teilgraph $S_{min}$ von Modellgraphen betrachtet. $S_{min}$ soll ein vollst�ndiger Graph sein, um einen Modellgraphen eindeutig bestimmen zu k�nnen. Zuerst wird ein Punkt $p_i$ vom Eingabegraph ausgew�hlt und die Distanzen von diesem zu allen seinen Nachbarn berechnet. S�mtliche Kantenl�ngen werden mit den Kanten der Modellgraphen verglichen, damit ein Kandidatenpunkt $v_k$ im Modellgraph gefunden werden kann, der genug Kanten mit dem ausgew�hlten Punkt $p_i$ identisch hat. Zweitens sollen drei Nachbarn von $p_i$ gefunden werden, die mit $p_i$ zusammen einen vollst�ndigen Graphen (Pyramide) darstellen k�nnen. Wenn ein Teilgraph von Modellgraphen zum obigen vollst�ndigen Graph assoziiert wird, ist ein Teilgraph Isomorphismus zwischen den Eingabegraphen und Modellgraphen gefunden. Der Algorithmus~\ref{isoAlgo} gibt eine genauere Beschreibung des Algorithmus von Rhijn und Mulder an.

\begin{algorithm}
\caption{Teilgraph Isomorphismus}
\label{isoAlgo}
\begin{algorithmic}
	\State Modellgraph:$G_m$, Eingabegraph:$G_d$
	\For {jede $p_i \in G_d$}
		\For {jeder Nachbar $p_j$ von $p_i$}
			\For {alle Punktpaare $(v_k, v_l) \in G_m$} 
				\If {$dist(p_i, p_j) = dist(v_k, v_l)$}
					\State F�gen assoziierte Punktpaar $<p_j, v_l>$ zum $S_i$ ein
				\EndIf 
			\EndFor
			\If {$\| S_i \| \geq \| S_{min} \|$}
				%\State F�gen den Punkt $<p_i, v_k>$ zum Kandidaten $K$ ein
				\If {Drei Punktpaare in $S_i$ gefunden k�nnen, damit deren ersten Punkte mit $p_i$ ein Pyramide aufbauen k�nnen}
					\State Teilgraph Isomorphismus ist gefunden
					\State Break
				\EndIf
			\EndIf 
		\EndFor
	\EndFor
\end{algorithmic}
\end{algorithm}



\chapter{Implementierung}

\begin{figure}[ftb]
\centering
\includegraphics[scale=0.55]{Abbildungen/HauptAblauf.png}
\caption{Ablaufdiagramm}
\label{hAblauf}
\end{figure}

Die Implementierung kann in vier Hauptmodule unterteilt werden:

\begin{itemize}
\item Markenanalyse,
\item Objekteinlernen,
\item Objekterkennung und Verfolgung,
\item Bilder Steuerung.
\end{itemize}

Der allgemeine Ablaufprozess des Programms ist in Abbildung~\ref{hAblauf} gegeben. Das ganze Programm ist eine Schleife, in jeden deren Schritte die Information der PMD Kamera angesammelt und als Eingabe genutzt wird. Nach Bewertung der neuen Eingabe bzw. der R�ckkopplung vom letzten Ablauf werden die entsprechenden Bilddaten vom Modul {\itshape Bilder Steuerung} f�r die weiteren Schritte ausgew�hlt. Das Modul {\itshape Markenanalyse} analysiert die eingegebenen Bilddaten und versucht die Marken zu finden und zu verfolgen. Die gefundenen Marken werden entweder von dem {\itshape Objekteinlernen} oder durch die {\itshape Objekterkennung und Verfolgung} benutzt, das am Anfang des Programms durch den Benutzer entschieden wird. In der Einlernphase wird zuerst die Transformation des Objekts bestimmt, und dann der charakteristische Graph des Objekts durch die Marken dargestellt. Wenn man ein Objekt wieder erkennen m�chte, werden die Marken vom neuen Objekt zu den vorhandenen charakteristischen Graphen verglichen, die nach dem Objekteinlernen im Speicher hinterlegt werden. Das Ergebnis wird in einem OpenGL Fenster angezeigt und als R�ckkopplung f�r den n�chsten Schritt genutzt. Alle diese Module werden in den folgenden Unterabschnitten genau erkl�rt. Ein ausf�hrliches Ablaufdiagramm wird in Abbildung~\ref{gAblauf} gezeigt.  

\begin{figure}
\centering
\includegraphics[scale=0.7]{Abbildungen/Ablaufsbild.png}
\caption{Ausf�hrliches Ablaufdiagramm}
\label{gAblauf}
\end{figure}

\section{Markenanalyse}
\label{MAna}
Zuerst m�ssen alle Marken aus den Eingabebildern entweder f�r das Objekteinlernen oder die Objekterkennung und Verfolgung erkannt werden. In diesem Abschnitt werden alle daf�r ben�tigten Verfahren und Algorithmen vorgestellt. Die Objekte sollen aus den Eingabebildern vor dem Lernen segmentiert werden, damit die St�rung von der Umgebung vermieden werden kann. Danach  vergr��ert die Helligkeitssteuerung den Unterschied zwischen den Marken und anderen Bildpunkten. Daraufhin wird der STAR Detektor durchgef�hrt. Dadurch k�nnen viele Merkmale �ber die k�nstlichen Marken erkannt werden. Da der Detektor keine 1 zu 1 Korrespondenz zwischen den realen Marken und den erkannten Merkmalen garantiert, ist die zus�tzliche Kombination der Merkmale f�r gleiche Marke notwendig. Die Segmentierung verteilt die Marken auf die verschiedenen Objekte und die Verfolgung liefert die Abh�ngigkeit der Marken zwischen den unterschiedlichen Bildern.

\subsection{Bildvorverarbeitung}

\subsubsection{Datenstruktur des Eingabebilds}
Wie im Abschnitt \ref{PMDData} erkl�rt, liefert die PMD Kamera insgesamt vier verschiedenen Arten der Vermessungsdaten. Alle diese Daten eines Bildes werden in einer Instanz von Klassen \textbf{BildData} gespeichert. Dazwischen sind die Amplituden und 3D-Koordinaten von hoher Relevanz und werden in verschiedenen Teilen des Programms verwendet: die Amplituden werden f�r die Markenerkennung und die 3D-Koordinaten mit r�umlicher Information werden sp�ter f�r die Korrespondenzuntersuchung und Sch�tzung der Lage des Objekts benutzt. Da die definierte Dimension und das definierte Intervall dieser zwei Arten von Daten sich von anderen unterscheiden, ist vor der weiteren Bildverarbeitung dieser Arbeit die �bereinstimmung beider Daten f�r jeden Pixel notwendig. Die Klasse \textbf{PMDPoint} wird nun definiert, um diesen Zweck zu erf�llen. Die Klassendiagramme beider Klassen werden in der Abbildung~\ref{BD} gezeigt.

\begin{figure}[ftb]
\centering
\includegraphics[scale=1]{Abbildungen/BildData.png}
\caption{Die Klassendiagramme f�r \textbf{BildData} und \textbf{PMDPoint}.}
\label{BD}
\end{figure}

\subsubsection{Abstand Filter}
Die Verbesserung der Ergebnisse der Segmentierung des fokussierten Objekts aus der Umgebung wurde im Abschnitt \ref{sbSeg} bemerkt. Deshalb soll am Anfang der Markenanalyse ein Abstand-Filter definiert werden. Der Abstand-Filter dieser Arbeit besteht aus zwei Teilen: der Teil der Initialisierung bzw. der Teil der Filterung. Zwischen der Initialisierung des Filters werden eine Menge der Tiefdaten der Eingabebilder mit vordefinierter Gr��e angesammelt, damit das durchschnittliche Tiefenbild der Szene erzeugt werden kann. Der Pseudocode wird in Algorithmus~\ref{AFIni} gezeigt.

\begin{algorithm}
\caption{Initialisierung des Abstand-Filters}
\label{AFIni}
\begin{algorithmic}
    \State Anzahl der notwendigen Eingabebilder: $N$
    \State Tiefbild f�r die durchschnittliche Szene: $D$
    \For {$i$ von $1$ bis $N$}
    	\State $E \gets$ aktuelle \textbf{BildData}
        \If {$D$ == NULL}
            \State $D \gets E.3DKoordinaten.Z$ 
        \Else
        	\For {jedes Pixel $d_j$ von $D$}
        		\State $d_j \gets \frac{1}{2}(d_j + e.3DKoordinaten_{j_z})$
        	\EndFor
        \EndIf
    \EndFor
\end{algorithmic}
\end{algorithm}

Durch Vergleich der in Initialisierungsphase erzeugten Szene und die aktuelle Tiefdaten der 3D-Eingabe, k�nnen die neu in der Szene eingef�gten Objekte aus dem Hintergrund bestimmt werden. Wenn der Abstand zwischen einem Pixel und dem Hintergrund gr��er als der vordefinierte Schwellwert ist, wird die Amplitude des gefilterten Bildes in diesem Pixel von originaler Amplitude �bertragen. Sonst wird die Amplitude als null ersetzt. Die Anzahl der unterschiedlichen Bildpunkte wird gleichzeitig abgez�hlt, damit eine boolesche Ausgabe �ber die Verschiedenheit mit dem gefilterten Bild zusammen zur�ckgegeben werden kann. Der genaue Ablauf wird in Algorithmus~\ref{AFFilter} beschrieben. 

\begin{algorithm}
\caption{Filterungsphase des Abstand-Filters}
\label{AFFilter}
\begin{algorithmic}
	\State Schwellwert f�r Abstandsvergleich: $\epsilon$
	\State Schwellwert f�r Verschiedenheit: $\alpha$	
	\State $E \gets$ aktuelle \textbf{BildData}
	\For {jedes Pixel $d_i$ von $D$}
		\If {$\|d_i - e.3DKoordinaten_{i_z} \| < \epsilon$}
			\State $e.filteredAmplitude_i \gets 0$
		\Else
			\State $e.filteredAmplitude_i \gets e.Amplitude_i$
		\EndIf
	\EndFor
	\State $q \gets \| \text{ver�nderten Pixeln} \| / \|E\|$
	\If {$q < \alpha$}
		\State \Return Falsch
	\Else
		\State \Return True, $E$
	\EndIf
\end{algorithmic} 
\end{algorithm}

\subsection{Markenerkennung}

\subsubsection{Auswahl der Gr��e der Marken}
Die Erkennungsergebnisse der Objekte werden von der Gr��e der Marken stark beeinflusst, weshalb ein Test �ber die Markengr��e vor der Implementierung notwendig ist. Die Testmarken werden als wei�e Punkte auf einem schwarzen Papier gedruckt. Es gibt insgesamt sieben verschiedene Gr��enstufen. Die Marken von jeder Stufe werden jeweils durch Quadrate und Kreise dargestellt. Abbildung~\ref{MS1} zeigt die unterschiedliche Erkennungsergebnisse der Marken, wenn das Objekt zwar an verschiedenen Positionen aber auf der gleichen H�henebene liegt. Abbildung~\ref{MS2} zeigt den Einfluss auf die Ergebnisse von der Distanz zu der Kamera. Hierbei wird deutlich, dass, je n�her das Objekt zu der Kamera angebracht wird, desto kleinere Markengr��en ben�tigt werden. (folgt aus dem Kapitel \ref{AueM})

\begin{figure}[htbp]
\centering
\includegraphics[scale=0.5]{Abbildungen/MarkerSize1.png}
\caption{Die Erkennungsergebnisse der Marken f�r unterschiedliche Positionen.}
\label{MS1}
\end{figure}

\begin{figure}[htbp]
\centering
\includegraphics[scale=0.75]{Abbildungen/MarkerSize2.png}
\caption{Die Erkennungsergebnisse der Marken f�r unterschiedliche Distanzen zur Kamera. Zu der ersten Reihe liegen die Graustufenbilder mit roten erkannten Marken; die entsprechenden 3D-Bilder in der zweiten Reihe zeigen die vertikalen Positionen des Objekts.}
\label{MS2}
\end{figure}

\subsubsection{STAR Detektor}
Wegen der in Abschnitt \ref{AusAlgo} festgestellten Gr�nde wird der STAR Detektor (CenSurE Algorithmus) in dieser Arbeit als Erkennungsalgorithmus ausgew�hlt. Die Parameter des Detektors werden im Dokument von OpenCV aufgelistet \cite{SDO} und von ,,ButterCookies'' in OpenCV Adventure \cite{SDOA} weiter deutlich erkl�rt. Der Parameter {\itshape maxSize} definiert die maximale Gr��e der Marken. {\itshape responseThreshold} ist ein Schwellwert �ber die Antwort von approximierten Laplacian. Die erkannten Marken mit niedrigerer Antwort als diesen Schwellwert werden dann eliminiert. Die Parameter {\itshape lineThresholdProjected} und {\itshape lineThresholdBinarized} stellen die St�rke der Linie-Suppression ein. Der letzte Parameter {\itshape suppressNonmaxSize} beschreibt die Gr��e des betrachteten Raums der Non-Maximal Suppression. In dieser Arbeit wird ResponseThreshold als der variable Parameter benutzt, d.h., dass dessen Wert in jedem Schritt ver�ndert wird.

\subsubsection{Markenerkennung} 
Algorithmus~\ref{ME} zeigt den Verlauf der Markenerkennung mit STAR Detektor. Der ganze Erkennungsprozess wird als eine Schleife mit vordefinierter, maximaler Anzahl der Durchl�ufe dargestellt. In jedem Schleifenrumpf wird das Graustufenbild als Eingabe des STAR Detektors mit dem aktuellen Kontrast vor der Erkennung erzeugt. Nach der Erkennung schickt das Programm dann das aktuelle Graustufenbild und die Anzahl der erkannten Marken zu der Funktion \textbf{Kontraststeuerung} (siehe Kap.\ref{SHuK} und Algorithmus~\ref{KS}). Die Kontraststeuerung �berpr�ft, ob zufriedenstellend so viel Marken erkannt werden. Eine Vorhersage �ber Kontrast bzw. ResponseThreshold f�r die n�chste Schleife wird gleichzeitig erzeugt, wenn es zu wenig oder zu viel Marken gefunden werden.

\begin{algorithm}[htbp]
\caption{Markenerkennung}
\label{ME}
\begin{algorithmic}
	\State Eingabebild: $I$, maximale Schleife: $N$
	\State $a \gets$ Anfangswert f�r Kontrast 
	\State $r \gets$ Anfangswert f�r ResponseThreshold
	\State $b \gets$ feste Helligkeit  
	\For {$i = 1 \to N$}
		\State Eingabe f�r STAR Detektor: $H_i \gets aI+b$
		\State Erkannte Marken: $P \gets CenSurE(H_i,r)$
		\State Result $\gets$ Kontraststeuerung($H_i, a, r, \|P\|$)
		\If {Result = False}
			\State continue mit aktualisiertem Kontrast und ResponseThreshold
		\Else
			\State break
		\EndIf
	\EndFor
	\Return $P$
\end{algorithmic}
\end{algorithm}

Die Punktmenge, die durch STAR-Detektor erkannt wird, hat leider keine 1 zu 1 Korrespondenz zu den realen Marken, die auf der Oberfl�che des Objektes angebracht werden. Daraus bedeutet, dass eine Marke des Objektes von dem Detektor h�ufig als viele verschiedene Merkmale erkannt wird. Deshalb ist eine Kombination der Merkmale n�tig, die zur gleichen Marke geh�ren. Dieser Prozess kann in zwei Schritten zusammengefasst werden. Zuerst werden alle Merkmale durch den Algorithmus DBSCAN (siehe Kapitel~\ref{Dbscan}) in viele Gruppen verteilt. Daraufhin wird  f�r jede Gruppe  der zentrale Punkt berechnet, der als einziges Merkmal eine reale Marke beschreibt.

\subsubsection{Steuerung der Helligkeit und des Kontrast}
\label{SHuK}
Die Helligkeit bzw. der Kontrast kann die Ergebnisse der Markenerkennung stark beeinflussen. Eine bekannte Arbeit �ber die radiometrische Kalibrierung wurde von Debevec und Malik in \cite{DM08} ver�ffentlicht. Ihr Verfahren ist robust und pr�zise, ben�tigt aber viele Bilder der gleichen Szene mit verschiedenen Belichtungszeiten, um alle Informationen der Szene f�r einen jeweils helleren Bereich bzw. dunkleren Bereich zu sammeln. Wegen der Einschr�nkung der Kamera kann diese Bedingung in unserer Arbeit leider nicht erf�llt werden. Au�erdem ist das Verfahren von Debevec sehr rechenintensiv. nur die Bilddaten um Marken sollen aber in der Markenerkennung dieser Arbeit betrachtet werden, wodurch die sonstigen Informationen des Bildes  einfach ignoriert werden k�nnen. Durch diese Grundidee kann ein Algorithmus erzeugt werden, dessen Endbedingung durch die Untersuchung der Anzahl der erkannten Marken eingestellt wird.
\\
\\
Der formulierte Durchlauf wird im Algorithmus~\ref{KS} dargestellt. Die Anzahl der erkannten Marken ist die erste Stufe der Pr�fnorm der Kontraststeuerung. Wenn nicht genug Marken erkannt werden, dann wird die Strahlungsenergie des Graustufenbildes als die zweite Stufe der Pr�fnorm �berpr�ft. Bei einer zu gro�en bzw. zu kleinen Strahlungsenergie verringert bzw. erh�ht sich der Kontrast  im erlaubten Intervall, welches  als Eingangsparameter vorher definiert wird. Wenn die Strahlungsenergie zufriedenstellend ist, aber noch nicht genug Marken gefunden sind, nimmt das ResponseThreshold ab, damit mehr Marken mit schw�cheren Antworten erkannt werden k�nnen. Falls zu viele Marken erkannt werden, wird der Wert von ResponseThreshold vergr��ert. Bei dieser Situation wird die Strahlungsenergie nicht mehr betrachtet, damit der Algorithmus zur Konvergenz halten kann. Ein boolescher Wert wird von dem Algorithmus zur�ckgegeben, der zeigt, ob befriedigend viele Marken gefunden werden.

\begin{algorithm}
\caption{Kontraststeuerung($H, \&a, \&r, \|P\|$)}
\label{KS}
\begin{algorithmic}
	\State Intervall der erlaubten Strahlungsenergie: $(E_{min}, E_{max})$
	\State Intervall des erlaubten Kontrastes: $(A_{min}, A_{max})$
	\State Intervall des erlaubten ResponseThreshold von STAR Detektor: $(R_{min}, R_{max})$
	\State Intervall des Erwartens der Anzahl des bekannten Marken: $(P_{min}, P_{max})$
	\State $e \gets$ aktuelle Strahlungsenergie von $H$
	\If {$\|P\| < P_{min}$}
		\If {$e < E_{min}$}
			\State $a \gets a-0.5$
			\If {$a < A_{min}$}
				\State $a \gets A_{min}$
				\State \Return False
			\EndIf
		\ElsIf {$e > E_{max}$}
			\State $a \gets a+0.5$
			\If {$a > A_{max}$}
				\State $a \gets A_{max}$
				\State \Return False
			\EndIf
		\Else 
			\State $r \gets r-5$
			\If {$r < R_{min}$}
				\State \Return False
			\EndIf
		\EndIf
	\ElsIf {$\|P\| > P_{max}$}
		\If {$r > R_{max}$}
			\State \Return False
		\Else
			\State $r \gets r+4$
		\EndIf
	\Else
		\State \Return True
	\EndIf
\end{algorithmic}
\end{algorithm}



\subsection{Verbesserung der Singul�rwertzerlegungsverfahren}
\label{MV}
\label{VdS}
Nach erfolgreicher Festlegung der Marken jedes Bildes sollen dann die Korrespondenzen der Marken von zwei nachfolgenden Bildern untersucht werden. Die Korrespondenzpunkte k�nnen durch das Verfahren der Singul�rwertzerlegung von Scott und Higgnis \cite{SL91} bestimmt werden. Die genaue Beschreibung ihres Verfahrens wird im Schnitt~\ref{KdS} aufgef�hrt, und Algorithmus~\ref{alg1} gibt den Peseudocode des Verfahrens an. In dieser Arbeit werden drei Verbesserungen f�r das Verfahren von Scott und Higgnis durchgef�hrt, damit das stabilere Ergebnis erzeugt werden kann. Die Qualit�t der Korrespondenzuntersuchung kann auch nach der Verbesserung bewertet werden. 
\\
\\
\textbf{Nebenbedingung f�r die Bestimmung der gr��ten Elemente}
\\
In dem originalen Algorithmus werden die Punkte $I_i$ und $J_j$ genau dann als Korrespondenzpunkte erkannt, wenn das Element $P_{ij}$ das gr��te Element von beiden Zeile $i$ und Spalte $j$ ist. Aber manchmal liefert das gr��te Element nicht die beste Korrespondenz, insbesondere in dem Fall einer  gro�en Transformation zwischen zwei betrachteten Bildern. Die L�sung besteht darin, eine weitere Beschr�nkung f�r die Untersuchung der gr��ten Elemente einzusetzen. D.h., dass die gr��ten Elemente nur dann akzeptiert werden, wenn sie gleichzeitig gr��er als ein eingegebener Schwellwert $\epsilon$ sind. Der Schwellwert $\epsilon$ wird in $[0,1]$ definiert. Je h�her $\epsilon$ ist, desto ordentlicher sind die gefundenen Korrespondenzpunkte. In Idealfall sind alle gr��ten Elemente $P_{ij}$ gleich 1, z.B. wenn die beiden betrachteten Bilder identisch sind. Der Nachteil der Verwendung des Schwellwerts liegt darin, dass manchmal keine Korrespondenz zwischen zwei Bilder gefunden kann. Deshalb ist eine boolesche Ausgabe des Algorithmus notwendig, welches sich als  eine wichtige Variable der Bildsteuerung darstellt (siehe Kapitel~\ref{BildS}). Das Programm kann durch diese Variable erkennen, ob die Korrespondenzuntersuchung erfolgreich durchgef�hrt wird.
\\
%\\
%\textbf{Boolesche Ausgabe}
%\\
%Weniger Korrespondenzpunkte werden mit obiger st�rkeren Nebenbedingung herausgefunden, was aber die weiteren Arbeitsschritte wenig beeinflusst, weil es zumindest nur 3 korrespondierenden Punktpaare ben�tigt, um die Orientierung zwischen zwei Bildern zu bestimmen. Trotzdem ist die �berpr�fung der Anzahl der gefundenen Punktpaare notwendig, und deren Ergebnis wird als eine boolesche Ausgabe des Algorithmus zur�ckgegeben.
%\\
\\
\textbf{Qualit�ts�berpr�fung der Korrespondenz}
\\
Neben der bin�ren Behauptung des Algorithmus ist auch die Bewertung der Korrespondenzqualit�t  wichtig, damit die gute und stabile Bilderkette f�r Markenverfolgung ausgew�hlt werden kann. Das wurde aber leider in der Arbeit von Scott und Higgnis nicht genannt. In der ersten Verbesserung wird die Gr��e der gr��ten Elemente von Matrix $P$ mit einem Schwellwert weiter beschr�nkt, um ein besseres Korrespondenzergebnis zu erhalten. Deshalb kann zur Umkehr die Gr��e der gr��ten Elemente von $P$ als die Messung der Korrespondenzqualit�t definiert werden. Dann wird die Summe aller gr��ten Elemente $\sum P_{ij}$ in dieser Arbeit zur Bewertung der Korrespondenzuntersuchung verwendet. Hier wird kein arithmetisches Mittel benutzt, weil die Anzahl der betrachteten Punkte, die als Eingaben in den Algorithmus eingegeben werden, auch ein wichtiger Faktor der Bewertung der Korrespondenz sind.
\\
\\
\textbf{Die Auswahl von $\sigma$ mit R�ckkopplung}
\\
Die Einheit des Abstandes $\sigma$ ist der wesentliche Parameter des Singul�rwertzerlegungsverfahrens. Das ungeeignete $\sigma$ erzeugt dann gro�es Chaos in den Korrespondenzl�sungen, was schon in \cite{SL91} mit Schaubildern verdeutlicht wird. Um die besten L�sungen zu finden, soll die Abstandseinheit so definiert werden, dass sie nicht kleiner als die durchschnittliche Distanz zwischen den Korrespondenzpunkten ist. Es gibt zwei Schwierigkeiten f�r die Bestimmung des $\sigma$. Erstens sind die korrespondierenden Punktpaare vor dem Durchlauf des Algorithmus noch nicht bekannt, weshalb die Abstandseinheit nicht direkt berechnet, sondern nur gesch�tzt werden kann. Zweitens ist die Auswahl einer festen Abstandseinheit schwierig und ineffizient, wenn sich die betrachteten Punkte oder Merkmale, wie z.B. in dieser Arbeit, uneingeschr�nkt bewegen k�nnen. Deshalb wird hier der Parameter $\sigma$ f�r jedes Bild mit der durchschnittlichen Distanz zwischen allen korrespondierenden Punktepaaren des vorherigen Bildes festgelegt. Wegen der festen Bildwiederholfrequenz der Eingabebilder ist im theoretischen Fall der durchschnittliche Abstand der Korrespondenzpunkte jedes Bildes gleich, wenn sich alle betrachteten Punkte gleichf�rmig bewegen. Aber f�r die schwach beschleunigenden Bewegungen der Merkmale kann das sich anpassende $\sigma$ trotzdem herausgefunden werden, und die Korrespondenzl�sungen stabil erzeugen lassen.
\\
\\
Der verbesserte Algorithmus wird im Algorithmus~\ref{algSVD} gezeigt.

\begin{algorithm}                     
\caption{Korrespondenzuntersuchung($Punktpaare$ $Result$)}         
\label{algSVD}                          
\begin{algorithmic}
	%\State Korrespondierende Punktpaare: $Result$                    
    \State Eingabebilder von jeweils Zeitpunkt $t_{i-1}$ und $t_{i}$: $I,J$
    \State Aktuelle approximierte Abstandseinheit: $\sigma_{i}$
	\State Schwellwert f�r die Beschr�nkung der gr��ten Elemente: $\epsilon$
	\State Messung der Korrespondenzqualit�t: $mess$ 
	\State Summe der Abst�nde der Korrespondenzpunkte: $sumDistance$
    \For{$i=1 \to m$, $j=1 \to n$}
    	\State $r_{ij} \gets Dis(I_i, J_j)$
    	\State $G_{ij} \gets exp(-\frac{r_{ij}^2}{\sigma_i^2})$
    \EndFor
    \State $T,U \gets$ Singul�rwertzerlegung von G
    \State $E \gets m \times n$ Diagonalmatrix mit $E_{ii} = 1$
    \State $P \gets TEU$
    \State $minMN \gets Min(m, n)$
    \For{$i=1 \to minMN$}
    	\State $MaxSpalteIndex[i] \gets$ Index der Spalte des maximalen Elements an Reihe $i$.
    	%\State $MaxSpalteIndex_i \gets Max$ 
    \EndFor
    \For{$i=1 \to minMN$}
    	\If {$P_{iMaxSpalteIndex[i]}$ ist Maximum der Spalte $MaxSpalteIndex[i]$}
    		\If {$P_{iMaxSpalteIndex[i]} > \epsilon$}
    			\State $Result \gets$ Punktpaar($I_i, J_{MaxSpalteIndex[i]}$)
    			\State $mess \gets mess + P_{iMaxSpalteIndex[i]}$
    		\EndIf 
    	\EndIf
    \EndFor 
    \For {Alle Punktpaare $(I_i, J_j) \in Result$}
    	\State $sumDistance \gets sumDistance + DistanceOf(I_i,J_j)$
    \EndFor
    \State $\sigma_{i+1} \gets sumDistance / \| Result \|$
    \If {$\| Result \| < 3$ }
    	\State \Return False
    \Else
    	\State \Return True, $mess$
    \EndIf
\end{algorithmic}
\end{algorithm}

\subsection{Segmentierung}
\label{Seg}
Bis jetzt werden alle erkannten Marken eines Bildes als eine Menge der Punkte zusammen betrachtet, die dann segmentiert werden soll, um die Marken auf verschiedene Objekte zu verteilen. Die Grundidee ist, dass die Marken genau dann als die Merkmale eines gleichen Objektes erkannt werden,  wenn die Abst�nde zwischen diesen Marken viel kleiner sind als die Abst�nde von ihr zu den anderen erkannten Marken. Die Problemstellung der Segmentierung ist gleich wie ein Clusting-Problem, weshalb in dieser Arbeit der Clusting-Algorithmus DBSCAN f�r die Segmentierung benutzt wird. Die Beschreibung des Verfahrens wird im Abschnitt \ref{Dbscan} gezeigt und der genaue Durchlauf wird durch den Algorithmus~\ref{algDBSCAN} erkl�rt.	Eine Liste der Punktmengen wird von DBSCAN ausgegeben, und jedes der Elemente stimmt mit der Merkmalen eines Objektes �berein. Da es zumindest drei  Punkte ben�tigt, um die Lage des r�umlichen K�rpers im dreidimensionalen Raum zu bestimmen, werden die Punktmengen von der Ausgabe der DBSCAN als Rauschen erkannt, die weniger als drei Punkte enthalten.

\section{Objektlernen}
\label{OL}
Nach der erfolgreichen Markenanalyse wird eine Menge der Marken aus den Eingabebildern herausgefunden. Weiterhin sind die Abh�ngigkeiten dieser Marken zwischen benachbarten Bildern auch bekannt. Durch die Segmentierung werden diese Marken danach f�r unterschiedliche Objekte weiter verteilt. Im diesen Abschnitt wird erkl�rt, wie ein charakteristischer Graph des Objektes mithilfe der erkannten Marken dargestellt werden kann. In dieser Arbeit wird im Objektlernen die vereinfachte Situation ber�cksichtigt, dass ein Objekt nur einmal in den Eingabebildern vorkommt. Deshalb wird nur die gr��te Punktmenge des Segmentierungsergebnisses f�r das Objektlernen ausgew�hlt, die als die Eingabe f�r die folgenden Arbeitsschritte genutzt wird. 

\subsection{Bestimmung der Orientierung}
\label{BdO}
Um ein Objekt zu lernen, muss dieses zuerst  mit Marken zu der Kamera gezeigt und langsam umdreht werden. Zwischen der Umdrehung werden dann die relativen Positionen der Marken an jeder Ebene bestimmt. Deshalb sollen die Lagen der bekannten Marken in diesem Ablauf rechtzeitig aktualisiert werden, was als ein Orientierungsproblem formuliert werden kann. In dieser Arbeit wird das Verfahren von \cite{H87} verwendet, um die Transformation der bekannten Marken zwischen nachfolgenden Bildern zu berechnen. Der genaue Durchlauf wird im Abschnitt \ref{QmE} aufgef�hrt. Zuerst sollen die Schwerpunkte der erkannten Marken herausgefunden werden (siehe Formel~\eqref{QuaSch}). Dann berechnet man die Vektoren, die von allen Marken zu ihren entsprechenden Schwerpunkten reichen. Als Drittes wird die Korrelationsmatrix $H$ durch Formel~\eqref{QuaH} mit diesen Vektoren bestimmt. Die Hilfsmatrix $N$ in der Formel~\eqref{QuaN} besteht aus den Elementen der Matrix $H$. Einer ihrer Eigenvektoren ist genau dann die Einheitsquaternion f�r die Rotation,  wenn dieser dem gr��ten positiven Eigenwert entspricht. Die Rotations- bzw. Translationsmatrix unter kartesischem Koordinatensystem wird jeweils durch die Formel~\eqref{QuaR} und \eqref{QTran} berechnet.
\\
\\
Die Lage des Objekts in jedem Bild h�ngt nur von der Lage des Objekts in dem vorherigen Bild ab. Die kleinen Fehler der Orientierung zwischen den nachfolgenden Bildern werden in einer langen Bildsequenz akkumuliert, das h�ufig einen gro�en Unterschied zwischen der realen Lage und berechneten Lage des Objekts erzeugt. In dieser Arbeit wird der Kalman-Filter �ber die Translation des Schwerpunktes des Objekts benutzt, um die kumulative negative Wirkung jedes Schrittes zu vermindern. Die Grundlage des Kalman-Filters wird im Abschnitt~\ref{KF} aufgef�hrt. Die �bergangsmatrix des Schwerpunkts des Objekts im dreidimensionalen Raum kann definiert als:

\[
F = 
\begin{pmatrix}
1 & 0 & 0 & 1 & 0 & 0 \\
0 & 1 & 0 & 0 & 1 & 0 \\
0 & 0 & 1 & 0 & 0 & 1 \\
0 & 0 & 0 & 1 & 0 & 0 \\
0 & 0 & 0 & 0 & 1 & 0 \\
0 & 0 & 0 & 0 & 0 & 1
\end{pmatrix},
\] 

was die Lage und Geschwindigkeit der Bewegung gleichzeitig betrachtet. Die Beobachtungsmatrix ist die Einheitsmatrix mit der Gr��e 3$\times$3, weil der Schwerpunkt des Objekts $\in R^{3\times 1}$ als Eingabe zur Korrekturphase eingegeben wird. OpenCV liefert eine komplette Implementierung des Kalman-Filters, was hier direkt verwendet wird.

\subsection{Markenanordnung}
\label{Ma}
Wie im Anfang des Abschnittes~\ref{BdO} beschrieben, soll w�hrend der Lernphase die Lage des betrachteten Objektes in jedem Bild bestimmt werden. Hierzu  ben�tigt man mindestens in zwei Bildern jeweils drei nicht kollineare Punkte des Objektes, damit die Transformation zwischen der Lage dieses Objektes in diesen zwei Bildern bestimmt werden kann. Je mehr die Punkte betrachtet werden, desto exaktere Transformationen k�nnen in der Theorie berechnet werden. Wegen der Aufl�sung der Kamera in dieser Arbeit werden aber die Gr��e der Marken bzw. die Abst�nde der Marken stark beschr�nkt. D.h. es ist unm�glich, beliebig viele Marken an dem Objekt anzubringen. Deshalb soll eine Strategie f�r die Anordnung der Marken entworfen werden, damit die beste Transformation mit m�glichst wenigen Marken bestimmt werden kann.
\\
\\
Die Anordnungsstrategie wird in 5 Regeln zusammengefasst.
\begin{enumerate}
	\item Die Gr��en der Marken d�rfen nicht zu klein sein.
	\item Die Abst�nde zweier Marken sollen nicht zu eng sein.
	\item Jedes Tripel der Marken soll nicht kollinear sein.
	\item Um die Grenzen zweier Ebenen des Objekts sollen mehr Marken als in Mitte angebracht werden.
	\item Der Strukturgraph jeder Ebene soll m�glichst unsymmetrisch sein.
\end{enumerate}

Die erste und zweite Regel h�ngen direkt von der Aufl�sung der Kamera ab, was schon im Abschnitt~\ref{AueM} diskutiert wurde. Die dritte Regel garantiert, dass die Orientierung des Objekts immer durch drei beliebige Marken berechnet werden kann. Die in der ersten Regel  definierte minimale erkennbare Markengr��e entspricht aber nur der Situation, in der  die Marken mit der Bildebene der Kamera parallel sind. Wenn sich ein Objekt umdreht, ver�ndert sich der Winkel zwischen der aktuell beobachteten Ebene und der Bildebene. Deswegen werden die gleichen Marken aber mit weniger Pixel in der Kamera abgebildet. Die Gr��en der Marken auf dieser beobachteten Ebene erscheinen immer kleiner in dem Bild mit der Vergr��erung des Drehwinkels, bis sie komplett senkrecht zu der Bildebene liegen und die n�chste Ebene vollst�ndig unter der Kamera vorkommt. Je kleiner die Marken sind, desto schwieriger werden sie erkannt. Deshalb sollen mehr Marken im Bereich zwischen zwei benachbarten Ebenen angebracht werden, damit das Programm genug Marken f�r die Berechnung der Orientierung erhalten kann. Das ist genau was in der vierten Regel  erkl�rt wird. Die Asymmetrie von der f�nften Regel   h�lt die Einzigartigkeit der Transformation zwischen zwei unterschiedlichen Lagen des Objekts, d.h. nur eine L�sung wird von dem Orientierungsverfahren bestimmt.


\subsection{Darstellung des Strukturgraphen}
\label{DdS}
Das Ziel des Lernens ist, dass ein Strukturgraph f�r jedes eingegebenes Objekt erzeugt wird, welcher als die Charakteristiken f�r die Wiedererkennung verwendet werden kann. Um das Ziel zu erreichen, sollen die Strukturgraphen stabil sein und genug charakteristische Information des Objektes enthalten. Die Stabilit�t bedeutet, dass die Strukturgraphen f�r ein Objekt bei �fterem Lernen gleich dargestellt werden sollen. Die Information, die zur Wiedererkennung notwendig ist, wird durch die Positionen der Marken bzw. die Abst�nde dazwischen erzeugt. Die Verfahren der Bestimmungen der charakteristischen Marken und Kanten werden in folgenden Abschnitten diskutiert.

\subsubsection{Bestimmung der stabilen Knoten}
\label{BdsK}
Nach der Markenerkennung werden viele Marken aus dem Bild erkannt, die aber viel Rauschen enthalten. Das Rauschen wird von den falsch erkannten Merkmalen des Objektes bzw. der Umgebung oder den schlechten Korrespondenzpunkten verursacht. Deshalb ben�tigt das Programm eine Strategie f�r die Auswahl der g�ltigen Marken. In dieser Arbeit basiert die Auswahl auf Erscheinungsh�ufigkeiten, d.h. nur die Marken, die mehrmals kontinuierlich vorgekommen sind, werden zu den g�ltigen Marken gerechnet und in den Strukturgraphen eingef�gt. Diese Marken werden als stabile Knoten in dem Strukturgraphen markiert, und d�rfen nicht ver�ndert werden. Abgesehen von der Beschr�nkung der Anzahl der Erscheinungen soll aber auch der Begriff von ,,Identit�t'' zweier Marken in unterschiedlichen Bildern definiert werden. Zwei Marken von verschiedenen Bildern sind genau dann identisch,  wenn der Abstand von einer Marke zu dem Punkt, der von anderer Marke nach Transformation erzeugt wird, kleiner als ein vorgegebener Schwellwert ist. Die Transformation umfasst die Rotation- bzw. Translationsmatrix, die mit dem Verfahren von Abschnitt~\ref{BdO} bestimmt werden.

\subsubsection{Kanteneinf�gung}
Die ungerichteten Kanten des Strukturgraphen speichern die Abst�nde zwischen den Knoten des Graphen, welches die wichtigste Eigenschaft f�r die Wiedererkennung ist. Die Knoten, die gleichzeitig beobachtet werden k�nnen, werden in einem vollst�ndigen Graphen mit den anderen verbunden. Wenn irgendwann ein neuer Knoten in den Graphen eingef�gt wird, verbindet er mit allen vorhandenen Knoten des Graphen. Der analoge Durchlauf wird w�hrend der Entfernung eines ung�ltigen Knotens aus dem Graphen durchgef�hrt. Der Algorithmus~\ref{algAG} zeigt, wie der aktuelle Strukturgraph mit neu eingegebenen Punkten aktualisiert wird.

\begin{algorithm}                     
\caption{GraphUpdate($Punkte$ $P$, $Mat$ $R$, $Mat$ $T$)}         
\label{algAG}                          
\begin{algorithmic}
\State Schwellwert des Abstand: $\epsilon$
\State Minimale Lebenszeit des stabilen Knoten: $minT$
\State Aktueller Graph: $G$
\State $G_{temp} \gets createCompleteGraph(P)$
\For {jeder Knoten $v_i \in G$}
	\State $v_i \gets R v_i + T$
\EndFor
\For {jeder Knoten $v_i \in G$}
	\For {jeder Knoten $v_{temp_j} \in G_{temp}$}
		\If {$DistanceOf(v_i, v_{temp_j}) < \epsilon$}
			\State $v_i.Lifetime \gets v_i.Lifetime+2$
			\If {$v_i.Lifetime > minT$}
				\State $v_i.isFixed \gets True$
			\EndIf
			\State Verbinden aller mit $v_{temp_j}$ verbundenen Knoten in $G_{temp}$ mit $v_i$
			\State Entfernen $v_{temp_j}$ aus $G_{temp}$
		\EndIf
	\EndFor
	\If {Kein entsprechender Knoten f�r $v_i$ aus $G_{temp}$ gefunden wird}
		\State $v_i.Lifetime \gets v_i.Lifetime-1$
		\If {$v_i.Lifetime<0$ und $!v_i.isFixed$}
			\State Entfernen $v_i$ aus $G$
		\EndIf
	\EndIf
\EndFor
\State Einf�gen aller �brigen Knoten von $G_{temp}$ in $G$
\end{algorithmic}
\end{algorithm}

\section{Zugriff des Strukturgraphen von Datei}
Das Ergebnis des Objektlernens ist ein Strukturgraph, der die Charakteristiken des betrachteten Objektes beschreibt. Die Knoten des Graphen werden als dreidimensionale Punkte mit Gleitkommazahl definiert. Jeder Knoten enth�lt eine Map, in der seine Nachbarn und die Kantenl�nge dazwischen paarweise gespeichert werden. Die Strukturgraphen sollen auf der Festplatte gespeichert und eingelesen werden k�nnen. In dieser Arbeit wird die Dateiform von VTK (\cite{VTK}) benutzt, welche als eine Open-Source-C++-Klassenbibliothek f�r die 3D-Computergraphik und wissenschaftliche Visualisierung h�ufig verwendet wird. Der Vorteil der Verwendung der VTK Datei ist, dass die Ergebnisgraphen einfach weiter von dem anderen Framework bzw. der Software verwendbar sind. Die Abbildung~\ref{PV} ist der Screenshot von der Visualisierung eines Strukturgraphen mit der Software ParaView \cite{PV}. Vor dem Speichern des Graphen werden alle Knoten noch einmal �berpr�ft. Die �hnlichen Knoten sollen in einem einzigen Knoten kombiniert werden, damit nur ein vereinfachter Graph ohne verdoppelten Knoten gespeichert wird. Die VTK-Dateiform liefert viele Stichw�rter f�r Beschreibung der unterschiedlichen grunds�tzlichen, geometrischen Elemente, wie z.B. die Punkte, die Gerade und die Ebene, usw. Deshalb werden die Knoten und die Kanten des Strukturgraphen jeweils mit dem Stichwort ,,POINTS '' und ,,LINES'' in der VTK-Datei geschrieben. Diese Stichw�rter sind auch die Kennzeichen beim Lesen der VTK-Datei. Die Punkte werden zuerst in einen neuen Graphen eingef�gt, und dann mit anderen Knoten durch die gespeicherten Gerade verbunden.

\begin{figure}
\centering
\includegraphics[scale=0.3]{Abbildungen/ParaView.png}
\caption{Visualisierung eines Strukturgraphen eines K�stchens mit ParaView. Das Foto des K�stchens ist in Abbildung~\ref{BOXES} dargestellt.}
\label{PV}
\end{figure}

\section{Objekterkennung und Verfolgung}
Die Strukturgraphen aller vom Programm erlernten Objekte sollen am Anfang der Wiedererkennungsphase von den auf der Festplatte gespeicherten VTK-Dateien wieder in das Programm eingelesen werden. Danach wird die Markenanalyse �ber die Eingabebilder genau wie in der Lernphase durchgef�hrt, und am Ende wird eine Liste von der Punktmenge erzeugt. Das Programm versucht dann, die Teilgraphen dieser Strukturgraphen aus den von den Punktmengen neu dargestellten Graphen zu finden, was im sogenannten Teilgraph-Isomorphismus-Problem zusammengefasst wird. F�r ein Bild werden alle m�glichen Kombinationen der Strukturgraphen und Punktmengen dieses Bildes durchgesucht. Die Objekte, deren Teilgraphen aus dem Eingabebild gefunden wurden, werden als erkannt angenommen. Die entsprechenden Punktmengen im Eingabebild beschreiben nun die aktuellen Lagen der Objekte. Mit dem gleichen Ablauf k�nnen die Objekte in einem Bilderstrom kontinuierlich erkannt werden, was aber nicht m�glich ist, ist die kontinuierliche Festlegung der entsprechenden Punktmengen, weil die Marken von jedem Bild neu segmentiert werden und keine Verbindungen zwischen den ,,gleichen'' Punktmengen von unterschiedlichen Bildern existieren. Ein direktes L�sungsverfahren besteht darin, dass jede Punktmenge aus der Segmentierung sofort als ein neues Objekt erkannt wird. Der Vergleich des Teilgraph-Isomorphismus-Problems findet dann zwischen den Strukturgraphen zweier Objekte statt. Dadurch sollen viele Lernprozesse w�hrend der Wiedererkennungsphase gleichzeitig durchgef�hrt werden. Das kostet aber zu viel Zeit, und ist schwierig zu implementieren. Die alternative L�sung liegt in der Darstellung der sogenannten Kandidaten. Diese Kandidaten interessieren nur die Abh�ngigkeiten der gleichen Punktmengen von unterschiedlichen Bildern, enthalten aber keine stabilen Knoten wie den Strukturgraphen. Die Abbildung~\ref{erAblauf} zeigt ein Beispiel �ber die gesamte Erkennungsphase.

\subsection{Kandidaten der Objekterkennung}
Wie im Abschnitt~\ref{Seg} beschrieben, werden viele Punktmengen nach der Segmentierung erzeugt, welche mit unterschiedlichen Objekten �bereinstimmen k�nnen. Die Kandidaten bestehen aus diesen Punktmengen und werden in einer Liste im Speicher gespeichert. Wenn das Programm ein neues Eingabebild einliest, werden alle neuen von der Segmentierung erhaltenen Punktmengen mit vorhandenen Kandidaten verglichen. Der Kandidat, der mit einer der neuen Punktmengen assoziiert, wird dann aktualisiert. Die �brigen Punktmengen, die keine entsprechenden Kandidaten finden k�nnen, werden als neuen Kandidaten in der Liste eingef�gt. Um die Leistungsf�higkeit zu halten, sollen die Kandidaten, die lange Zeit nicht aktualisiert wurden, aus der Liste entfernt werden. Die Erkennung wird dann f�r jeden neu aktualisierten Kandidaten durchgef�hrt. Der Index des am besten entsprechenden Objektes wird im Kandidat gespeichert. In dem Block ,,Behandlung der Kandidaten'' von Abbildung~\ref{erAblauf} wird ein Beispielablauf �ber die Aktualisierung bzw. Darstellung der Kandidaten aufgef�hrt. Die neu erzeugten Kandidaten werden in dem Ende der Kandidatenliste eingef�gt. Der zweite und dritte Kandidat hat in dem Beispiel keine entsprechenden Punktmengen und wird nach der Aktualisierung aus der Liste entfernt. Nach der Behandlung der Kandidaten sollen nur drei Kandidaten im Teilgraph-Isomorphismus mit dem Eingabemodel verglichen werden.

\begin{figure}
\centering
\includegraphics[scale=0.7]{Abbildungen/Kandidaten.png}
\caption{Ausf�hrliches Ablaufdiagramm der Objekterkennung. Die Kandidaten, die schraffiert markiert sind (K2, K3), werden nach der Aktualisierung aus der Liste entfernt.}
\label{erAblauf}
\end{figure}

\subsection{Objekterkennung}
\label{Oerk}
Die Objekterkennung basiert auf dem Vergleich zwischen den Strukturgraphen der erlernten Objekte und den Kandidaten, die von den neu segmentierten Punktmengen erzeugt werden. Ein neuer Graph wird zuerst von den Punkten eines Kandidaten dargestellt. Der Vergleich zwischen diesen Kandidaten und den vorhandenen Strukturgraphen  kann dann zum Teilgraph-Isomorphismus Problem abgleitet werden. Das L�sungsverfahren �bernimmt die Idee von Rhijn und Mulder \cite{AJ05}, die bereits im Abschnitt \ref{TI} erkl�rt wurde. Es werden zuerst die isomorphen Knoten zwischen zwei Graphen herausgefunden, welche jeweils genug isomorphe Nachbarn haben, die in gleichen Abst�nden zu den isomorphen Knoten liegen. Ein Beispiel der isomorphen Knoten wird in Abbildung~\ref{IK} gezeigt. Der Teilgraph-Isomorphismus zwischen diesen Graphen existiert genau dann, wenn mindestens ein gefundener isomorpher Knoten und seine Nachbarn zusammen die gleichen Pyramiden darstellen k�nnen. In dieser Arbeit wird das Verfahren von Rhijn und Mulder in zwei Bereichen verbessert: Die effizientere Nebenbedingung f�r das Suchen der isomorphen Knoten und die vereinfachte Endbedingung anstatt der Feststellung der Pyramide.

\begin{figure}[hftb]
\centering
\includegraphics[scale=0.47]{Abbildungen/IsomorphKnoten.png}
\caption{Ein Beispiel der isomorphen Knoten. $P$ und $P'$ sind zwei isomorphen Knoten, die jeweils vier Nachbarn mit gleicher Abst�nden haben. ($(P,V_1) = (P',V'_1)$, $(P,V_2) = (P',V'_2)$, $(P,V_3) = (P',V'_3)$, $(P,V_4) = (P',V'_4)$)}
\label{IK}
\end{figure}

\textbf{Nebenbedingung f�r das Suchen der isomorphen Knoten}
\\
Um die isomorphen Knoten schneller zu finden, werden zwei Quoten f�r das Suchen definiert. Die Abstandsquote wird anstelle des absoluten Schwellwertes benutzt, damit der Vergleich der Distanzen zwischen dem betrachteten Knoten und seiner Nachbarn bewertet werden kann. Offensichtlich wird die Kante zwischen zwei Marken auf dem Objekt zu mehr Bildpunkten abgebildet, wenn sich das Objekt zur Kamera bewegt. Je gr��er die Aufl�sung des Objektes ist, desto besser ist das Erkennungsergebnis. Hieraus folgt, dass die absolute Kantenl�nge von der Distanz zwischen dem Objekt und der Kamera abh�ngt. Analog dazu liefert der gleiche absolute Schwellwert f�r verschiedene Abst�nde zur Kamera aber unterschiedliche Vergleichsgenauigkeiten. Die Verwendung der Abstandsquote kann die Wirkung der Entfernung der Kamera gut vermeiden.
\\
\\
Die zweite Nachbarquote beschr�nkt die minimale Anzahl der ben�tigen, assoziierten Nachbarn zur Bestimmung eines isomorphen Knotens, was im Verfahren von Rhijn und Mulder aber als Invariante definiert wird. Die Beziehung zwischen der Nachbarquote und der minimalen Anzahl der Nachbarn kann in

\begin{equation}
N_{min} = \left\{
\begin{array}{l l}
3 & \text{falls } \| V \| * Nachbarquote<3 \\
\| V \| * Nachbarquote & sonst \\
\end{array}
\right.
\end{equation}

formuliert werden, wobei $V$ die Knotenmenge des Graphen beschreibt. Die minimale Anzahl der Nachbarn darf nicht kleiner als drei sein, weil es mindesten drei Nachbarn ben�tigt, um einen einzigen Knoten eines Graphen durch den Vergleich der Kantenl�nge seiner Nachbarn im dreidimensionalen Raum zu bestimmen. Diese Ver�nderung liefert eine bessere Toleranz f�r die Erkennung. Wenn die Graphen nur mit weniger Knoten als die Eingaben des Verfahrens angegeben werden, garantiert das Verfahren aber auch genug Nachbarn f�r die Auswahl der isomorphen Knoten. Au�erdem k�nnen die Genauigkeit und die Stabilit�t des Teilgraph-Isomorphismus mit dem Vergr��ern der Anzahl der Knoten gleichzeitig ansteigen.
\\
\\
\textbf{Vereinfachung der Endbedingung}
\\
Im Verfahren von Rhijn und Mulder sollen schlie�lich die Pyramiden f�r den isomorphen Knoten und seine Nachbarn gefunden werden, damit der Isomorphismus bestimmt werden kann. Die Pyramide ist ein vollst�ndiger Graph mit vier Knoten. Deshalb sollen f�r einen Knoten, der $n$ Nachbarn hat, h�chstens 

\[
C_n^3 = \binom{n}{3} = \frac{n!}{k!(n-k)!} = \frac{1}{6} n(n-1)(n-2) \approx \mathcal{O}(\frac{1}{6}(n^3-3n^2))
\]

verschiedenen Kombinationen seiner Nachbarn betrachtet werden, um die Pyramide zu finden. Seien $m$ die Anzahl der gefundenen isomorphen Knoten und der Zeitaufwand von der �berpr�fung der Nachbarschaft der Zeitkomplexit�t $\mathcal{O}(1)$, dann ist der Zeitaufwand der Suche der Pyramide im schlechtesten Fall $\mathcal{O}(\frac{1}{6} m (n^3-n^2))$. Offensichtlich gilt es $m < n$ wegen der Definition des isomorphen Knoten. Die Idee der Vereinfachung ist, anstatt der Nachbarn eines isomorphen Knoten die isomorphen Knoten selbst zu betrachten, um den obigen schlechtesten Fall zu vermeiden. Die Bedingung f�r die Pyramide kann auch geschw�cht werden, d.h. man kann die einfacheren geometrischen Elemente benutzen, wie z.B. das Dreieck sogar nur zwei verbundenen Kanten. $m$ isomorphe Knoten werden durchgesucht, um zu bestimmen, ob sie sich zumindest mit zwei anderen isomorphen Knoten verbinden. Der Zeitaufwand dieses Ablaufes ist nur $\mathcal{O}(m^2)$, was deutlich kleiner als das Verfahren mit Suche der Pyramide ist. Abbildung~\ref{AI} zeigt ein Beispiel f�r den Vergleich der beiden Nebenbedingung des Isomorphen-Problems. Auf der linken Seite versucht das Programm, eine Pyramide an den isomorphen Knoten zu finden. Alle Kombinationen der drei Nachbarn werden f�r jeden isomorphen Knoten betrachtet. Dadurch ist der Zeitaufwand der linken Seite:

\[
\binom{3}{3}(V_1) + \binom{3}{3}(V_2) + \binom{4}{3}(V_3) + \binom{6}{3}(V_4) + \binom{5}{3}(V_5) = 1 + 1 + 4 + 20 + 10 = 36.
\]

An der rechten Seite werden aber nur die isomorphen Knoten selbst betrachtet, wobei $m^2=5^2$ Zeitaufwand kostet. Alle isomorphen Knoten, die diese Bedingung erf�llen, werden dann gespeichert, um die Orientierung des erkannten Objektes zu berechnen. 

\begin{figure}
\centering
\includegraphics[scale=0.5]{Abbildungen/AlgoIsomorphismus.png}
\caption{Ein Beispiel f�r den Vergleich der zwei Endbedingungen. $V_1$ bis $V_4$ sind die isomorphen Knoten und $N_1$ bis $N_6$ sind die Nachbarn von Knoten $V_4$. Auf der linken Seite wird das Verfahren mit der Pyramide gezeigt, wobei alle isomorphen Knoten durchgesucht werden, bis eine Pyramide $V_4 N_5 N_6 N_1$ gefunden ist. Auf der rechten Seite werden aber nur die Kanten zwischen den isomorphen Knoten betrachtet.}
\label{AI}
\end{figure}

\subsection{Bestimmung der Orientierung}
Die Verfolgung des Objekts ist die andere wichtige Aufgabe dieser Arbeit, was durch die Berechnung der Transformation realisiert wird. Die Anfangslage eines Objekts wird von dem Strukturgraphen definiert. In jedem Bild wird die Rotations- bzw. die Translationsmatrix zwischen dem Anfangs- und aktuellen Zustand mithilfe der ausgew�hlten isomorphen Knoten bestimmt. Das Orientierungsverfahren verl�uft in gleicher Weise wie das Verfahren im Lernprozess, der im Abschnitt~\ref{BdO} erkl�rt wurde.

\section{Bildersteuerung}
\label{BildS}
Das Speichern und die Auswahl der Bilder sind die Hauptaufgaben des Teilprogramms der Bildersteuerung. Die Verfolgung der Marken wird durch den Vergleich der Marken von zwei nachfolgenden Bildern realisiert, d.h. das Programm soll immer zwei Bilder gleichzeitig betrachten: Das aktuelle- und das historische Bild. Au�erdem werden die Marken wegen dem Rauschen und anderen St�rungen aus den Eingabebildern teilweise schlecht erkannt. Die falschen Marken verst�ren die Korrespondenzuntersuchung und verschlechtern weiterhin die Orientierung der Objektlagen zwischen zwei Bildern, was aber ganz wichtig f�r das Lernen des Objekts ist. Wenn irgendwelche kleinen Fehler in der Transformation vorkommen, wird der Strukturgraph des lernenden Objekts komplett anders dargestellt. Deshalb sollen die schlechten Eingabebilder vor der Darstellung des Strukturgraphen ausgew�hlt und von dem Bilderstrom entfernt werden.
\\
\\
In dieser Arbeit werden alle Eingabebilder zuerst in einer Warteschlange gespeichert. Die Warteschlange wird mit fester Gr��e definiert. Wenn ein neues Bild eingegeben wird, wird es am Ende der Schlange eingef�gt und das �lteste Bild aus der ersten Stelle der Schlange gestrichen. In jedem Zyklus des Programms werden das erste und letzte Bild verglichen und die korrespondierenden Punktpaare daraus gefunden. Mit anderen Worten definiert die Gr��e der Warteschlange aber auch das Intervall der betrachteten Bilder. Die Markenanalyse und die Orientierung von Objektlernen werden regelm��ig f�r jedes Eingabebild durchgef�hrt, aber die davon erhaltenen Rotations- bzw. Translationsmatrix werden nicht direkt f�r die Aktualisierung des Strukturgraphen benutzt, sondern zuerst gepr�ft, damit die Qualit�t der Transformation bewerten werden kann. 
\\
\\
Das Pr�fungsprogramm besteht aus zwei Teilen. Der erste Teil ist ein Pr�fer, der eine boolesche Aussage liefert, ob aus dem Bild genug hochqualitative Marken gefunden werden k�nnen. Der Pr�fer l�uft analog wie die Aktualisierung der Knoten von dem Teilprogramm des Strukturgraphen, was im Abschnitt~\ref{BdsK} beschrieben wurde. Der Unterschied liegt darin, dass der Pr�fer die Lebenszeit der Marken nicht betrachtet, sondern nur die neu gefundenen Marken z�hlt, die mit den vorhandenen Knoten des Strukturgraphen nach gerade berechneter Transformation identisch sind. Nur das Bild mit genug Marken, die die obige Bedingung erf�llen, wird von dem Pr�fer akzeptiert. Die Beurteilung �ber die Anzahl der Marken wird durch den Vergleich der Quote, die den Anteil der mit vorhandenen Knoten �bereinstimmenden Marken an allen neuen Marken beschreibt, mit einem vordefinierten Schwellwert realisiert. Die vom Pr�fer akzeptierten Bilder k�nnen direkt f�r die Aktualisierung der Strukturgraphen genutzt werden. Die anderen Bilder sind zwar �berfl�ssig, k�nnen aber nicht direkt verworfen werden. Zu einer schlechten Beobachtungssituation liefert die Kamera m�glicherweise in langer Zeit gar keine hochqualitativen Bilder. Wenn das Programm all diese Bilder �berspringt, wird die Verfolgung des Objekts abgebrochen und falsche Erkennungsergebnisse erzeugt. Ein extremes Beispiel ist bei der Drehung des K�stchens gegeben. Wenn nur die Bilder von zwei benachbarten Ebenen des Objekts von dem Programm akzeptiert, aber die Bilder des Rotationsablaufs dazwischen nicht ber�cksichtigt werden, hat das Programm aber gar keine M�glichkeit, diese zwei Ebenen zu unterscheiden. Sie werden als eine gro�e Ebene erkannt und gespeichert. Der zweite Teil des Pr�fungsprogramms vermeidet diese Situation. Die maximale Anzahl der �bersprungenen Bilder wird beschr�nkt. Wenn kein hochqualitatives Bild gefunden werden kann, wird ein verh�ltnism��ig besseres Bild aus den schlechten Bildern ausgew�hlt. Die Summe aller gr��ten Elemente von $P$, die im Abschnitt \textbf{Qualit�ts�berpr�fung der Markenverfolgung} von \ref{MV} erkl�rt wurde, wird hier als die Bewertungsvariable verwendet. Der Durchlauf des Pr�fungsprogramms wird im Algorithmus~\ref{algBP} aufgef�hrt.

\begin{algorithm}                     
\caption{Bilderpr�fer($Strukturgraph$, $Eingabebild$, $UebersprungeneAnzahl$)}         
\label{algBP}                          
\begin{algorithmic}
\State die minimale Quote der mit vorhandenen Knoten �bereinstimmten Marken: $q$
\State die maximale Anzahl der Bilder, die �bersprungen werden d�rfen: $N_{jump}$
\State Iterator des Besten Bild in $Bildschlange$: $i_b$
\If {$Korrespondenzuntersuchung()$ == True und $Identische Quote>q$}
	\State Aktualisieren $Strukturgraph$ mit $Eingabebild$
\Else
	\If {$Korrespondenzuntersuchung()$ == False}
		\State �berspringen
	\EndIf
	\If {$Eingabebild.SumP > Bildschlange[i_b].SumP$}
		\State Entfernen $Bildschlange[i_b]$
		\State $i_b \gets Eingabebild.iterator$
	\Else
		\State �berspringen
	\EndIf
	\State $UebersprungeneAnzahl$ ++
	\If {$UebersprungeneAnzahl > N_{jump}$}
		\State Aktualisieren $Strukturgraph$ mit $Bildschlange[i_b]$
		\State $UebersprungeneAnzahl \gets 0$
	\EndIf
\EndIf
\end{algorithmic}
\end{algorithm}
\chapter{Experimentelle Auswertung}
In diesem Abschnitt wird das Programm von verschiedenen Richtungen ausgewertet. Zwei schwarze K�stchen mit wei�en Marken werden als Testobjekte in der Evaluation verwendet (siehe Abb.~\ref{BOXES}. Jede Translation bzw. Rotation des Objektes wird manuell durchgef�hrt, d.h. die kleine Schwingung von menschlicher Bewegung ist in unserem Test auch betrachtet.

\begin{figure}[htbp]
\centering
\includegraphics[scale=0.42]{Abbildungen/Boxes.jpg}
\caption{Die Testobjekte. Das gr��ere K�stchen auf der linken Seite ist das Haupttestobjekt, was in den meisten Evaluationen �ber ein Objekt verwendet wurde.}
\label{BOXES}
\end{figure}

\section{Teilweise Evaluation}
Um die bestm�glichen Ergebnisse zu erhalten, werden viele Hilfsteilprogramme neben dem Lernen- bzw. Wiedererkennungsprozess gleichzeitig durchgef�hrt. Die Wirkungen und der zus�tzlicher Zeitaufwand dieser Teilprogramme werden in diesem Abschnitt separat diskutiert.

\subsection{Abstands-Filter}
Das Ziel des Abstands-Filters ist, die interessanten Objekte aus der Umgebung zu extrahieren. Dadurch kann der Detektor nur den kleinen Bereich um die Objekte fokussieren, wodurch St�rungen der Umgebung vermieden werden k�nnen. Wegen der starken Abh�ngigkeit zwischen dem Detektor und dem Abstands-Filter, kann der Einfluss des Filters durch die Anzahl der erkannten Marken bewertet werden. Hier werden aber zwei Testbildstr�me f�r ein bewegendes Objekt mit unterschiedlichen Hintergr�nden verwendet, um die Wirkung des Abstands-Filters zu erkl�ren. Die Screenshots f�r je 30 Bilder wird in Abbildung~\ref{E} bzw. \ref{F} gezeigt. Insgesamt acht Marken werden auf der Ebene aufgebracht, die von der Kamera als wei�e Kreise aufgenommen werden sollen. Die Marken, die von dem Programm erkannt werden k�nnen, werden dann wieder in rot gef�rbt.

\begin{figure}[htbp]
\centering
\includegraphics[scale=0.42]{Abbildungen/Empty.png}
\caption{Der Bildstrom mit homogenem Hintergrund. Von links nach rechts und oben nach unten sind die Screenshots der Bilder, deren Index ab 60 beginnt und deren Schrittweite 30 betr�gt.}
\label{E}
\end{figure}

\begin{figure}[htbp]
\centering
\includegraphics[scale=0.5]{Abbildungen/DF_Empty.png}
\caption{Der Vergleich der Erkennungsergebnisse mit und ohne Abstands-Filter. Die Eingabebilder enthalten nur ein Objekt und die homogene Umgebung, was in Abbildung~\ref{E} teilweise gezeigt wird.}
\label{DFE}
\end{figure}

Abbildung~\ref{DFE} zeigt die Anzahl der erkannten Marken aus dem Bildstrom mit homogenem Hintergrund. Die blaue und die gr�ne Kurve zeigen jeweils das Erkennungsergebnis mit und ohne Abstands-Filter. Die Kurve der Erkennungsergebnisse mit Abstands-Filter schwingt zwischen dem Intervall von sechs bis neun, was aber deutlicher schw�cher als die Gr�ne ist. D.h, dass die Verwendung des Abstands-Filters die Erkennungsergebnisse verbessert und die Ausgaben viel stabiler sein l�sst. Die Vergleichsergebnisse f�r den Bildstrom mit inhomogener Umgebung wird in der Abbildung~\ref{DFF} gezeigt. Die Erkennungsergebnisse mit und ohne Abstands-Filter dieses Tests scheinen ein bisschen schlechter als die Ergebnisse in der Abbildung~\ref{DFE} zu sein. Die Verbesserung durch den Abstands-Filter ist dennoch deutlich.   

\begin{figure}[htbp]
\centering
\includegraphics[scale=0.42]{Abbildungen/Full.png}
\caption{Der Bildstrom mit inhomogener Umgebung. Von links nach rechts und oben nach unten sind die Screenshots der Bilder, deren Index ab 60 beginnt und deren Schrittweite 30 betr�gt.}
\label{F}
\end{figure}

\begin{figure}[htbp]
\centering
\includegraphics[scale=0.5]{Abbildungen/DF_Full.png}
\caption{Der Vergleich der Erkennungsergebnisse mit und ohne Abstand-Filter. Die Eingabebilder enthalten mehr Objekte, was in Abbildung~\ref{F} teilweise gezeigt wird.}
\label{DFF}
\end{figure}

\subsection{Helligkeitssteuerung}
\label{Hs}

\begin{figure}
\centering
\includegraphics[scale=0.5]{Abbildungen/BC_Empty.png}
\caption{Der Vergleich der Erkennungsergebnisse mit und ohne Helligkeitssteuerung. Die Eingabebilder enthalten nur ein Objekt und die homogene Umgebung, was in Abbildung~\ref{E} teilweise gezeigt wird.}
\label{BCE}
\end{figure}

In der Bewertung der Helligkeitssteuerung wird der gleiche Testbildstrom verwendet, der in der Abbildung~\ref{E} aufgef�hrt wird. Die Abbildung~\ref{BCE} zeigt die Vergleichsergebnisse der Anzahl der erkannten Marken von dem Programm, das jeweils mit und ohne Helligkeitssteuerung durchgef�hrt wird. Die blaue Kurve ist die normale statistische Kurve f�r die erkannten Marken, welche identisch  zu der blauen Kurve in Abbildung~\ref{DFE} ist. Die gr�ne Kurve, die die Erkennungsergebnisse ohne Helligkeitssteuerung beschreibt, schwingt zwischen dem Intervall des Bilderindexes von 200 bis 300 ziemlich stark. Diese unregelm��ige Schwingung liegt darin, dass das Objekt w�hrend diesen Bildern entlang der Blickrichtung der Kamera bewegt wird. Wegen dem Arbeitsprinzip der PMD Kamera steigt die Amplitude �ber das Objekt in dem Bild an, wenn das Objekt zu der Kamera bewegt wird. Das verringert aber den Unterschied zwischen den Marken und deren Umgebung. Dadurch werden viel mehr Marken erkannt als tats�chlich vorhanden sind, weil einige Bereiche auf der Ebene des Objektes eine �hnliche Helligkeit wie die Marken haben. Im umgekehrten Falle sinkt die Amplitude des Objektes ab, wenn das Objekt weiter von der Kamera entfernt wird. Dann werden aber nur weniger Marken als die gew�nschte Anzahl aus dem Objekt erkannt.

\subsection{Verbesserung des Singul�rwertzerlegungsverfahrens}
\label{VdS}
\begin{figure}[htbp]
\centering
\includegraphics[scale=0.35]{Abbildungen/2DT.png}
\caption{Der Testbildstrom mit dem Objekt, das nur in der Ebene bewegt wird.}
\label{DT}
\end{figure}

\begin{figure}
\centering
\includegraphics[scale=0.5]{Abbildungen/2dt_a.png}
\caption{Der Vergleich der umdrehenden Winkel mit und ohne Verbesserung f�r ein im zweidimensionalen Raum bewegendes Objekt.}
\label{DTA}
\end{figure}

Im Abschnitt~\ref{VdS} wird die Verbesserung des Singul�rwertzerlegungsverfahrens erkl�rt. In diesem Teil der Evaluation werden die Wirkungen dieser Verbesserung bewertet. Die verschiedenen bewegenden bzw. umdrehenden Bewegungen des Objekts im zweidimensionalen bzw. dreidimensionalen Raum werden in drei Testbildstr�men zusammengefasst, welche in Abbildungen~\ref{DT}, \ref{DR} und \ref{3DR} teilweise gezeigt werden. Das Ziel des Singul�rwertzerlegungsverfahrens ist, die Korrespondenzpunkte zu finden, damit die Orientierung des Objekts zwischen zwei Zeitpunkten berechnet werden kann. Deshalb soll die Transformation des Objekts m�glich genau bestimmt werden, um das bessere Ergebnis zu erhalten. Aus diesem Grund wird der umdrehende Winkel des Objekts als Bewertungsparameter in diesem Teil der Evaluation verwendet. Die Diagramme �ber den Vergleich des umdrehenden Winkels mit und ohne Verbesserung des Singul�rwertzerlegungsverfahrens werden in Abbildungen~\ref{DTA}, \ref{DRA} und \ref{3DRA} gezeigt.

\begin{figure}[htbp]
\centering
\includegraphics[scale=0.26]{Abbildungen/2DR.png}
\caption{Der Testbildstrom mit dem Objekt, das nur in der Ebene umgedreht wird.}
\label{DR}
\end{figure}

\begin{figure}[htbp]
\centering
\includegraphics[scale=0.5]{Abbildungen/2dr_a.png}
\caption{Der Vergleich der umdrehenden Winkel mit und ohne Verbesserung f�r ein im zweidimensionalen Raum umdrehendes Objekt.}
\label{DRA}
\end{figure}

\begin{figure}[htbp]
\centering
\includegraphics[scale=0.42]{Abbildungen/3DR.png}
\caption{Der Testbildstrom mit dem Objekt, das im dreidimensionalen Raum umgedreht wird.}
\label{3DR}
\end{figure}

\begin{figure}[htbp]
\centering
\includegraphics[scale=0.5]{Abbildungen/3dr_a.png}
\caption{Der Vergleich der umdrehenden Winkel mit und ohne Verbesserung f�r ein im dreidimensionalen Raum umdrehendes Objekt.}
\label{3DRA}
\end{figure}

In den Testbeispielen von zweidimensionaler Bewegung und Rotation wird der Vorteil der Verbesserung nicht deutlich gezeigt: In der Abbildung~\ref{DTA} gibt es entweder in der blauen Kurve oder in der gr�nen Kurve einige hohen Spitzen; in der Abbildung~\ref{DRA} laufen beide Kurven aber ruhig durch. Die Wirkung der Verbesserung scheint in Abbildung~\ref{3DRA} deutlicher, welche eine Statistik �ber den umdrehenden Winkel des Objekts im dreidimensionalen Raum erstellt. Mithilfe der Screenshots von Abbildung~\ref{3DR} findet man, dass die umdrehenden Winkel zu gro� berechnet werden, wenn eine neue Ebene des Objektes unter der Kamera vorkommt. Die Verbesserung des Singul�rwertzerlegungsverfahrens l�sst den Fehler des berechneten Winkels in den Bereichen von Gr��e und Zeitraum (in wie vielen Bildern der Fehler vorkommt) unterdr�cken. Abbildung~\ref{3DREC} zeigt die Endergebnisse mit und ohne Verbesserung �ber den Testbildstrom mit dreidimensionaler Umdrehung. Die gr�nen, kleinen Kugeln sind die neu erkannten Marken und die gro�en Kugeln beschreiben die stabilen Knoten. Um die Struktur des Objekts deutlicher zu zeigen, werden die stabilen Knoten in gleicher Farbe gesetzt, wenn sie in der gleichen Ebene erkannt worden sind. Die roten Knoten beschreiben die erste Ebene des K�stchens. Danach befindet sich die zweite, dritte bzw. vierte Ebene, die jeweils in Gelb, Cyan und Magenta gezeichnet werden. Die f�nfte Farbe, Orange, beschreibt die gleiche Ebene wie Rot, die nach einer kompletten Rotation wieder von dem Programm erkannt wird. Die Verbesserung durch das verbesserte Singul�rwertzerlegungsverfahren erkennt man durch die zwei Strukturgraphen aus Abbildung~\ref{3DREC} deutlich. Die Knoten aus dem Schaubild der ersten Zeile k�nnen die vier Ebene eines Hexaeders sehr gut darstellen. Aber die Situationen in zweiter Zeile sind ungeordnet. Es gibt keine M�glichkeit, eine Gestalt aus den stabilen Knoten herauszufinden.

\begin{figure}[htbp]
\centering
\includegraphics[scale=0.18]{Abbildungen/3DR_end_c.png}
\caption{Die Strukturgraphen eines K�stchens nach Lernen mit (erste Zeile) und ohne (zweite Zeile) Verbesserung des Singul�rwertzerlegungsverfahrens, die jeweils von vorne, oben, und einem Punkt an der Verl�ngerung der Diagonale des Objektes beobachtet werden. Die Knoten mit gleicher Farbe werden in gleicher Ebene erkannt.}
\label{3DREC}
\end{figure}

\subsection{Aktualisierung des Strukturgraphen}
Der Strukturgraph des Objektes wird durch den Algorithmus~\ref{algAG} in Kapitel 4 dargestellt. Die ,,Lebenszeit'' und der ,,maximale Abstand der identischen Knoten'' sind die zwei wichtigsten Parameter des Algorithmus, wodurch die stabilen Knoten aus dem Rauschen erkannt werden k�nnen. In diesem Abschnitt werden die Wirkungen mit verschiedenen Zuordnungen dieser zwei Parameter bzw. die Verbesserung der zus�tzlichen Kombination der mehrfach erkannten Knoten diskutiert. Das erste Objekt mit 25 Marken auf vier Ebenen wird hier als das Testobjekt benutzt. Eine Statistik �ber die Anzahl der stabilen Knoten, die nach dem Ablauf des Programms in VTK  Daten gespeichert werden sollen, wird zuerst erstellt und dann mit der tats�chlichen Anzahl der Knoten des Objekts verglichen, damit die unterschiedliche Auswahl der Parameter bewertet werden kann.   

\subsubsection{Lebenszeit}

\begin{figure}
\centering
\includegraphics[scale=0.5]{Abbildungen/gu_lt_all.png}
\caption{Der Vergleich der Anzahl der erkannten Knoten und der tats�chlichen Knoten, die am Testobjekt angebracht werden. Die verschiedenen Lebenszeiten werden als die vordefinierten Parameter des Programms getestet.}
\label{GULTA}
\end{figure}

Die blau gepunktete Linie in Abbildung~\ref{GULTA} bleibt bei 25, was der tats�chlichen Anzahl der Marken auf dem Testobjekt entspricht. Die gr�ne Kurve zeigt die Anzahl der erkannten Knoten im Strukturgraphen nach dem Lernen mit verschiedenen Eingaben der Lebenszeit. Wenn die Lebenszeit zu klein definiert ist, werden die neu erkannten Marken ziemlich schnell als die stabilen Knoten markiert und im Strukturgraphen eingef�gt. Es folgen zwei negative Wirkungen. Zuerst werden gleiche Marken mehrmals als unterschiedliche Knoten erkannt. Zweitens werden die zuf�llig vorkommenden Rauschpunkte auch als stabile Knoten erkannt und im Endergebnis gespeichert. Die beste Anpassung kommt zwischen 30 bis 40 vor, wo fast gleich so viele Marken wie am realen Objekt erkannt werden. Wegen der Verst�rkung der Erkennungsbedingung �ber die stabilen Knoten sinkt danach die Anzahl der stabilen Knoten langsam mit der Vergr��erung der Lebenszeit ab. Die erste Zeile des Schaubildes~\ref{GUE} zeigt die Strukturgraphen nach dem Lernen mit verschiedener Lebenszeit.

\begin{figure}
\centering
\includegraphics[scale=0.29]{Abbildungen/GU_end.png}
\caption{Die Screenshots f�r die Endergebnisse des Lernens mit verschiedenen Lebenszeit. Die Zahlen am Boden der Abbildung zeigen die Testwerte der Lebenszeit. Die erste Zeile zeigt die direkten Ergebnisse von dem Programm. Die L�sungen mit der zus�tzlichen Kombination der mehrfach erkannten Knoten werden in der zweiten Zeile dargestellt.}
\label{GUE}
\end{figure}

\subsubsection{Abstandschwellenwert f�r die identischen Knoten}

\begin{figure}
\centering
\includegraphics[scale=0.5]{Abbildungen/gu_dis_all.png}
\caption{Der Vergleich der Anzahl der erkannten Knoten und der tats�chlichen Knoten, die am Testobjekt angebracht werden. Die verschiedenen, maximalen Abst�nde der identischen Knoten werden als die vordefinierten Parameter des Programms getestet.}
\label{GUDA}
\end{figure}

Der Abstandschwellenwert f�r die identischen Knoten erkl�rt, wie nahe zwei Knoten beieinander liegen sollen, die als identische Knoten des Strukturgraphen erkannt werden. Die Ver�nderung f�r die Anzahl der stabilen Knoten mit Anstieg des Abstandschwellwerts wird durch die gr�ne Kurve in Abbildung~\ref{GUDA} gezeigt. Je kleiner der Abstandschwellwert ist, desto schwieriger werden die neu gefundenen Knoten erkannt, die mit den vorhandenen Knoten identisch sind. Aber auf der anderen Seite werden die Strukturgraphen falsch dargestellt, wenn der Abstandschwellwert zu gro� definiert wird. In diesem Fall k�nnten zwei benachbarte Marken als ein Knoten erkannt werden, was sogar die Orientierung st�ren kann. Ein Beispiel daf�r findet man im letzten Schaubild der Abbildung~\ref{GUDE} mit dem Abstandschwellenwert gleich 0.0045.

\begin{figure}
\centering
\includegraphics[scale=0.29]{Abbildungen/GU_dis_end.png}
\caption{Die Screenshots f�r die Endergebnisse des Lernens mit verschiedenen Abstandschwellwerten f�r die identischen Knoten. Die Zahlen an dem Boden der Abbildung listen die getesteten Werte auf.}
\label{GUDE}
\end{figure}

\subsubsection{Kombination der mehrfach erkannten Knoten}
Was deutlich in beiden Abbildungen~\ref{GULTA} und \ref{GUDA} erscheint, ist, dass es nur ein Schnittpunkt zwischen der blauen und der gr�nen Kurven gibt. D.h., dass entweder die Lebenszeit oder der Abstandschwellwert schwierig zu bestimmen ist, damit genau so viele stabile Knoten wie die tats�chliche Anzahl der Marken am Objekt erkannt werden. Deshalb soll die Kombination der mehrfach erkannten Knoten nach dem Lernen durchgef�hrt werden. Dadurch kann die g�ltige Definitionsmenge der zwei Parameter vergr��ert werden. Die roten Kurven in den Abbildungen~\ref{GULTA} und \ref{GUDA} zeigen die Anzahl der stabilen Knoten nach der Kombination. In beiden Kurven gibt es ein relativ gro�es Intervall, in dem sich die Anzahl der erkannten stabilen Knoten der tats�chlichen Anzahl der Marken des Objekts sehr gut anpasst (24 gegen 25). Der Vergleich der Lernergebnisse mit und ohne die Kombination wird durch das Schaubild~\ref{GUE} erkl�rt. Die Bilder in zweiter Zeile zeigen die Strukturgraphen nach dem Lernen mit der Kombination �ber verschiedenen Eingaben der Lebenszeit. Wegen dieser Ver�nderung gibt es kaum einen Unterschied zwischen den L�sungen mit Lebenszeit von 12 bis 35.

\subsection{Bildersteuerung}
\label{Bs}
Wie in der Abbildung~\ref{3DRA} dargestellt, werden die umdrehenden Winkel zu gro� berechnet, wenn eine Ebene des Objekts allm�hlich verschwindet und die nachfolgende Ebene langsam vorkommt. Diese falschen Winkel k�nnen einen gro�en Fehler f�r die Darstellung des Strukturgraphen erzeugen, was in Abbildung~\ref{3DRFC} klar gezeigt wird. Alle Ebenen sind richtig erkannt, k�nnen aber leider keine sinnvolle Gestalt aufbauen. Deshalb ist die Bildersteuerung notwendig, damit die schlechten Bilder aus dem Bildstrom entfernt werden k�nnen. Abbildung~\ref{3DRFCA} zeigt das statische Diagramm �ber den Vergleich der umdrehenden Winkel mit und ohne Bildersteuerung w�hrend des Testbildstroms aus Abbildung~\ref{3DR}. Abgesehen von der gro�en Spitze  der blauen Kurve am Anfang, die wegen der schnellen Bewegung zu dem Bildbereich der Kamera als Rauschen erzeugt wird, l�uft die blaue Kurve ruhig in einem kleinen Intervall ohne gro�e Spitze, die aber bei der gr�nen Kurve h�ufig beobachtet werden. D.h. mithilfe der Bildersteuerung kann das Programm die negative Wirkung der falsch berechneten Winkel vollst�ndig vermeiden.

\begin{figure}
\centering
\includegraphics[scale=0.5]{Abbildungen/3dr_fc.png}
\caption{Der Vergleich der umdrehenden Winkel mit und ohne Bildersteuerung f�r ein im dreidimensionalen Raum umdrehendes Objekt. Der Bildstrom wird im Abbildung~\ref{3DR} teilweise gezeigt.}
\label{3DRFCA}
\end{figure}

\begin{figure}
\centering
\includegraphics[scale=0.18]{Abbildungen/3DR_end_wfc.png}
\caption{Der Strukturgraph eines K�stchen nach dem Lernen ohne Bildersteuerung, der jeweils von vorne, oben und einem Punkt an der Verl�ngerung der Diagonale des Objekts beobachtet wird. Die Knoten mit gleicher Farbe werden in gleicher Ebene erkannt.}
\label{3DRFC}
\end{figure}

\subsection{Teilgraph-Isomorphismus}
Um die Suche der sogenannten isomorphen Knoten zu vereinfachen, wurden zwei Quoten im Abschnitt~\ref{Oerk} definiert. In diesem Teil der Evaluation �ber Teilgraph-Isomorphismus wird das Wiedererkennungsteilprogramm mehrmals mit unterschiedlichen Abstandsquoten und Nachbarquoten durchgef�hrt, und die Anzahl der Bilder, in denen das Objekt erfolgreich erkannt werden kann, gespeichert. Die Quote dieser Anzahl und der Anzahl gesamter Eingabebilder, welche Erkennungsquote genannt wird, kann die Qualit�t der Wiedererkennung bewerten. Die blaue Kurven in den Abbildungen~\ref{GIDIS} und \ref{GINODE} zeigen genau diese Erkennungsquote mit verschiedenen Vorgaben der Abstandsquote bzw. der Nachbarquote. Der Testbildstrom ist den Testdaten identisch, was in den Abschnitten~\ref{VdS} und \ref{Bs} verwendet, und in Abbildung~\ref{3DR} teilweise aufgef�hrt wird. Aus den statistischen Diagrammen ist zu entnehmen, dass die beste Wiedererkennungsquote dann vorkommt, wenn die Abstandsquote gleich 5\% und die Nachbarquote gleich 80\% ist. 
\\
\\
Die gr�nen Kurven beschreiben die richtigen Wiedererkennungsquoten, wenn ein zus�tzlicher Strukturgraph der erlernten Objekte von dem Programm als zweites Eingabemodell eingelesen wird. Der Unterschied zwischen den gr�nen und den blauen Kurven zeigt die St�rung von dem zweiten Eingabemodell, welches in einem akzeptierten Intervall (weniger als 2\%) liegt. Das erbringt auch einen �berzeugenden Nachweis f�r die hohe Stabilit�t unseres Teilgraph-Isomorphismus-Algorithmus.
 
\begin{figure}
\centering
\includegraphics[scale=0.5]{Abbildungen/gi_3dr_dis.png}
\caption{Die Quote der richtigen Wiedererkennungen mit unterschiedlichen Abstandschwellenwerten f�r ein im dreidimensionalen Raum umdrehendes Objekt. Der Bildstrom wird im Abbildung~\ref{3DR} teilweise gezeigt.}
\label{GIDIS}
\end{figure}

\begin{figure}
\centering
\includegraphics[scale=0.5]{Abbildungen/gi_3dr_node.png}
\caption{Die Quote der richtigen Wiedererkennungen mit unterschiedlichen Nachbarquoten f�r ein im dreidimensionalen Raum umdrehendes Objekt. Der Bildstrom wird im Abbildung~\ref{3DR} teilweise gezeigt.}
\label{GINODE}
\end{figure}

\section{Globale Evaluation}
In diesem Abschnitt werden die kompletten Abl�ufe von Lernen und Wiedererkennung bewertet. F�r das Lernen werden die Strukturgraphen, die w�hrend des Lernprozesses durch die erkannten Marken erzeugt werden, mit den originalen Objekten verglichen. F�r die Wiedererkennung wird die sogenannte richtige Erkennungsquote betrachtet, die die Qualit�t der Wiedererkennung sehr gut beschreiben kann. Nat�rlich ist der Zeitaufwand beider Teile von hoher Bedeutung, was jeweils in einigen Tabellen aufgelistet wird.

\subsection{Objektlernen}
Wegen der fehlenden F�higkeit der Messungen ist die quantitative Evaluation des Lernens nicht m�glich. Die Lernergebnisse k�nnen aber mithilfe der graphischen Visualisierung auch sehr gut beobachtet werden. Deshalb werden im folgenden Teil dieses Abschnitts einige Screenshots des Programms gezeigt, wodurch man die komplette Lernphase f�r verschiedene Eingabeobjekte bewerten kann. 
\\
\\
Die Abbildung~\ref{gl} zeigt einen normalen Lernprozess eines Objekts basierend auf dem Eingabebildstrom von \ref{3DR}. Die Strukturgraphen, die in der dritten Zeile gezeichnet werden, werden schrittweise durch die Eingabebilder dargestellt und orientieren sich an der aktuellen Position des Objektes. Um die verschiedenen Ebenen des Objektes deutlich zu unterscheiden, werden die Farben jeder Ebene manuell ver�ndert. Das Programm stoppt, wenn die erste Ebene wieder vorkommt, welche in der Abbildung~\ref{gl} in Rot und Organe f�r die jeweils erste und zweite Erscheinung gef�rbt werden. Was deutlich beobachtet werden kann, ist, dass alle Knoten, die zu einer gleichen Ebene geh�ren, in einer Fl�che sehr gut erkannt werden. Weiterhin passen sich die relativen Ausrichtungen zwischen je zwei Ebenen der Wirklichkeit gut an. Der Nachweis daf�r ist die teilweise �bereinstimmung von den roten Knoten und den orangenen Knoten, welche eigentlich von den gleichen Marken des Objekts erkannt werden. Die Abbildung~\ref{Lend} zeigt den Vergleich zwischen den Strukturgraphen und der realen Anordnung der Marken am Objekt.

\begin{figure}[htbp]
\centering
\includegraphics[scale=0.42]{Abbildungen/global_l.png}
\caption{Die Screenshots des Programms. Die Schaubilder der ersten Zeile zeigen die Grauwertbilder der Amplitude mit roten Marken. Die entsprechenden 3D-Daten werden in der zweiten Zeile aufgef�hrt. In der dritten Zeile werden die mit bis aktuellen Bildern erzeugenden Strukturgraphen (ohne Kanten, ohne Kombination des mehrfach erkannten Knoten) gezeichnet.}
\label{gl}
\end{figure}

\begin{figure}[htbp]
\centering
\includegraphics[scale=0.26]{Abbildungen/L_end.png}
\caption{Der Vergleich zwischen den Strukturgraphen und der realen Anordnung der Marken des Objekts. Die roten Kugeln in den Schaubildern in der zweiten Zeile zeigen die stabilen Knoten des Strukturgraphen, welche den wei�en Marken der Fotos in der ersten Zeile entsprechen.}
\label{Lend}
\end{figure}

In Abbildung~\ref{BC2} werden die Screenshots des Eingabebilderstroms f�r das zweite Objekt teilweise dargestellt. Da dieses Objekt schlanker ist als das erste Objekt, werden die Abst�nde zwischen den Marken verkleinert, was die Erkennung bzw. die Verfolgung der Marken schwieriger werden l�sst. Aber mithilfe der Ver�nderung der Parameter der Algorithmen in Lernprozess kann trotzdem das erw�nschte Lernergebnis erhalten werden. Die erste Zeile der Abbildung~\ref{B2} zeigt den Strukturgraph des zweiten Objektes mit gleichen Parametern wie beim Lernen des ersten Objektes. Die falsche Orientierung zwischen der gelben und blauen Ebene sieht man deutlich im ersten Bild. Nach Abstieg des Erwartungsintervalls der Anzahl der bekannten Marken in der Helligkeitssteuerung (siehe Algorithmus~\ref{KS}), das Verst�rken der Beschr�nkung der gr��ten Element in Korrespondenzuntersuchung (siehe Algorithmus~\ref{algSVD}) und das Verkleinern der minimalen Lebenszeit des stabilen Knoten in der Darstellung der Strukturgraphen (siehe Algorithmus~\ref{algAG}) kann das Lernergebnis wie die zweite Zeile in der Abbildung~\ref{B2} gefunden werden. 

\begin{figure}[htbp]
\centering
\includegraphics[scale=0.35]{Abbildungen/Box2_cv.png}
\caption{Die Screenshots des Testbildstroms f�r das zweite Objekt.}
\label{BC2}
\end{figure}


\begin{figure}[thbp]
\centering
\includegraphics[scale=0.18]{Abbildungen/Box2.png}
\caption{Die Strukturgraphen des zweiten Objekts, wobei die erste Zeile das Lernergebnis mit gleichen Parametern wie beim Lernen des ersten Objektes zeigt, und die zweite Zeile das Lernergebnis mit unterschiedlichen Parametern aufzeichnet.}
\label{B2}
\end{figure}


\subsubsection{Zeitaufwand}
Die Tabelle~\ref{LZ} f�hrt die Laufzeit aller Teilprogramme in der Lernphase auf. Die Teilprogramme von CenSurE  Detektor und Helligkeitssteuerung, Markenerkennung und Visualisierung kosten viel mehr Zeit als die anderen Teilprogramme, was 77.55\% des gesamten Zeitaufwands betr�gt (siehe Abb.~\ref{LZP}). Die gr��te Anforderung der Markenerkennung ist ungeplant. Der Grund liegt darin, dass die Marken des Objektes nicht direkt durch den CenSurE Detektor erkannt werden k�nnen, sondern eine zus�tzliche Kombination der Merkmale ben�tigt. Diese Merkmale sind direkt von dem Detektor erkennbar und liegen in der N�he von einem anderen. Der reine durchschnittliche Berechnungszeitaufwand ohne Visualisierung betr�gt 53.7065 $ms$ und die entsprechende Framerate liegt bei 18.6 fps, was die Echtzeitbedingung leider nicht sehr gut erf�llt. 


\begin{table}[htbp]
\centering
\scalebox{0.69}{
\begin{tabular}{| c | c | c | c | c | c |}
\hline
Abstand-Filter & \multicolumn{2}{c|}{CenSurE Detektor und Helligkeitssteuerung} & \multicolumn{2}{c|}{Markenerkennung} & gesamter Zeitaufwand \\
\hline
3.5325 & \multicolumn{2}{c|}{13.2369} & \multicolumn{2}{c|}{24.2683} & \multicolumn{1}{c|}{\multirow{3}{*}{72.1992}}\\
\cline{1-5}
Segmentierung & Korrespondenzuntersuchung & Orientierung & Bildersteuerung & Visualisierung & \multicolumn{1}{c|}{} \\
\cline{1-5}
0.7505 & 0.1258  & 1.0755 & 2.3291 & 18.4927 & \multicolumn{1}{c|}{} \\
\hline
\end{tabular}
}
\caption{Die Zeitaufw�nde (in $ms$) aller Teilprogramme der Lernensphase.}
\label{LZ}
\end{table}

\begin{figure}[htbp]
\centering
\includegraphics[scale=0.55]{Abbildungen/l_time.png}
\caption{Die Anteile der Laufzeit aller Teilprogramme der Lernphase.}
\label{LZP}
\end{figure}

\subsection{Objektwiedererkennung}
In diesem Abschnitt wird die Qualit�t der Wiedererkennungen mit verschiedenen initialisierten Bedingungen betrachtet. Wie im \ref{Oerk} erkl�rt, werden die Strukturgraphen der erlernten Objekte, die auch als Eingabemodelle bezeichnet werden, am Anfang der Wiedererkennungsphase in dem Programm eingegeben. Die Anzahl der Eingabemodelle liefert einen starken Einfluss auf die Erkennungsergebnisse bzw. den Zeitaufwand. Je mehr die Eingabemodelle ber�cksichtigt werden, desto schlechter die Erkennungsergebnisse sind und mehr Zeit daf�r aufgewendet werden muss. Au�erdem ist die Anzahl der Objekte im aktuellen Eingabebildstrom auch eine wichtige Variable f�r Bewertung unseres Programmes. Deshalb werden drei verschiedene Kombinationen von der Anzahl der Eingabemodelle und der Anzahl der Eingabeobjekte in folgenden Teilabschnitten diskutiert.
\\
\\
Die Testbildstr�me, die im Abschnitt~\ref{VdS} verwendet wurden, werden hier f�r den Test mit nur einem Eingabeobjekt wieder benutzt. Diese drei Testbildstr�me beschreiben jeweils ein Objekt, das sich in der Ebene bewegt (siehe Abb.~\ref{DT}) und im zweidimensionalen bzw. dreidimensionalen Raum umdreht (siehe Abb.~\ref{DR} und Abb.~\ref{3DR}). Analog zu der Teilevaluation �ber den Teilgraph-Isomorphismus wird die Quote der richtigen Wiedererkennungen nach dem kompletten Durchlauf des Testbildstroms berechnet, was als die wichtigste Variable der Bewertung betrachtet wird. Au�erdem wird der durchschnittliche Zeitaufwand bzw. die Anzahl der verglichenen Knoten im Teilgraph-Isomorphismus aufgezeichnet.

\subsubsection{1 Eingabeobjekt mit 1 Eingabemodel}
Die Testergebnisse findet man in der Tabelle~\ref{11}. Da nur ein Eingabemodell mit den aktuellen Eingabebildern verglichen werden soll, sind die gesamte und die richtige Erkennungsquote identisch. Der durchschnittliche Zeitaufwand liegt unter 26 $ms$, was die Echtzeitbedingung sehr gut erf�llt.

\begin{table}[htbp]
\centering
\scalebox{0.63}{
\begin{tabular}{| l | c | c | c | c |}
\hline
Testbildstr�me & gesamte Erkennungsquote & richtige Erkennungsquote & Anzahl der isomorphen Knoten & Zeitaufwand jedes Bilds \\
\hline
Statisch & 91.46\% & 91.46\% & 4.1068 & 17.0391 $ms$ \\
\hline
2D Translation & 55.81\% & 55.81\% & 5.7813 & 25.7035 $ms$ \\
\hline
2D Rotation & 66.67\% & 66.67\% & 6.6237 & 22.9068 $ms$ \\
\hline
3D Rotation & 59.94\% & 59.94\% & 2.6265 & 18.8584 $ms$ \\
\hline
\end{tabular}
}
\caption{Die durchschnittlichen statistischen Daten der Wiedererkennung f�r unterschiedliche Testsamples, wobei nur ein Eingabeobjekt und ein Eingabemodell ber�cksichtigt werden.}
\label{11}
\end{table}

Die andere Situation f�r ein Eingabeobjekt und ein Eingabemodell ist, dass das Eingabeobjekt und Eingabemodel sich auf verschiedenen Objekten beziehen. In Idealfall soll das Eingabeobjekt w�hrend der Bildsequenz gar nicht erkannt werden. Die Tabelle~\ref{11f} listet die falschen Erkennungsquoten f�r verschiedenen Testbildstr�me auf. Im schlechtesten Fall gibt es aber nur weniger als 1\% Bilder in einer Bildsequenz, in den das Eingabeobjekt als das unabh�ngige Eingabemodell falsch erkannt wird.

\begin{table}[htbp]
\centering
\scalebox{0.8}{
\begin{tabular}{| l | c | c | c | c |}
\hline
Testbildstr�me & Statisch & 2D Translation & 2D Rotation & 3D Rotation \\
\hline
Falsche Erkennungsquote & 0.38\% & 0\% & 0\% & 0.86\% \\
\hline
\end{tabular}
}
\caption{Die durchschnittlichen falschen Erkennungsquoten f�r unterschiedliche Testsamples, wobei nur ein Eingabeobjekt und ein Eingabemodell ber�cksichtigt werden, die sich aber auf verschiedenen Objekten beziehen.}
\label{11f}
\end{table}

%\begin{table}[htbp]
%\centering
%\scalebox{0.63}{
%\begin{tabular}{| l | c | c | c | c |}
%\hline
%Testbildstr�me & gesamte Erkennungsquote & falsche Erkennungsquote & Anzahl der isomorphen Knoten & Zeitaufwand jedes Bilds \\
%\hline
%Statisch & 100\% & 16.41\% & 0.0115 & 27.9885 $ms$ \\
%\hline
%2D Translation & 100\% & 0\% & 0 & 30.6570 $ms$ \\
%\hline
%2D Rotation & 100\% & 0\% &0 & 29.0143 $ms$ \\
%\hline
%3D Rotation & 100\% & 46.80\% & 0.0259 & 25.7478 $ms$ \\
%\hline
%\end{tabular}
%}
%\caption{Die durchschnittlichen statistischen Daten der Wiedererkennung f�r unterschiedliche Testsamples, wobei nur ein Eingabeobjekt und ein Eingabemodell ber�cksichtigt werden.}
%\label{11}
%\end{table}

\subsubsection{1 Eingabeobjekt mit 2 Eingabemodellen}
Der deutliche Unterschied zu den Testergebnissen im letzten Abschnitt besteht darin, dass die richtige Erkennungsquote nicht immer gleich wie die gesamte Erkennungsquote ist. Der Grund liegt darin, dass das Eingabeobjekt in einigen Bildern als falsches Modell erkannt wird. Aber wegen der zufriedenstellenden Stabilit�t des Teilgraph-Isomorphismus-Algorithmus ist die Quote der falschen Wiedererkennungen klein (weniger als 2\%). Au�erdem steigt die durchschnittliche Laufzeit jedes Bildes mit der Zunahme der Eingabemodelle deutlich an, weil jetzt f�r jedes Eingabeobjekt der Teilgraph-Isomorphismus zweimal durchgef�hrt werden muss. Alle Testdaten werden in der Tabelle~\ref{12} aufgef�hrt.

\begin{table}[htbp]
\centering
\scalebox{0.63}{
\begin{tabular}{| l | c | c | c | c |}
\hline
Testbildstr�me & gesamte Erkennungsquote & richtige Erkennungsquote & Anzahl der isomorphen Knoten & Zeitaufwand jedes Bilds \\
\hline
Statisch & 91.46\% & 91.46\% & 4.1174 & 29.4591 $ms$ \\
\hline 
2D Translation & 56.98\% & 55.81\% & 5.7245 & 43.3256 $ms$ \\
\hline
2D Rotation & 66.67\% & 66.67\% & 6.6237 & 38.6487 $ms$ \\
\hline
3D Rotation & 60.24\% & 59.94\% & 2.7078 & 33.0331 $ms$ \\
\hline
\end{tabular}
}
\caption{Die durchschnittlichen, statistischen Daten der Wiedererkennung f�r unterschiedliche Testsamples, wobei nur ein Eingabeobjekt aber zwei Eingabemodelle ber�cksichtigt werden.}
\label{12}
\end{table}

\subsubsection{Verbesserung mit Kandidaten}
Mithilfe der Kandidaten kann ein Eingabeobjekt in dem Bildstrom verfolgt werden. Die Erkennungsergebnisse werden  in den Kandidaten gespeichert. Dadurch liefert das Programm st�ndig die Ausgaben, obwohl von dem aktuellen Bild kein entsprechendes Objekt gefunden werden kann. Tabelle~\ref{QA} listet die Erkennungsquoten f�r unterschiedlichen Testbildstr�me mit verschiedener Anzahl der Eingabemodelle auf, in der die Verbesserung der Nutzung der Kandidaten deutlich gezeigt wird. Diese Ver�nderungen k�nnen auch in dem Balkendiagramm in Abbildung~\ref{RR} beobachtet werden.

\begin{table}[htbp]
\centering
\scalebox{0.625}{
\begin{tabular}{| l | c | c | c | c | c | c |}
\hline
\multirow{2}{*}{Testbildstr�me} & Anzahl der  & \multicolumn{2}{c|}{ohne die Verbesserung der Kandidaten} & \multicolumn{2}{c|}{mit der Verbesserung der Kandidaten} & Differenz des\\
\cline{3-6}
& Eingabemodelle & Erkennungsquote & Zeitaufwand ($ms$) & Erkennungsquote & Zeitaufwand ($ms$) & Zeitaufwands ($ms$) \\
\hline
\multirow{2}{*}{Statisch} & 1 & 91.46\% & 17.0391 & 97.86\% & 25.8078 & 8.7687\\
\cline{2-7}
& 2 & 91.46\% & 29.4591 & 97.86\% & 40.5302 & 11.0712 \\
\hline
\multirow{2}{*}{2D Translation} & 1 & 55.81\% & 25.7035 & 100\% & 33.5872 & 7.8837\\
\cline{2-7}
& 2 & 55.81\% & 43.3256 & 97.67\% & 53.7907 & 10.4651 \\
\hline
\multirow{2}{*}{2D Rotation} & 1 & 66.67\% & 22.9068 & 89.96\% & 31.0645 & 8.1577\\
\cline{2-7}
& 2 & 66.67\% & 38.6487 & 89.96\% & 48.2867 & 9.6380\\
\hline
\multirow{2}{*}{3D Rotation} & 1 & 59.94\% & 18.8584 & 99.40\% & 27.8223 & 8.9639\\
\cline{2-7}
& 2 & 59.94\% & 33.0331 & 99.10\% & 45.4458 & 12.4127\\
\hline  
\end{tabular}
}
\caption{Die Erkennungsquote und der Zeitwand mit und ohne die Verbesserung der Kandidaten f�r unterschiedliche Testbildstr�me mit verschiedener Anzahl der Eingabemodelle.}
\label{QA}
\end{table}

\begin{figure}
\centering
\includegraphics[scale=0.5]{Abbildungen/recog_rate_new2.png}
\caption{Der Vergleich der Erkennungsquoten mit und ohne Verbesserung der Verwendung der Kandidaten. Alle drei Testbildstr�me werden mit jeweils 1 bzw. 2 Eingabemodellen betrachtet.}
\label{RR}
\end{figure}

Neben der Vergr��erung der Erkennungsquote steigt die durchschnittliche Laufzeit jedes Bildes deutlich an. Die Zunahme des Zeitaufwandes, die in der rechten Spalte der Tabelle~\ref{QA} aufgelistet wird, schwankt um 10 $ms$. D.h., dass diese Zunahme nicht von der durchschnittlichen Laufzeit der Erkennung stark beeinflusst wird, was die Konvergenz des Zeitaufwands unserer Verbesserung mit Kandidaten experimentell nachweist. Das entsprechende Balkendiagramm findet man in Abbildung~\ref{RT}.  

\begin{figure}[htbp]
\centering
\includegraphics[scale=0.5]{Abbildungen/recog_time_new2.png}
\caption{Der Vergleich des Zeitaufwandes mit und ohne Verbesserung der Verwendung der Kandidaten. Alle drei Testbildstr�me werden mit jeweils 1 bzw. 2 Eingabemodellen betrachtet.}
\label{RT}
\end{figure}

\subsubsection{2 Eingabeobjekte mit 2 Eingabemodellen}
Die Wirkung der Erkennungskandidaten vergr��ert sich, wenn zwei Eingabemodelle von dem Programm eingelesen werden und gleichzeitig zwei Objekte im Testbildstrom vorkommen. Wegen dem �berlappenden Vergleich zwischen den Eingabemodellen und Eingabeobjekten wird die falsche Erkennungsquote deutlich erh�ht. Au�erdem st�rt die unterschiedliche Qualit�t der Markenerkennung verschiedener Objekte auch die Wiedererkennung. In unserem Test ist es beispielsweise  h�ufig schwierig, das zweite Objekt st�ndig zu verfolgen, weil nicht genug Marken f�r das Objekt aus dem aktuellen Bild erkannt werden k�nnen. Dadurch nimmt die Genauigkeit der Korrespondenzuntersuchung und Orientierung weiterhin ab. Deshalb sind die historischen Erkennungsergebnisse f�r die Vorhersage bzw. Korrektur des aktuellen Erkennungsergebnisses  besonders wichtig, welches in den Kandidaten regelm��ig gespeichert wird. Die Tabelle~\ref{22} gibt den Vergleich der Erkennungsergebnisse mit und ohne die Verbesserung der Kandidaten wieder. Was unbedingt beobachtet werden soll, ist der 50\% Anstieg der Erkennungsquote f�r das zweite Objekt. Einige Screenshots werden in Abbildung~\ref{R2B} gezeigt.

\begin{table}[htbp]
\centering
\scalebox{0.635}{
\begin{tabular}{ r | c | c | c |}
\cline{2-4}
& Erkennungsquote erstes Objekts & Erkennungsquote zweites Objekts & Zeitaufwand ($ms$) \\
\hline
\multicolumn{1}{|r|}{Ohne die Verbesserung der Kandidaten} & 53.09\% & 29.06\% & 49.1496 \\
\hline
\multicolumn{1}{|r|}{Mit der Verbesserung der Kandidaten} & 94.53\% & 79.14\% & 59.9338 \\
\hline
\end{tabular}
}
\caption{Die statistischen Daten der Wiedererkennung, welche zwei Eingabeobjekte und zwei Eingabemodelle hat.}
\label{22}
\end{table}

\begin{figure}[htbp]
\centering
\includegraphics[scale=0.5]{Abbildungen/Recog2Boxes.png}
\caption{Die Visualisierung der Erkennungs- und Verfolgungsergebnisse.}
\label{R2B}
\end{figure}
\chapter{Zusammenfassung und Ausblick}
Diese Arbeit implementiert die auf Marken basierte Objekterkennung bzw. Verfolgung. Die Testobjekte werden in schwarz gef�rbt und mit wei�en Marken auf der Oberfl�che markiert. Eine PMD-Kamera beobachtet die ganze Szene von oben und liefert direkt die 3D-Daten. Das gesamte Programm kann in zwei Teilen, Lernen und Wiedererkennung, zusammengefasst werden. Im Lernprozess wird ein Fremdobjekt unter der Kamera gezeigt. Die Marken des Objekts werden von dem Programm erkannt und in einem Strukturgraph eingef�gt, was die r�umliche Struktur des Objekts beschreiben kann. Der Strukturgraph wird nach dem Lernen als eine VTK-Datei gespeichert. Wenn es mehr Objekte im Eingabebildstrom gibt, wird die Markenmenge f�r jedes Objekt zuerst segmentiert. Am Anfang der Wiedererkennungsphase werden die vorhandenen Strukturgraphen eingelesen, was als die Eingabemodelle der Wiedererkennung definiert wird. F�r jedes Eingabebild werden alle Marken zuerst erkannt, genau wie es beim Lernen durchgef�hrt wird. Die erkannten Marken werden dann in unterschiedlichen Kandidaten des Objekts aufgeteilt. Ein Eingabemodell ist genau dann wiedererkannt, wenn mindestens ein Kandidat existiert, der den gleichen Teilgraph zu diesem Eingabemodell erh�lt. Die aktuelle Orientierung und Lage des Objektes kann danach durch die Korrespondenzpunktepaare bestimmt werden.
\\
%\section{Verbesserungen}
%In dieser Arbeit werden viele vorhandene Ideen und Algorithmen verwendet. Bez�glich dieser Vorkenntnisse werden weiterhin viele Verbesserungen gemacht, wie z.B. die Helligkeitssteuerung, das neue Singul�rwertzerlegungsverfahren und die Bildsteuerung, usw. Dadurch k�nnen die Ergebnisse des Lernens und der Wiedererkennung stabiler und schneller erhalten werden. 
\\
Diese Arbeit verfolgt die Idee und einige Algorithmen von \cite{AJ05}. Aber wegen der verschiedenen Arbeitsumgebung ist die Implementierung in dieser Arbeit im viele Bereichen unterschiedlich: Nur eine 3D-Kamera beobachtet die ganz Szene und die Objekte; Die Aufl�sung der PMD Kamera ist nur $204 \times 204$, welche viel kleiner als das Stereokamerasystem in \cite{AJ05} ist; Der Abstand zwischen der Kamera und den Objekt ist viel gr��er. Dazu sollen die Markenerkennung bzw. die Korrespondenzuntersuchung verbessert werden, und die verrauschte Bilder aus der Bildsequenz l�schen lassen. Alle diese Verbesserungen wurden in dieser Arbeit sehr gut implementiert. Au�erdem werden den 3D Strukturgraphen nach dem Lernen erzeugt, welche nicht nur die Charakteristik sondern auch die geometrischen Informationen der Objekte enthalten. Mithilfe dieser Strukturgraphen k�nnen die Modelle f�r die Objekte in einem Mensch-Roboter-Kooperation System, wie z.B. MAROCO, einfacher dargestellt werden. Die dazu entsprechende VTK-Datei k�nnen auch durch andere Software weiter verarbeitet werden.
\\
\\
%\section{Vorhandene Probleme}
Nat�rlich k�nnte das Programm weiter entwickelt werden. Die weitere Verbesserungen k�nnen in zwei Bereichen, Stabilit�t und Zeitaufwand, zusammengefasst werden. 
%Einige Probleme sollen noch in der Zukunft gel�st werden. Die existierenden Probleme k�nnen in zwei Bereichen, Stabilit�t und Zeitaufwand, zusammengefasst werden.
\\
\\
\textbf{Stabilit�t}
\\
Die Gr��e und Position der Marken muss f�r aktuelles Programm exakt entworfen werden, damit die richtigen Erkennungsergebnisse erhalten werden k�nnen. Sonst ist das Objekt schwierig zu erkennen. Aus diesem Grund sind die Lange bzw. die Breite des Objektes stark beschr�nkt. Die Anordnung der Marken beeinflusst auch die Quote der erfolgreichen Erkennungen in der Wiedererkennungsphase. 
\\
\\
\textbf{Zeitaufwand}
\\
Wie in den Tabellen~\ref{LZ} bis \ref{22} gezeigt, erf�llt der Zeitaufwand aber nur teilweise die Echtzeitbedingung. In der Lernphase kann das Programm aber nur die Framerate bis zu 18.6 fps erreichen. In der Wiedererkennungsphase h�ngt die Framerate von der Anzahl der betrachteten Eingabeobjekte und Eingabemodelle ab. Wenn nur einen Eingabemodell betrachtet wird, kann das Programm im Durchschnitt 30 Bilder pro Sekunde bearbeiten. F�r zwei Eingabemodelle wird diese Zahl aber leider sofort auf 20 sinken.
\\
\\
\textbf{M�gliche Verbesserungsverfahren}
\\
Entweder die Stabilit�t oder der Zeitaufwand h�ngt stark von dem Detektor ab. Deshalb ist Verwendung eines neuen, besseren Detektors eine M�glichkeit, um das Programm im vorherigen Bereichen zu verbessern. Dieser Detektor soll bessere Genauigkeit und Stabilit�t haben. Im Idealfall sollen alle Marken eindeutig erkannt werden k�nnen, und die Ergebnisse unabh�ngig von der Helligkeit der Eingabebilder sein. Dadurch kann die Anzahl der Schleifen in der Helligkeitssteuerung stark reduziert und die Kombination der Merkmale in der Markenerkennung sogar komplett entfernt werden. Diese Vereinfachungen des Programmes k�nnen mehr als die H�lfte der gesamten Laufzeit einsparen, was in dem Kreisdiagramm in Abbildung~\ref{LZP} deutlich gezeigt wird. Die bessere Stabilit�t des Detektors fordert an, dass die Erkennungsergebnisse nicht empfindlich f�r die Abst�nde zwischen den Marken sein sollen. Damit k�nnen die Marken nicht genau mit den Regeln vom Abschnitt~\ref{Ma} angebracht werden.
\\
\\
Au�erdem kann der Berechnungsteil des Programmes mit CUDA neu formuliert werden, um damit die Laufzeit zu reduzieren. Der GPU hat eine st�rkere Berechnungsf�higkeit �ber die Gleitkommazahl im Vergleich zur CPU, und liefert gleichzeitig eine bessere M�glichkeit f�r die parallelen Arbeiten.

%Au�erdem ist die Umschreibung des Programms mit CUDA die andere M�glichkeit zur Abnahme der Laufzeit, wegen der st�rkeren Berechnungsf�higkeit �ber die Gleitkommazahl der GPU. 


%Das zweite Beschleunigungsverfahren ist gleich wie Vorher: ein besserer Markendetektor zu finden. Das Kreisdiagramm~\ref{LZP} zeigt deutlich, dass die Teilprogramme ,,CenSurE Detektor und Helligkeitssteuerung'' und ,,Markenerkennung'' mehr als Halb der gesamten Laufzeit brauchen. Diese beide Teile h�ngen von dem Markendetektor stark ab. Wenn ein besserer Detektor verwendet wird, kann die Anzahl der Schleifen in Helligkeitssteuerung stark reduziert. Die Kombination der Merkmalen in Markenerkennung, die der 1 zu 1 Korrespondenz zwischen der durch Detektor erkannten Merkmale und der realen Marken anpasst, kann aber auch komplett entfernt. Diese Vereinfachung der Programms k�nnen die Laufzeit erfolgreich erniedrigen.    

%Die Oberfl�chen der Zielobjekt werden mit retroreflektierenden Marker markiert.  Die totale Laufzeit kann in zwei Phasen zusammengefasst werden. In der Initialisierungsphase wird die Fremdobjekt unter der Kamera gezeigt und die Marker darauf sollen herausgekannt und im System gespeichert werden. Wenn es mehr Objekte gibt, wird eine Kalibrierung am Anfang durchgef�hrt. Mills in \cite{MN00} hat eine kompakte Segmentierung der Bewegung mithilfe des sogenannten \glqq Feature Interval Graph\grqq  dargestellt. \cite{AJ05} erweitert die Arbeit von Mills. Ein auf Pyramide basiertes Clustering-Verfahren wird statt des alten auf Dreiecke basierten Clustering-Verfahren vorgeschlagen. Nach der Initialisierungsphase wird das Objekt aus dem Gesichtsfeld der Kamera verschoben. Die Erkennungsphase f�ngt genau an, wenn das gleiche Objekt wieder unter der Kamera eingebracht wird. Die reflektierende Marker sollen nochmal gesammelt werden und ein von \cite{AJ05} repr�sentierter Teilgraph-Tracker wird dann implementiert, um die Objekt zu kennen und die Position zu bestimmen.         

\cleardoublepage

% -------------------------------------------------------------------
% Anhang:
% -------------------------------------------------------------------

\appendix

\renewcommand{\baselinestretch}{1.0}

% hier die einzelnen Kapitel des Anhangs einfuegen...

%\input{MathematischeHerleitungen}

% ...
% -------------------------------------------------------------------
% Literatur:
% -------------------------------------------------------------------
\begin{thebibliography}{A}

% Journal
\bibitem[Big\"un et al. 1991]{bigun91}
 	J.~Big�n, G.~Granlund, and J.~Wiklund: {\em  Multidimensional Orientation Estimation with Application to Texture Analysis and Optical Flow}.
	\newblock IEEE Trans. Pattern Analysis and Machine Intelligence 13:8 (1991) 775--790.

% Interner Bericht
\bibitem[Gerber 2004]{gerber04}
 	R.~Gerber: {\em  Neustrukturierung der Generierung von Text aus Ergebnissen der Bildfolgenauswertung.}
	\newblock Interner Bericht, Fassung vom 15. September 2004.

% Buch	
\bibitem[Golub \& Van Loan 1996]{golub96}
 	G.H.~Golub and C.F.~Van Loan: {\em  Matrix Computations.}
	\newblock Johns Hopkins University Press, Baltimore and London, 1996.

% Dissertation
\bibitem[Middendorf 2003]{middendorf03}
	M.~Middendorf: {\em Zur Auswertung lokaler Grauwertstrukturen.}
	\newblock Dissertation, Fakult�t f�r Informatik der Universit�t Karlsruhe (TH), Juli 2003.

% Konferenzbeitrag
\bibitem[Galvin et al. 1998]{galvin98}
 	G.~Galvin, B.~McCane, K.~Novins, D.~Mason, and S.~Mills: {\em  Recovering Motion Fields: An Analysis of Eight Optical Flow Algorithms}.
	\newblock J.N.~Carter and M.S.~Nixon (Eds.): Proceedings of the British Machine Vision Conference 1998, pp. 195-204, Southampton, UK. British Machine Vision Association 1998.

\end{thebibliography}

\bibliographystyle{geralpha}
% -------------------------------------------------------------------
% Stichwortverzeichnis (optional)
% -------------------------------------------------------------------
%\def\indexname{Stichwortverzeichnis}
%\printindex
%\begin{thebibliography}{99}
%\bibitem{asimo} H.~Partl:
%
%\end{thebibliography}
\end{document}
